\documentclass[solutions.tex]{subfiles}

\xtitle

\begin{document}
\maketitle
\begin{exercise}
Given a force that varies with time according to $F = 2t^2$,
and with the initial condition at time zero, $x(0) = \pi$, use
Aristotle's law to find $x(t)$ at all times.
\end{exercise}
Let us recall that Aristotle's law of motion is defined, for a
one-dimensional particle (otherwise, $F(t)$ and $x(t)$ would be
vector-values functions $\bm{F}(t)$ and $\bm{x}(t)$) earlier
in the book as:
\[
	\frac{d}{dt}x(t) = \frac{F(t)}{m}
\]
And that by integrating both sides, thanks to the fundamental
theorem of calculus\footnote{\url{
https://en.wikipedia.org/wiki/Fundamental\_theorem_of\_calculus}},
assuming the mass is constant over time, we obtain:
\[
	x(t) = \frac{1}{m} \int F(t)\,dt
\]
Which is our case, for $F(t) = 2t^2$, develops in:
\begin{align*}
	x(t) &= \frac{1}{m} \int 2t^2\,dt \\
	~ &= \frac{2}{3m}t^3+c,\,c\in\mathbb{R}
\end{align*}
The initial condition $x(0)=\pi$ implies that $c=\pi$, hence the position
at all time would be:
\[
	\boxed{x(t) = \frac{2}{3m}t^3+\pi} \qed
\]
\end{document}