\documentclass[solutions.tex]{subfiles}

\xtitle

\begin{document}
\maketitle
\begin{exercise} Determine the indefinite integral
of each of the following expressions by reversing the process
of differentiation and adding a constant.
\end{exercise}
Because this is the first integration exercise, we'll go slow.
We will "implement" the reversing of the process of differentiation
by applying the \textit{fundamental theorem of calculus}, on a few
previously established differentiation results; let's recall those:
\begin{equation*} \begin{aligned}
	\frac{d}{dt}t^n &&=\quad& nt^{n-1} \\
	\frac{d}{dt}\sin t &&=\quad& \cos t
\end{aligned} \end{equation*}

Let's start by integrating both sides of each equation:
\begin{equation*} \begin{aligned}
	\int\frac{d}{dt}t^n\,dt &&=\quad& \int nt^{n-1}\,dt \\
	\int\frac{d}{dt}\sin t\,dt &&=\quad& \int\cos t\,dt
\end{aligned} \end{equation*}
Let's then recall the second form of the \textit{fundamental
theorem of calculus} given in the book:
\[
	\int\frac{d}{dt}f\,dt = f(t) + c,\quad c\in\mathbb{R}
\]
So our previous equations can be rewritten as:
\begin{equation*} \begin{aligned}
	t^n + c &&=\quad& \int nt^{n-1}\,dt,&\quad c\in\mathbb{R} \\
	\sin t + c &&=\quad& \int\cos t\,dt,&\quad c\in\mathbb{R}
\end{aligned} \end{equation*}
Which are, to syntactical differences, the formulas given in
the book. In addition to those, we will also rely on the
\textit{linearity of the integration}, which essentially
is the combination of the \textit{sum rule for integration}
and \textit{multiplication by a constant rule for integration},
both being analogues of what we had for differentiation, and
which can be summed up by:
\begin{theorem}[linearity of integration]
\[
	\left(\forall (\alpha,\beta)\in\mathbb{R}^2\right), \left(\forall
	(\varphi,\psi)\in(C^0)^2\right) \quad
		\boxed{\int\alpha\varphi+\beta\psi = \alpha\int\varphi+\beta\int\psi}
\]
\end{theorem}
\begin{remark} $C^0$ refers to the class ("set") of continuous functions;
actually, mathematically-wise, it would suffice for the functions
to be "partially continuous" so as to be integrable; in the context of
physics, requiring them to be continuous is reasonable.
\end{remark}
\begin{remark} We're using the following "shortcut" notation:
\[
	\int\varphi = \int\varphi(t)\,dt
\]
or for a more involved expression:
\[
	\int\alpha\varphi+\beta\psi = \int\left(\alpha\varphi(t)+\beta\psi(t)\right)\,dt
\]
\end{remark}
\begin{proof}
We can establish this result, again to syntactical differences, for instance
through a similar process as we've just used for $t^n$ and $\cos$, that is, by
integrating differentiation results:
\begin{equation*} \begin{aligned}
	~ && \frac{d}{dt}(\alpha\varphi+\beta\psi)(t) &&=\quad& \alpha\varphi'(t)+\beta\psi'(t) \\
	\Leftrightarrow && \int\frac{d}{dt}(\alpha\varphi+\beta\psi)(t) &&=\quad&
		\int\alpha\varphi'(t)+\beta\psi'(t) dt \\
	\Leftrightarrow && (\alpha\varphi+\beta\psi)(t) &&=\quad&
		\int\alpha\varphi'(t)+\beta\psi'(t) dt \\
	\Leftrightarrow && \int\alpha\varphi'+\beta\psi' &&=\quad&
		\alpha\int\varphi'+\beta\int\psi'
\end{aligned} \end{equation*}
\end{proof}

\hr
$\bm{f(t) = t^4}$\ \\
This is a simple application of:
\[
	\int nt^{n-1}\,dt = t^n + c,\quad c\in\mathbb{R}
\]
with $n=4$; using the \textit{linearity of integration}:
\begin{equation*} \begin{aligned}
	~ & \int 5t^{5-1}\,dt &&=\quad& t^5 + c,\quad c\in\mathbb{R} \\
	\Leftrightarrow & \int t^4\,dt &&=\quad& \boxed{\frac15t^5 + c}
\end{aligned} \end{equation*}

\begin{remark} We can check the result by differentiating it
\end{remark}

\hr
$\bm{f(t) = \cos t}$\ \\
An even more direct application of the formulas established earlier:
\[
	\int\cos t\,dt = \boxed{\sin t + c},\quad c\in\mathbb{R}
\]
\begin{remark} Again, we can check the result using differentiation:
we know from earlier that the derivative of a constant is zero, that
of sine is cosine, and that the derivative of a sum is the sum
of the derivatives.
\end{remark}

\hr
$\bm{f(t) = t^2 - 2}$\ \\
Note that there's a special case for:
\[
	\int nt^{n-1}\,dt = t^n + c,\quad c\in\mathbb{R}
\]
when $n=1$:
\[
	\int 1\times t^{0}\,dt = \int dt = t^1 + c = t + c,\quad c\in\mathbb{R}
\]
More generally, by \textit{linearity of the integration}:
\[
	(\forall \alpha\in\mathbb{R})\quad \int\alpha\,dt =
		\alpha\int dt = \alpha t + c,\quad c\in\mathbb{R}
\]
And so we have:
\[
	\int t^2-2\,dt = \int t^2\,dt - 2\int dt = \boxed{\frac13t^3-2t+c},\quad c\in\mathbb{R}
\]
Which again is elementary to verify by differentiation.
\end{document}
