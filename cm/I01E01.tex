\documentclass[solutions.tex]{subfiles}

\xtitle

\begin{document}
\maketitle
\begin{exercise}
Using a graphic calculator or a program like \textit{Mathematica},
plot each of the following functions. See the next section if you
are unfamiliar with the trigonometric functions.

\begin{equation*} \begin{aligned}
	f(t) &&=\quad& t^4 + 3t^3 - 12t^2 + t - 6 \\
	g(x) &&=\quad& \sin x - \cos x \\
	\theta(\alpha) &&=\quad& e^\alpha + \alpha\ln\alpha \\
	x(t) &&=\quad& \sin^2t - \cos t
\end{aligned} \end{equation*}
\end{exercise}

\begin{figure}[H]
	\centering
	\begin{tikzpicture}
		\tikzmath{
			\xmin = -2;
			\xmax = 2;
			\ymin = -16;
			\ymax = 4;
		}
		\draw[->] (\xmin-1, 0) -- (\xmax+1, 0) node[right] {$t$};
		\draw[->] (0, \ymin-1) -- (0, \ymax+1) node[above] {$f(t)$};
		\draw[color=gray!30, dashed] (\xmin-1,\ymin-1) grid (\xmax+1,\ymax+1);
		\draw[scale=1, domain=\xmin:\xmax, smooth, variable=\t, blue] plot ({\t}, {\t^4+3*\t^3-12*\t^2+\t-6});
	\end{tikzpicture}
	\caption{$f(t) = t^4 + 3t^3 - 12t^2 + t - 6$}
\end{figure}

\begin{figure}[H]
	\centering
	\begin{tikzpicture}
		\tikzmath{
			\xmin = -7;
			\xmax = 7;
			\ymin = -1;
			\ymax = 1;
		}
		\draw[->] (\xmin-1, 0) -- (\xmax+1, 0) node[right] {$x$};
		\draw[->] (0, \ymin-1) -- (0, \ymax+1) node[above] {$g(x)$};
		\draw[color=gray!30, dashed] (\xmin-1,\ymin-1) grid (\xmax+1,\ymax+1);
		\draw[scale=1, domain=\xmin:\xmax, smooth, variable=\x, samples=100, blue] plot ({\x}, {sin(\x r)-cos(\x r)});
	\end{tikzpicture}
	\caption{$g(x) = \sin x - \cos x$}
\end{figure}

\begin{figure}[H]
	\centering
	\begin{tikzpicture}
		\tikzmath{
			\xmin = 0.00001;
			\xmax = 2;
			\ymin = 0;
			\ymax = 10;
		}
		\draw[->] (\xmin-1, 0) -- (\xmax+1, 0) node[right] {$\alpha$};
		\draw[->] (0, \ymin-1) -- (0, \ymax+1) node[above] {$\theta(\alpha)$};
		\draw[color=gray!30, dashed] (\xmin-1,\ymin-1) grid (\xmax+1,\ymax+1);
		\draw[scale=1, domain=\xmin:\xmax, smooth, variable=\x, blue] plot ({\x}, {exp(\x)+\x*ln(\x)});
	\end{tikzpicture}
	\caption{$\theta(\alpha) = e^\alpha + \alpha\ln\alpha$}
\end{figure}

\begin{figure}[H]
	\centering
	\begin{tikzpicture}
		\tikzmath{
			\xmin = -7;
			\xmax = 7;
			\ymin = -2;
			\ymax = 2;
		}
		\draw[->] (\xmin-1, 0) -- (\xmax+1, 0) node[right] {$t$};
		\draw[->] (0, \ymin-1) -- (0, \ymax+1) node[above] {$x(t)$};
		\draw[color=gray!30, dashed]
			(\xmin-1,\ymin-1) grid (\xmax+1,\ymax+1);
		\draw[scale=1, domain=\xmin:\xmax, smooth, samples=100, variable=\t, blue]
			plot ({\t}, {sin(\t r)^2 - cos(\t r)});
	\end{tikzpicture}
	\caption{$x(t) =  \sin^2t - \cos t$}
\end{figure}

\begin{remark} All those plots were created using TiKz (with \LaTeX\ then).
For instance, here's the code for the last plot:

\begin{verbatimtab}[4]
\begin{figure}[H]
	\centering
	\begin{tikzpicture}
		\tikzmath{
			\xmin = -7;
			\xmax = 7;
			\ymin = -2;
			\ymax = 2;
		}
		\draw[->] (\xmin-1, 0) -- (\xmax+1, 0) node[right] {$t$};
		\draw[->] (0, \ymin-1) -- (0, \ymax+1) node[above] {$x(t)$};
		\draw[color=gray!30, dashed]
			(\xmin-1,\ymin-1) grid (\xmax+1,\ymax+1);
		\draw[scale=1, domain=\xmin:\xmax, smooth, samples=100, variable=\t, blue]
			plot ({\t}, {sin(\t r)^2 - cos(\t r)});
	\end{tikzpicture}
	\caption{$x(t) =  \sin^2t - \cos t$}
\end{figure}
\end{verbatimtab}
\end{remark}
\end{document}
