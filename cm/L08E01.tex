\documentclass[solutions.tex]{subfiles}

\xtitle

\begin{document}
\maketitle
\begin{exercise}
Start with the Lagrangian $\dfrac{m\dot{x}^2}{2}-\dfrac{k}{2}x^2$ and
show that if you make the change in variables $q=(k m )^{1/4}x$, the Lagrangian
has the form of Eq. $(14)$. What is the connection among $k$, $m$
and $\omega$ ?
\end{exercise}
Let's recall some context. We're in the case of a the
harmonic oscillator, covered in-depth in
\href{https://tales.mbivert.com/on-the-harmonic-oscillator/}{L03E04}. \\

More precisely, the authors just shown us how the harmonic
oscillator can be used as an approximation in the case of
an equilibrium state that is slightly disturbed. Physically,
you can for instance consider the case of a pendulum: if the
oscillations are kept small, then the mass of the pendulum
will describe something that \textit{locally} looks like
the bottom of a quadratic polynomial (e.g. $ax^2+bx+c = 0$).
This is similar to how the derivative is a local linear
approximation, except things jiggle a little more,
so to speak. \\

Eq. $(14)$ of the book refers to the following Lagrangian:
\[
	L = \frac{1}{2\omega}\dot{q}^2 - \frac{\omega}{2}q^2
\]

Now let's simply perform the required change of variable.
Both $m$ and $k$ are constants, so $\dot{q} = (km)^{1/4}\dot{x}$,
and:

\[
	L = \frac{1}{2\omega}\sqrt{km}\dot{x}^2 - \frac{\omega}{2}\sqrt{km}x^2
\]

Finally, if we can try to identify the previous expression with the
one we're supposed to find, by finding a relation between $\omega$, $k$
and $m$. We have respectively for each term:

\[ m = \frac{\sqrt{km}}\omega;\qquad k = \omega\sqrt{km} \]
\[
	\Leftrightarrow
	\omega = \frac{\sqrt{k}\sqrt{m}}{\sqrt{m}^2};\qquad
	\omega = \frac{\sqrt{k}^2}{\sqrt{k}\sqrt{m}}
\]
\[
	\Leftrightarrow
	\boxed{\omega = \sqrt{\frac{k}m}}
\]

So we can consistently identify both expression by defining $\omega$
as previously stated. Such an $\omega$ furthermore matches the usual
definition we have in a harmonic oscillator setting.

\end{document}
