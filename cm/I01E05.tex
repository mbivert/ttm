\documentclass[solutions.tex]{subfiles}

\xtitle

\begin{document}
\maketitle
\begin{exercise} Determine which pair of vectors
are orthogonal. $(1,1,1), (2,-1,3), (3,1,0), (-3,0,2)$.
\end{exercise}
This is again an immediate application of dot product formula.
We'll respectively name the vectors $\bm{A}$, $\bm{B}$, $\bm{C}$
and $\bm{D}$; let's start by plotting them:
\begin{figure}[H]
	\centering
	\tdplotsetmaincoords{60}{120}
	\begin{tikzpicture}[scale=2,tdplot_main_coords]
		\tikzmath{
			\xmin = -4.5;
			\xmax = 3;
			\ymin = -3.5;
			\ymax = 2;
			\zmin = -1;
			\zmax = 3;
			\Ax = 1;
			\Ay = 1;
			\Az = 1;
			\Bx = 2;
			\By = -1;
			\Bz = 3;
			\Cx = 3;
			\Cy = 1;
			\Cz = 0;
			\Dx = -3;
			\Dy = 0;
			\Dz = 2;
		}

		\coordinate (orig) at (0,0,0);
		\coordinate (A) at (\Ax, \Ay, \Az);
		\coordinate (B) at (\Bx, \By, \Bz);
		\coordinate (C) at (\Cx, \Cy, \Cz);
		\coordinate (D) at (\Dx, \Dy, \Dz);

		\draw[->] (orig) -- (1,0,0) node[anchor=north east,xshift=10]{$\uvec{x}$};
		\draw[->] (orig) -- (0,1,0) node[anchor=north west]{$\uvec{y}$};
		\draw[->] (orig) -- (0,0,1) node[anchor=south,xshift=10]{$\uvec{z}$};

		\draw[->,dashed] (\xmin,0,0) -- (\xmax,0,0) node[anchor=north west]{$x$};
		\draw[->,dashed] (0,\ymin,0) -- (0,\ymax,0) node[anchor=north west]{$y$};
		\draw[->,dashed] (0,0,\zmin) -- (0,0,\zmax) node[anchor=north west]{$z$};

		\draw[->] (orig) -- (A) node [yshift=10]{$\bm{A}$};
		\draw[->] (orig) -- (B) node [xshift=10]{$\bm{B}$};
		\draw[->] (orig) -- (C) node [xshift=10]{$\bm{C}$};
		\draw[->] (orig) -- (D) node [xshift=10]{$\bm{D}$};

		\draw[dashed,gray] (0,0,0) -- (\Ax,\Ay,0);
		\draw[dashed,gray] (\Ax,\Ay,0) -- (A);
		\draw[dashed,gray]         (A) -- (0,0,\Az) node [xshift=10,yshift=-3]{$\Az$};
		\draw[dashed,gray] (\Ax,\Ay,0) -- (0,\Ay,0) node [yshift=12,xshift=-2]{$\Ay$};
		\draw[dashed,gray] (\Ax,\Ay,0) -- (\Ax,0,0) node [xshift=10,yshift=-3]{$\Ax$};

		\draw[dashed,gray] (0,0,0) -- (\Bx,\By,0);
		\draw[dashed,gray] (\Bx,\By,0) -- (B);
		\draw[dashed,gray]         (B) -- (0,0,\Bz) node [xshift=10,yshift=-3]{$\Bz$};
		\draw[dashed,gray] (\Bx,\By,0) -- (0,\By,0) node [yshift=12,xshift=-2]{$\Ay$};
		\draw[dashed,gray] (\Bx,\By,0) -- (\Bx,0,0) node [xshift=10,yshift=-3]{$\Bx$};

		\draw[dashed,gray] (0,0,0) -- (\Cx,\Cy,0);
		\draw[dashed,gray] (\Cx,\Cy,0) -- (C);
		\draw[dashed,gray]         (C) -- (0,0,\Cz) node [xshift=10,yshift=-3]{$\Cz$};
		\draw[dashed,gray] (\Cx,\Cy,0) -- (0,\Cy,0) node [yshift=12,xshift=-2]{$\Ay$};
		\draw[dashed,gray] (\Cx,\Cy,0) -- (\Cx,0,0) node [xshift=10,yshift=-3]{$\Cx$};

		\draw[dashed,gray] (0,0,0) -- (\Dx,\Dy,0);
		\draw[dashed,gray] (\Dx,\Dy,0) -- (D);
		\draw[dashed,gray]         (D) -- (0,0,\Dz) node [xshift=10,yshift=-3]{$\Dz$};
		\draw[dashed,gray] (\Dx,\Dy,0) -- (0,\Dy,0) node [yshift=12,xshift=-2]{$\Ay$};
		\draw[dashed,gray] (\Dx,\Dy,0) -- (\Dx,0,0) node [xshift=10,yshift=-3]{$\Dx$};
	\end{tikzpicture}
\end{figure}
Right before this exercise, the authors wrote:
\begin{quote}
An important property of the dot product is that it is zero
if the vectors are \textit{orthogonal}.
\end{quote}
Let us recall the "components-based" dot product formula we'll
be using:
\begin{equation*} \begin{aligned}
	\bm{u}\cdot\bm{v} &&=\quad& u_xv_x + u_yv_y + u_zv_z
\end{aligned} \end{equation*}
Then, it's just a matter of crunching numbers:
\begin{equation*} \begin{aligned}
	\bm{A}\cdot\bm{B} &&=\quad& A_xB_x + A_yB_y + A_zB_z &&&
		\bm{A}\cdot\bm{C} &&=\quad& A_xC_x + A_yC_y + A_zC_z \\
	~ &&=\quad& (1\times 2) + (1\times (-1)) + (1\times 3) &&&
		~ &&=\quad& (1\times 3) + (1\times 1) + (1\times 0) \\
	~ &&=\quad& \boxed{4} &&&
		~ &&=\quad& \boxed{4} \\
	\bm{A}\cdot\bm{D} &&=\quad& A_xD_x + A_yD_y + A_zD_z &&&
		\bm{B}\cdot\bm{C} &&=\quad& B_xC_x + B_yC_y + B_zC_z \\
	~ &&=\quad& (1\times (-3)) + (1\times 0) + (1\times 2) &&&
		~ &&=\quad& (2\times 3) + (-1\times 1) + (3\times 0) \\
	~ &&=\quad& \boxed{-1} &&&
		~ &&=\quad& \boxed{5} \\
	\bm{B}\cdot\bm{D} &&=\quad& B_xD_x + B_yD_y + B_zD_z &&&
		\bm{D}\cdot\bm{C} &&=\quad& D_xC_x + D_yC_y + D_zC_z \\
	~ &&=\quad& (2\times (-3)) + (-1\times 0) + (3\times 2) &&&
		~ &&=\quad& (-3\times 3) + (0\times 1) + (2\times 0) \\
	~ &&=\quad& \boxed{0} &&&
		~ &&=\quad& \boxed{-9} \\
\end{aligned} \end{equation*}
Hence, the only two orthogonal vectors are $\bm{B}$ and $\bm{D}$, or
$(2,-1,3)$ and $(-3,0,2)$.
\end{document}
