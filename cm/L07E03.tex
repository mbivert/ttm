\documentclass[solutions.tex]{subfiles}

\title{L07E03}

\begin{document}
\maketitle
\begin{exercise}
Show that the combination $a q_1 + b q_2$, along with the Lagrangian,
is invariant under Equations $(7)$.
\end{exercise}
Let us first recall the equations for the potential (Equations $(3)$):
\[
	V(q_1, q_2) = V(a q_1 - b q_2)
\]

Which is meant to be considered in the case of the following
Lagrangian:
\begin{align}
	L = \frac{1}{2}(\dot{q_1}^2+\dot{q_2}^2) - V(a q_1-b q_2)
	\label{eqn:l07e03:lagrangian}
\end{align}

Finally, the "Equations $(7)$" relate to the following change of
coordinates:
\begin{align}
	q_1 &\rightarrow q_1 - b\delta \nonumber\\
	q_2 &\rightarrow q_2 + a\delta
	\label{eqn:l07e03:cov}
\end{align}

\begin{remark} There are typos around here in the book. In my printed
version, it is as previously described, but in an online version, it is given
by (mind the signs):
\begin{align*}
	q_1 &\rightarrow q_1 + b\delta \nonumber\\
	q_2 &\rightarrow q_2 - a\delta
\end{align*}
yet in that same online version, the potential is said to depend on $a q_1 + b q_2$
in accordance to Equations $(3)$, but said Equations $(3)$ actually make
it depend on $a q_1 - b q_2$! \\

To summarize, with a $V(a q_1 + b q_2)$, the two previous transformations
will keep the Lagrangian unchanged.

But with a $V(a q_1 - b q_2)$, none of the previous transformations will keep
the Lagrangian; those two will:
\begin{align}
	q_1 &\rightarrow q_1 - b\delta
	& q_1 &\rightarrow q_1 + b\delta \nonumber\\
	q_2 &\rightarrow q_2 - a\delta
	& q_2 &\rightarrow q_2 + b\delta \nonumber\\
	\label{eqn:l07e03:cov2}
\end{align}

In what follows, we will arbitrarily assume a $V(a q_1 - b q_2)$, and,
say, the first transformation of \eqref{eqn:l07e03:cov2}.
\end{remark}

Assuming $a$, $b$ and $\delta$ are time-invariant, it follows that
$\dot{q_1}$ and $\dot{q_2}$ are unchanged by this transformation, hence
\begin{align*}
	\dot{q_i} &\rightarrow \dot{q_i}\\
	\dot{q_1}^2+\dot{q_2}^2 &\rightarrow \dot{q_1}^2+\dot{q_2}^2
\end{align*}

Injecting \eqref{eqn:l07e03:cov} into \eqref{eqn:l07e03:lagrangian} gives
us the following Lagrangian:
\begin{align*}
	L &= \frac{1}{2}(\dot{q_1}^2+\dot{q_2}^2)
		- V(a(q_1-b\delta) - b(q_2-a\delta))\\
	~ &= \frac{1}{2}(
		\dot{q_1}^2+\dot{q_2}^2
	) - V(a q_1 - b q_2)
\end{align*}

We can see that indeed, the Lagrangian is unchanged; because the
$\dot{q_i}$ are also unchanged, we would derive the exact
same equation of motions as we did for the previous exercise.

\end{document}
