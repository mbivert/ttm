\documentclass[solutions.tex]{subfiles}

\xtitle

\begin{document}
\maketitle
\begin{exercise}
Show that in the $x, y$ plane, the solution to Eq. $(25)$
and the solution to Eq. $(26)$ are a circular orbit with the
center of the orbit being anywhere on the plane. Find the radius
of the orbit in terms of the velocity.
\end{exercise}

Let's recall Eq. $(25)$ and Eq. $(26)$:
\[
	a_y = -\frac{eb}{mc}v_x;\qquad
	a_x = \frac{eb}{mc}v_y
\]

Let's rewrite them with dots instead:

\[
	\ddot{y} = -\frac{eb}{mc}\dot{x};\qquad
	\ddot{x} = \frac{eb}{mc}\dot{y}
\]

First, note that each equation was obtained from a different gauge.
But, as the Hamiltonian is gauge-invariant, the equation of motions
aren't affected by a gauge shift. So they do both describe the
motion we're interested in and can be "combined". \\

Then, we've been using similar-looking differential equations (for
instance when considering the harmonic oscillator), and we know they
were solved by a sine/cosine-like function. So we're going to make
some guess and calibrate a sine to have it work. We'll then have
an expression for both $x(t)$ and $y(t)$, and if we can find a $r$
such as:

\[
	(x(t)-a)^2 + (y(t)-b)^2 = r^2
\]

We'll know that our coordinate function can draw a circle of radius $r$,
centered at $(a, b)$. \\

Alright so let's say $\dot{x}$ is a $\sin$ function; then

\begin{itemize}
	\item $x$ would be a $-\sin$;
	\item $\ddot{x}$ would be a $\sin$;
	\item $\ddot{y}$ would also be a $\cos$;
	\item $\dot{y}$ would be a $-\sin$;
	\item and $y$ would be a $-\cos$.
\end{itemize}

Looks promising, as $x^2+y^2$ would involve a $\cos^2+\sin^2=1^2$.
Now we'd just have to "caliber" the $\sin$ properly. Say, if it's
a $\sin(\omega t)$, then by repeated differentiation, the $\omega$
would become a multiplicative factor, outside of the $\sin$. So,
the following feels reasonable:
\[
	\omega = \frac{eb}{mc}
\]

Let's see what would happen to the first equation, if we start
with a $\dot{x}=\sin(\omega t)$:

\[ \ddot{y} = -\frac{eb}{mc}\dot{x} = -\omega\sin(\omega t) \]

From which we can derive both:
\[
	\dot{y} = \cos(\omega t);\qquad
	\ddot{x} = \omega\cos(\omega t)
\]

And those two do validate the second equation:
\[
	\ddot{x} = \frac{eb}{mc}\dot{y}
\]

So we've found a solution; note that we can shift both
component by arbitrary constants $a$ and $b$ without
affecting their correctness:
\[
	\boxed{
		y(t) = \frac1\omega\sin(\omega t) + a;\qquad
		x(t) = -\frac1\omega\cos(\omega t) + b
	}
\]

And $\omega$ was well-named, as it correspond to the
angular velocity. Does it describe an orbit around a point $(a, b)$?
Let's find out:
\[
	(x(t)-a)^2 + (y(t)-b)^2 =
	\frac1{\omega^2}\Bigl(\sin^2(\omega t)+\cos^2(\omega t)\Bigr)
	= \Bigl(\frac1\omega\Bigr)^2
\]

So they do draw a circle of radius:
\[
	\boxed{r = \frac1\omega = \frac{mc}{eb}}
\]

\end{document}
