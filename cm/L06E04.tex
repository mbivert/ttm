\documentclass[solutions.tex]{subfiles}

\xtitle

\begin{document}
\maketitle
\begin{exercise}
Work out George's Lagrangian and Euler-Lagrange equations
in polar coordinates.
\end{exercise}
As always, let us recall the general form of Euler-Lagrange equations
for a configuration space of size $n\in\mathbb{N}$:
$(\forall i \in\llbracket 1, n\rrbracket)$,
\begin{align}
	\frac{d}{dt}\left(\frac{\partial}{\partial\dot{x_i}}L\right)
		= \frac{\partial}{\partial x_i}L
	\label{eqn:l06e04:euler-lagrange}
\end{align}
The original Lagrangian $L$ in our case is defined by the Eq. $(10)$
of this chapter as:
\begin{align}
	L = \frac{m}{2}\left(\dot{x}^2+\dot{y}^2\right)
	\label{eqn:l06e04:lagrangian}
\end{align}
After the following coordinate shift (Eq. $(9)$ of the book):
\begin{align}
	x &= X\cos(\omega t) + Y\sin(\omega t)& y &= -X \sin(\omega t)+Y\cos(\omega t)
	\label{eqn:l06e04-1st-shift}
\end{align}
We obtained this Lagrangian (Eq. $(12)$ of the book):
\begin{align}
	L = \frac{m}{2}(\dot{X}^2+\dot{Y}^2)
	  + \frac{m\omega^2}{2}(X^2+Y^2)
	  + m\omega(\dot{X}Y-\dot{Y}X)
	\label{eqn:l06e04:lagrangian-1st-shift}
\end{align}

For the current exercice, the coordinate shift to polar equations is:
\begin{align}
	X &= R \cos\theta & Y &= R \sin\theta
	\label{eqn:l06e04:coordinate-shift}
\end{align}
Where, implicitely, both $R$ and $\theta$ are, as $X$ and $Y$, functions of
time. \\

Now, we have at least two ways of solving this exercice:
\begin{enumerate}
	\item Either perform the coordinate shit \eqref{eqn:l06e04:coordinate-shift}
	in \eqref{eqn:l06e04:lagrangian-1st-shift}: this will be a tedious
	but very similar developement as the one performed in the book to obtain
	\eqref{eqn:l06e04:lagrangian-1st-shift} from \eqref{eqn:l06e04:lagrangian}
	and \eqref{eqn:l06e04-1st-shift};
	\item or perform this new coordinate shift \eqref{eqn:l06e04:coordinate-shift}
	directly in the first coordinate shift \eqref{eqn:l06e04-1st-shift}, and
	work from the first Lagrangian \eqref{eqn:l06e04:lagrangian} instead: some
	trigonometric identities are likely to ease at least the beginning of the
	work here.
\end{enumerate}
We will try both aproaches, and expect to find the exact same solutions in
the end. \\
\hr
\textbf{First approach}\\
Let use start by computing the time derivative of $X$ and $Y$ as defined
by \eqref{eqn:l06e04:coordinate-shift}, using the both the product
\footnote{\url{https://en.wikipedia.org/wiki/Product\_rule}}
and the chain rule
\footnote{\url{https://en.wikipedia.org/wiki/Chain\_rule}}:
\begin{align}
	\dot{X} &= \dot{R}\cos\theta - R\dot\theta\sin\theta
	& \dot{Y} &= \dot{R}\sin\theta + R\dot\theta\cos\theta
	\label{eqn:l06e04:coordinate-shift-speeds}
\end{align}

\begin{remark} For clarity, as a similar developement will happen
a few times, let's go into details for the first one:
the product rule for two functions $u$ and $v$ of a single variable, of
respective derivatives $u'$ and $v'$ is
\[ (u v)' = u'v+uv' \]
Now the chain rule is, again for the same kind of functions:
\[ (u(v(x))' = v'(x)u'(v(x)) \]
In the present case, we have a $X(t)$ defined as the product of two
functions: $X(t) = R(t)\cos(\omega(t))$, where the second one is
itself a composition of two functions $\cos(\omega(t))$. Hence,
by applying first the product rule, we obtain:
\[
	X'(t) = R'(t)\cos(\omega(t)) + R(t)\left(\cos(\omega(t))\right)'
\]
While the chain rule gives us:
\[
	(\cos(\omega(t)))' = -\omega'(t)\sin(\omega(t))
\]
Hence,
\[
	X'(t) = R'(t)\cos(\omega(t)) - R(t)\omega'(t)\sin(\omega(t))
\]
\end{remark}
Now, our goal will be to plug \eqref{eqn:l06e04:coordinate-shift} and
\eqref{eqn:l06e04:coordinate-shift-speeds} into the Lagrangian
\eqref{eqn:l06e04:lagrangian-1st-shift} obtained after the first
coordinate shift, but doing that transformation at once will gives
a difficult to read equation. Instead, we'll work in smaller steps,
simplying our results using trigonometric identities along the way. \\

Let us start with $X^2+Y^2$, using the fact that
$\sin^2\theta+\cos^2\theta = 1$:
\begin{align}
	X^2+Y^2 &= R^2\cos^2\theta + R^2\sin^2\theta\nonumber\\
	~ &= R^2(\cos^2\theta + \sin^2\theta) \nonumber\\
	~ &= R^2
	\label{eqn:l06e04:sum-squares}
\end{align}

Now for $\dot{X}^2+\dot{Y}^2$, using the same trigonometric
identity:
\begin{align*}
	\dot{X}^2 &= (\dot{R}\cos\theta-R\dot\theta\sin\theta)^2 &
	\dot{Y}^2 &= (\dot{R}\sin\theta+R\dot\theta\cos\theta)^2\\
	~ &= \dot{R}^2\cos^2\theta
		-2R\dot{R}\dot\theta\cos\theta\sin\theta
		+R^2\dot\theta^2\sin^2\theta &
	~ & = \dot{R}^2\sin^2\theta
		+2R\dot{R}\dot\theta\sin\theta\cos\theta
		+R^2\dot\theta^2\cos^2\theta
\end{align*}
\begin{align}
	\dot{X}^2+\dot{Y}^2 &= \dot{R}^2(\cos^2\theta+\sin^2\theta)
		+R^2\dot\theta^2(\cos^2\theta+\sin^2\theta) \nonumber\\
	~ &= \dot{R}^2 + R^2\dot\theta^2
	\label{eqn:l06e04:sum-squares-dot}
\end{align}
Finally, for $\dot{X}Y-\dot{Y}X$:
\begin{align*}
	\dot{X}Y &= (\dot{R}\cos\theta-R\dot\theta\sin\theta)R\sin\theta &
	\dot{Y}X &= (\dot{R}\sin\theta+R\dot\theta\cos\theta)R\cos\theta \\
	~ &= R\dot{R}\cos\theta\sin\theta-R^2\dot\theta\sin^2\theta &
	~ &= R\dot{R}\cos\theta\sin\theta+R^2\dot\theta\cos^2\theta
\end{align*}
\begin{align}
	\dot{X}Y-\dot{Y}X &= -R^2\dot\theta(\sin^2\theta+\cos^2\theta)\nonumber\\
	~ &= -R^2\dot\theta
	\label{eqn:l06e04:diff-mixed}
\end{align}
Now we're ready to plug \eqref{eqn:l06e04:sum-squares},
\eqref{eqn:l06e04:sum-squares-dot} and \eqref{eqn:l06e04:diff-mixed}
into \eqref{eqn:l06e04:lagrangian-1st-shift}:
\begin{align}
	L &= \boxed{\frac{m}{2}(\dot{R}^2 +R^2\dot\theta^2)
		+\frac{m\omega^2}{2}R^2-m\omega R^2\dot\theta}
	\label{eqn:l06e04:lagrangian-2nd-shift}
\end{align}
Now, let's compute the partial derivatives of our new Lagrangian:
\begin{align}
	\frac{\partial}{\partial\dot{R}}L &=
		\frac{\partial}{\partial\dot{R}}
			\left(\frac{m}{2}\dot{R}^2\right)
	&  \frac{\partial}{\partial R}L &=
		\frac{\partial}{\partial R}
			\left(\frac{m}{2}R^2\dot\theta^2+
			\frac{m\omega^2}{2}R^2 - m\omega R^2\dot\theta\right)\nonumber\\
	~ &= m\dot{R}
	& ~ &= (\dot\theta^2+\omega^2-2\omega\dot\theta)mR\nonumber\\
	~ & ~
	& ~ &= (\dot\theta-\omega)^2mR\nonumber\\
	\frac{\partial}{\partial\dot\theta}L &=
		\frac{\partial}{\partial\dot\theta}
			\left(\frac{m}{2}R^2\dot\theta^2-m\omega R^2\dot\theta\right)
	& \frac{\partial}{\partial\theta}L &= 0\nonumber\\
	~ &= mR^2(\dot\theta-\omega)
	& ~ & ~
	\label{eqn:l06e04:partial-derivs}
\end{align}
And from there, plug \eqref{eqn:l06e04:partial-derivs} in
Euler-Lagrange \eqref{eqn:l06e04:euler-lagrange} to derive the
equations of motion (again for the second one, we use a combination
of the product and chain rules for derivatives):
\begin{align*}
	~ & \frac{d}{dt}\left(m\dot{R}\right) = (\dot\theta-\omega)^2mR
	& ~ & \frac{d}{dt}\left(mR^2(\dot\theta-\omega)\right) = 0\\
	\Leftrightarrow & \boxed{\ddot{R} = (\dot\theta-\omega)^2R}
	& ~ \Leftrightarrow & m\left((\dot\theta-\omega)2\dot{R}R+R^2\ddot\theta\right)=0 \\
	~ & ~
	& ~ \Leftrightarrow & \boxed{R\ddot\theta = (\omega-\dot\theta)2\dot{R}} \qed
\end{align*}
\hr
\textbf{Second approach}\\
We'll now try to see if we can get a cleaner derivation, hopefully
with the same results, by combining the two coordinate shifts
\eqref{eqn:l06e04-1st-shift} and \eqref{eqn:l06e04:coordinate-shift}
first, and then rely on the original Lagrangian \eqref{eqn:l06e04:lagrangian}.\\

The combined coordinate shift is:
\begin{align*}
	x &= R\cos\theta\cos(\omega t) + R\sin\theta\sin(\omega t)\\
	y &= -R\cos\theta\sin(\omega t) + R\sin\theta\cos(\omega t)
\end{align*}
We have the four following trigonometric identities\footnote{
\url{https://en.wikipedia.org/wiki/List\_of\_trigonometric\_identities\#Product-to-sum\_and\_sum-to-product\_identities}}:
\begin{align*}
	cos\theta\cos\varphi &= \frac{\cos(\theta-\varphi)+cos(\theta+\varphi)}{2}
	& sin\theta\sin\varphi &= \frac{\cos(\theta-\varphi)-cos(\theta+\varphi)}{2}\\
	cos\theta\sin\varphi &= \frac{\sin(\theta+\varphi)-sin(\theta-\varphi)}{2}
	& sin\theta\cos\varphi &= \frac{\sin(\theta+\varphi)+sin(\theta-\varphi)}{2}
\end{align*}
Hence the coordinate shift can be rewritten:
\begin{align}
	x &= R\cos(\theta-\omega t)\nonumber\\
	y &= R\sin(\theta-\omega t)
	\label{eqn:l06e04:combined-shift}
\end{align}
To inject it in the original Lagrangian \eqref{eqn:l06e04:lagrangian},
we need to compute $\dot{x}^2+\dot{y}^2$. For the derivation, as previously,
we'll rely on a combination of the product/chain rule; we'll note
$\varphi=\theta-\omega t$:
\begin{align*}
	\dot{x} &= \dot{R}\cos\varphi - R(\dot\theta-\omega)\sin\varphi \\
	\dot{y} &= \dot{R}\sin\varphi + R(\dot\theta-\omega)\cos\varphi
\end{align*}
\begin{align*}
	\dot{x}^2 &= \dot{R}^2\cos^2\varphi
		- 2R\dot{R}(\dot\theta-\omega)\cos\varphi\sin\varphi
		+ R^2(\dot\theta-\omega)^2\sin^2\varphi \\
	\dot{y}^2 &= \dot{R}^2\sin^2\varphi
		+ 2R\dot{R}(\dot\theta-\omega)\cos\varphi\sin\varphi
		+R^2(\dot\theta-\omega)^2\cos^2\varphi
\end{align*}
Hence the Lagrangian becomes, again using the pythagorean trigonometric
identity $\cos^2\theta+\sin^2\theta = 1$:
\begin{align*}
	L &= \frac{m}{2}\left(\dot{x}^2+\dot{y}^2\right) \\
	~ &= \boxed{\frac{m}{2}\left(\dot{R}^2+R^2(\dot\theta-\omega)^2\right)} \\
	~ &= \frac{m}{2}\left(\dot{R}^2+R^2(\dot\theta^2-2\dot\theta\omega+\omega^2)\right) \\
	~ &= \frac{m}{2}(\dot{R}^2+R^2\dot\theta^2)+\frac{m}{2}R^2\omega^2-m\omega R^2\dot\theta
\end{align*}
Which is the same Lagrangian we had before in
\eqref{eqn:l06e04:lagrangian-2nd-shift}, from which
we would obviously derive the exact same equation of motion. $\qed$.

\begin{remark} As expected, the derivation is overall less tedious,
but only because the complexity is now hidden behind the
trigonometric identities.
\end{remark}
\begin{remark} A little later in the book, a solution to this
exercice is proposed: it starts with this Lagrangian:
\[
	L = \frac{m}{2}\left(\dot{r}^2+r^2\dot\theta^2\right)
\]
Which is exactly our Lagrangian, however assuming for some
reason that $\omega = 0$. From which follows the same equation
of motions, again with the same assumption regarding $\omega$:
\begin{align*}
	\ddot{r} &= r\dot\theta^2 \\
	\frac{d}{dt}\left(mr^2\dot\theta\right) &= 0
\end{align*}
Let's remind ourselves that $\omega$ represents the rotation of the polar
coordinate system of the present exercice, a rotation which won't
exist for a general polar coordinate system, hence the reason we have
$\omega=0$ in the general case.
\end{remark}
\end{document}
