\documentclass[solutions.tex]{subfiles}

\xtitle

\begin{document}
\maketitle
\begin{exercise}
Show that Eq. $(6)$ is just another form of Newton's
equation of motion $F_i = m_i \ddot{x_i}$.
\end{exercise}
Where Eq. $(6)$ are the following set of equation, defined for
all $i\in\llbracket 1, n\rrbracket$:
\begin{align}
	\frac{d}{dt}\left(\frac{\partial}{\partial\dot{x_i}}L\right)
		= \frac{\partial}{\partial x_i}L
	\label{eqn:l06e02:euler-lagrange}
\end{align}

\begin{remark} This exercice is simply a generalization
of the previous exercice (L06E01) to a configuration space
of size $n\in\mathbb{N}$.
\end{remark}

Then again, let us recall the Lagrangian defined slightly earlier
in the related section of the book:
\begin{align}
	L = \sum_{i=1}^n\left(\frac{1}{2}m_i\dot{x_i}^2\right)-V(\{x\})
	\label{eqn:l06e02:lagrangian}
\end{align}

Hence, $(\forall i \in\llbracket 1, n\rrbracket)$:
\begin{align}
	\frac{\partial}{\partial\dot{x_i}}L &=
		\frac{\partial}{\partial\dot{x_i}}
			\sum_{j=1}^n\frac{1}{2}m_j\dot{x_j}^2
	& \frac{\partial}{\partial x_i}L &= -\frac{\partial}{\partial x_i}V(\{x\})
		\nonumber \\
	~ &= \sum_{j=1}^n m_j \dot{x_j}\delta_{ij} & ~ & ~\nonumber \\
	~ &= m_i \dot{x_i} & ~ & ~
	\label{eqn:l06e02:euler-lagrange-split}
\end{align}

Again, we need the \textit{potential energy principle}, stated
as Eq. $(5)$ of the previous chapter \textit{Lecture 5: Energy},
for abstract configuration space $\{x\} = \{x_i\}$, as:
\begin{align}
	F_i(\{x\}) = -\frac{\partial}{\partial x_i}V(\{x\})
	\label{eqn:l06e02:potential}
\end{align}

From which we can conclude, by injecting \eqref{eqn:l06e02:potential}
in the second half of \eqref{eqn:l06e02:euler-lagrange-split}, and
connecting each side with Euler-Lagrange's equations
\eqref{eqn:l06e02:euler-lagrange}, $(\forall i \in\llbracket 1, n\rrbracket)$:
\begin{align*}
	~ & \frac{d}{dt}\left(\frac{\partial}{\partial\dot{x_i}}L\right)
		= \frac{\partial}{\partial x_i}L \\
	\Leftrightarrow & \frac{d}{dt}m_i \dot{x_i} = F_i(\{x\}) \\
	\Leftrightarrow & \boxed{F_i = m_i \ddot{x_i}} \qed
\end{align*}
\end{document}
