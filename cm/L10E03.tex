\documentclass[solutions.tex]{subfiles}

\xtitle

\begin{document}
\maketitle
\begin{exercise}
Using the definition of PB's and the axioms, work out the PB's
in Equations $(19)$. \textit{Hint: In each expression, look for things
in the parentheses that have non-zero Poisson Brackets with the coordinate
$x$, $y$ or $z$. For example, in the first PB, $x$ has a nonzero PB with $p_x$.}
\end{exercise}

Let's start by recalling Equations $(19)$:

\begin{equation*}\begin{aligned}
	\{ x, L_z \} &&=&& \{ x, (xp_y - yp_x) \} \\
	\{ y, L_z \} &&=&& \{ y, (xp_y - yp_x) \} \\
	\{ z, L_z \} &&=&& \{ z, (xp_y - yp_x) \} \\
\end{aligned}\end{equation*}

Then, let's make things a little clearer/regular by renaming our
coordinate variables:

\[ x = q_x;\qquad y = q_y;\qquad z = q_z \]

So, for $k\in\{x, y, z\}$ (that's the set containing $x$, $y$ and $z$,
not a weird Poisson bracket), then all the Poisson brackets to compute are
of the form:

\[ \{ q_k, L_z \} = \{ q_k, (q_xp_y - q_yp_x) \} \]

Let's reduce it from the axioms:

\begin{equation*}\begin{aligned}
	\{ q_k, L_z \} &&=&& \{ q_k, (q_xp_y - q_yp_x) \} && ~ \\
	~ &&=&& - \{ (q_xp_y - q_yp_x), q_k \} && \text{(anti-symmetry)} \\
	~ &&=&& - \left(\{q_xp_y, q_k \} - \{ q_yp_x, q_k \}\right) && \text{(linearity)} \\
	~ &&=&& \{ q_yp_x, q_k \}-\{q_xp_y, q_k \} && ~ \\
	~ &&=&& \Bigl(q_y\{ p_x, q_k \} + p_x \underbrace{\{ q_y, q_k \}}_{=0}\Bigr)
		- \Bigl( q_x\{ p_y, q_k \} + p_y \underbrace{\{ q_x, q_k \}}_{=0}\Bigr)
		&& \text{(product rule)} \\
	~ &&=&& q_y\{ p_x, q_k \} - q_x\{ p_y, q_k \} && \{ q_i, q_j \} = 0
\end{aligned}\end{equation*}

Now suffice for us to evaluate that last expression with each value of $k$,
and simplify the result with $\{ q_i, p_j \} = \delta_i^j$:

\begin{equation*}\begin{aligned}
	k = x &&:&& \{ q_x, L_z \} = q_y\{ p_x, q_x \} - q_x\{ p_y, q_x \} = q_y \\
	k = y &&:&& \{ q_x, L_z \} = q_y\{ p_x, q_y \} - q_x\{ p_y, q_y \} = -q_x \\
	k = z &&:&& \{ q_x, L_z \} = q_y\{ p_x, q_z \} - q_x\{ p_y, q_z \} = 0
\end{aligned}\end{equation*}

Or, with the original notations:

\fbox{\parbox{\textwidth}{
\begin{equation*}\begin{aligned}
	\{ x, L_z \} &&=&& y \\
	\{ y, L_z \} &&=&& -x \\
	\{ z, L_z \} &&=&& 0 \\
\end{aligned}\end{equation*}
}}

\begin{remark} Our solution slightly differs from the one in the book,
as the latter contains a small sign error: the infinitesimal
rotation is said to be:
\begin{equation*}\begin{aligned}
	\delta_x &&=&& -\epsilon y \\
	\delta_y &&=&& \epsilon x
\end{aligned}\end{equation*}

But earlier in the 7th lecture (p135), it was defined to be,
small renaming aside:
\begin{equation*}\begin{aligned}
	\delta_x &&=&& \epsilon y \\
	\delta_y &&=&& -\epsilon x
\end{aligned}\end{equation*}
\end{remark}

\end{document}
