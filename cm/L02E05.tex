\documentclass[solutions.tex]{subfiles}

\xtitle

\begin{document}
\maketitle
\begin{remark} This is a WIP; an important limit still isn't proved
for instance. This should however largely cover what have been expected
by the authors of the book from the students.
\end{remark}
\begin{exercise} Prove each of the formulas in Eq.s $(2)$.
\textit{Hint}: Look up trigonometric identities and limit
properties in a reference book.
\end{exercise}

Let's recall the formulas of Eq.s $(2)$:
\begin{equation*} \begin{aligned}
	\frac{d}{dt}(\sin t) &&=&& \cos t \\
	\frac{d}{dt}(\cos t) &&=&& \sin t \\
	\frac{d}{dt}(e^t) &&=&& e^t \\
	\frac{d}{dt}(\ln t) &&=&& \frac{1}{t}
\end{aligned} \end{equation*}

\begin{remark}
Interestingly, there are multiple ways of \textit{defining}
those functions\footnote{For instance, consider this page
containing $6$ equivalent definitions of the exponentials:
\url{https://en.wikipedia.org/wiki/Characterizations\_of\_the\_exponential\_function}}.
As a result, they are different ways to compute the derivatives,
depending on which definitions we choose. \\

As the definitions given in the book for $\sin$, $\cos$ and
the exponential are rather standard, we'll recall them and use
those. The natural logarithm hasn't been clearly defined though,
so we'll have to do it.
\end{remark}
\begin{remark} While the book suggest to look up trigonometric
identities and common properties in some reference material, we
will actually take the time to prove the intermediate results. \\

One reason to do so for a physicist would be that this is an
interesting mathematical exercise, but also to useful to sharpen
one's rigor and reasoning skills.
\end{remark}

Let's start by recalling that a function $\varphi : E \rightarrow \mathbb{R}$
is said to be differentiable at a point $e\in E$ if the following limit exists:
\[
	\varphi'(e) = \frac{d}{dx}\varphi(e) =
		\boxed{\lim_{\epsilon \rightarrow 0}\frac{\varphi(e+\epsilon)-\varphi(e)}{\epsilon}}
\]

\hr
$\bm{d\sin t/dt}$\,\\
Then, let's remind ourselves of the common definitions of $\cos$ and $\sin$,
in the context of a right triangle:

\begin{equation*}
%\begin{figure}[H]
%	\centering
\vcenter{\hbox{
	\begin{tikzpicture}
		\coordinate (angleat) at (4, 0);
		\draw (0, 0) -- (angleat) node (a) [midway,yshift=-10] {$a$};
		\draw (0, 0) -- (0, 3) node (b) [midway,xshift=-10] {$b$};
		\draw (0, 3) -- (4, 0) node (c) [midway,xshift=10,yshift=2] {$c$};
		% right angle
		\draw [gray] (0,.5)-|(.5,0);
		% \theta angle
		\pic [draw,"$\theta$",angle eccentricity=1.5,gray] {angle = c--angleat--a};
	\end{tikzpicture}}}
%\end{figure}
\quad\quad
\begin{aligned}
	\boxed{\cos \theta \triangleq \frac{c}{a}} \\
	\boxed{\sin \theta \triangleq \frac{b}{c}}
\end{aligned}
\end{equation*}

In particular, we can identify points on the unit-circle
by the angle between the $x$ axis and the radius connecting
the center of the circle to such points. Then, each point will
then be located in the $xy$-plane as $(\cos\theta, \sin\theta)$,
where $\theta$ is the angle previously described, associated
to the point.

\begin{figure}[H]
	\centering
	\begin{tikzpicture}
		\tikzmath{
			\xmin = -5;
			\xmax = 5;
			\ymin = -5;
			\ymax = 5;
			\r = 4;
			\rd = 2*\r; % \r * 2cm, where 2cm is the default x length
			\t = 30;
			\px = \r * cos(\t);
			\py = \r * sin(\t);
		}
		\coordinate (O) at (0, 0);
		\coordinate (p) at (\t:\r);

		\draw[color=gray!30, dashed] (\xmin,\ymin) grid (\xmax,\ymax);
		% axes
		\draw[dashed,gray,->] (\xmin,0) -- (\xmax,0) node (xaxis) [gray,right]{$\vec{x}$};
		\draw[dashed,gray,->] (0,\ymin) -- (0,\ymax) node (yaxis) [gray,right]{$\vec{y}$};

		% 8cm = \r * 2cm (which is the default x length)
		\node[draw,circle, minimum size=\rd cm,inner sep=0] (c) at (0,0){};
		\draw[->] (O) -- (p) node (pvec) [midway,yshift=10,xshift=-2]{$r=1$};

		\pic [draw,"$\theta$",angle eccentricity=1.5,gray] {angle = xaxis--O--pvec};
		\draw[dashed,gray] (p) -- (\px,0) node () [yshift=-10]{$\cos(\theta)$};
		\draw[dashed,gray] (p) -- (0,\py) node () [xshift=-20]{$\sin(\theta)$};
	\end{tikzpicture}
\end{figure}
Note that we have:
\begin{theorem}
\[
	(\forall x \in \mathbb{R}),\quad \boxed{\sin^2 x + \cos^2 x = 1}
\]
\end{theorem}
\begin{proof} This follows immediately from the Pythagorean theorem applied
to the right triangle formed by $r=1$, $\cos\theta$ and $\sin\theta$.
\end{proof}

In order to establish $\sin'$, we will need a few intermediate
results that we're going to prove now. First will be to find a formula
for $\sin(\alpha+\beta)$. Indeed, if you try to apply the definition of the
derivative to $\sin$, you should see a $\sin(x+\epsilon)$: we will need
to have it expressed differently to develop the proof.
\[
	\sin'(x) \triangleq \lim_{\epsilon \rightarrow 0}
		\frac{\sin(x+\epsilon)-\sin x}{\epsilon}
\]

\begin{theorem}[$\sin(\alpha+\beta)$]
\begin{equation}
	(\forall(\alpha, \beta)\in\mathbb{R}^2),\quad
		\boxed{\sin(\alpha+\beta) = \sin\alpha\cos\beta +\cos\alpha\sin\beta}
	\label{L02E05:thm:sinsum}
\end{equation}
\end{theorem}
There's a also a formula for the cosine of a sum of angles, that we will
need later, for the derivative of $\cos$, but that will be rather immediate
to prove along the one regarding the sine of a sum of angles.
\begin{theorem}[$\cos(\alpha+\beta)$]
\begin{equation}
	(\forall(\alpha, \beta)\in\mathbb{R}^2),\quad
		\boxed{\cos(\alpha+\beta) = \cos\alpha\cos\beta +\sin\alpha\sin\beta}
	\label{L02E05:thm:cossum}
\end{equation}
\end{theorem}
\begin{proof} This will be a "visual proof". Besides the aforementioned
definitions of $\sin$ and $\cos$, we will also use the "fact" that sum of
angles in a right triangle is $\pi$, which is actually the \textit{triangle
postulate}, an axiom of Euclidean geometry, equivalent to the
\href{https://en.wikipedia.org/wiki/Parallel\_postulate}{parallel postulate}.
\begin{figure}[H]
	% see https://tex.stackexchange.com/a/383890
	\def\rightangle[size=#1](#2,#3,#4){%
		\draw ($(#3)!#1!(#2)$) --
			($($(#3)!#1!(#2)$)!#1!90:(#2)$) --
			($(#3)!#1!(#4)$);
	}
	\centering
	\begin{tikzpicture}[scale=2]
		\tikzmath{
			\xmin = -0.5;
			\xmax = 4;
			\ymin = -0.5;
			\ymax = 5.5;
			\c = 4;
			\a = 30;
			\b = 40;
			\d = \c/cos(\b);
			\dy = \d*cos(90-\a-\b);
			\dx = \d*sin(90-\a-\b);
			\cx = \c*cos(\a);
			\cy = \c*sin(\a);
		}
		\coordinate (O) at (0, 0);
		\coordinate (pa) at (\a:\c);
		\coordinate (pb) at (\a+\b:\d);
		\coordinate (dy) at (0, \dy);
		\coordinate (dx) at (\dx, 0);
		\coordinate (cx) at (\cx, 0);
		\coordinate (cy) at (0, \cy);
		\coordinate (cxdy) at (\cx, \dy);

		\draw (O)  -- (pa) node [midway,right,xshift=15,yshift=3] {$c$};
		\draw[blue,line width=.5mm] (O) -- (pb) node [black,midway,above,xshift=-5] {$d$};
		\draw (pa) -- (pb) node [midway,above,xshift=3]  {$e$};
		\rightangle[size=10pt](pb,pa,O);

		\draw[dashed,gray] (\xmin,0)      -- (\xmax,0)   node (xaxis) []{};
		\draw[dashed,gray] (0,\ymin)      -- (0,\ymax)   node (yaxis) []{};
		\draw[dashed,gray] (-.5,\dy)        -- (\xmax,\dy) node (dyaxis) []{};
		\draw[dashed,gray] (\dx,\ymin+.5) -- (\dx,\ymax) node (dxaxis) []{};
		\draw[dashed,gray] (\cx,\ymin)    -- (\cx,\ymax) node (cxaxis) []{};
		\draw[dashed,gray] (pa)           -- (\xmax,\cy) node (cyaxis) []{};

		\draw[blue,line width=.5mm] (O) -- (\dx,0);
		\draw[blue,line width=.5mm] (\dx,0) -- (\dx,\dy);

		\draw[gray,<->] (\cx+.5,\cy) -- (\cx+.5,\dy) node (ecosa)
			[midway,right,black]{$e\cos\alpha$};
		\draw[gray,<->] (\cx+.5,0)   -- (\cx+.5,\cy) node (csina)
			[midway,right,black]{$c\sin\alpha$};

		\draw[gray,<->] (\dx,\dy+.5) -- (\cx,\dy+.5) node (esina)
			[midway,above,black]{$e\sin\alpha$};
		\draw[gray,<->] (0,-.5) -- (\cx,-.5) node (ccosa)
			[midway,above,black]{$c\cos\alpha$};

		\draw[gray,<->] (0,\dy+.5) -- (\dx,\dy+.5) node (dcosab)
			[midway,above,black]{$\bm{d\cos(\alpha+\beta)}$};
		\draw[gray,<->] (-.5,0) -- (-.5,\dy) node (dsinab)
			[midway,left,black]{$\bm{d\sin(\alpha+\beta)}$};

		\rightangle[size=10pt](O,dy,dyaxis);
		\rightangle[size=10pt](dxaxis,dx,O);
		\rightangle[size=10pt](cxaxis,cx,O);
		\rightangle[size=10pt](dy,cxdy,cx);

		\pic [draw,"$\alpha$",angle eccentricity=2,gray] {angle = xaxis--O--pa};
		\pic [draw,"$\beta$",angle eccentricity=2,gray] {angle = pa--O--pb};
		\pic [draw,"$\dfrac{\pi}2-\alpha$",angle eccentricity=2.1,gray] {angle = O--pa--cx};
		\pic [draw,"$\alpha$",angle eccentricity=2,gray] {angle = cxdy--pa--pb};
%		\pic [draw,"$\dfrac{\pi}2-\alpha$",angle eccentricity=3,gray] {angle = pa--pb--cxdy};
%		\pic [draw,"$\alpha$",angle eccentricity=2,gray] {angle = dx--pb--pa};
	\end{tikzpicture}
\end{figure}
In the previous picture, considering the right triangle formed by $c$, $d$
and $e$, we have:
\begin{equation*} \begin{aligned}
	c &&=\quad& d\cos\beta; &&\quad& e &&=\quad& d\sin\beta
\end{aligned} \end{equation*}
If we look at the blue/thick triangle right triangle (of hypotenuse
$d$, with an angle of $\beta+\alpha$, and whose other sides are
created by projecting the point formed by $d$ and $e$ down to the
bottom), we find a new relation, to which we can inject our previous
results for $c$ and $e$:
\begin{equation*} \begin{aligned}
	~ && d\sin(\alpha+\beta) &&=\quad& c\sin\alpha+e\cos\alpha \\
	\Leftrightarrow && ~ &&=\quad& (d\cos\beta)\sin\alpha+(d\sin\beta)\cos\alpha \\
	\Leftrightarrow && \sin(\alpha+\beta) &&=\quad& \boxed{\cos\beta\sin\alpha+\sin\beta\cos\alpha}
\end{aligned} \end{equation*}\hfill$\qed$\\
In the same blue/thick triangle, we can also establish a relation
for $\cos(\alpha+\beta)$, using the same definition of $c$ and $e$
as before to conclude:
\begin{equation*} \begin{aligned}
	~ && d\cos(\alpha+\beta) &&=\quad& c\cos\alpha-e\sin\alpha \\
	\Leftrightarrow && ~ &&=\quad& (d\cos\beta)\cos\alpha-(d\sin\beta)\sin\alpha \\
	\Leftrightarrow && \cos(\alpha+\beta) &&=\quad& \boxed{\cos\beta\cos\alpha-\sin\beta\sin\alpha}
\end{aligned} \end{equation*}

\end{proof}

Now if you try to write down $\sin'$ as previously suggested,
and if you decompose $\cos(x+\epsilon)$ with that formula, you see
that this will give birth to two limits:
\begin{equation*} \begin{aligned}
	\sin'(x) &\triangleq&& \lim_{\epsilon \rightarrow 0}
		\frac{\sin(x+\epsilon)-\sin x}{\epsilon} \\
	~ &=&& \lim_{\epsilon \rightarrow 0}
		\frac{\sin x\cos\epsilon+\cos x\sin\epsilon-\sin x}{\epsilon} \\
	~ &=&& \lim_{\epsilon \rightarrow 0}\left(
		\frac{\sin x\cos\epsilon-1}{\epsilon}+\frac{\cos x\sin\epsilon}{\epsilon}
		\right) \\
\end{aligned} \end{equation*}

As we've already explained in
\href{https://github.com/mbivert/ttm/blob/master/cm/L02E04.pdf}{L02E04},
we can recursively split the previous limits, assuming each individual
limit exists:
\[
	\lim_{x \rightarrow a}\left(\varphi(x)+\psi(x)\right) =
		\lim_{x \rightarrow a}\varphi(x)
		+ \lim_{x \rightarrow a}\psi(x))
\]
\[
	\lim_{x \rightarrow a}\left(\varphi(x)\psi(x)\right) =
		\lim_{x \rightarrow a}\varphi(x)
		\times\lim_{x \rightarrow a}\psi(x))
\]
Again, you may want to refer to \textit{Paul's Online Notes}
\footnote{\url{https://tutorial.math.lamar.edu/classes/calci/limitproofs.aspx}}
for a proof. So we need to see if those limits actually exists, and
because we want to compute $\sin'$ in the end, we also want to know
their value.

\begin{theorem}
\[
	\boxed{\lim_{\epsilon\rightarrow 0} \frac{\sin\epsilon}{\epsilon} = 1}
\]
\end{theorem}
\begin{proof}
TODO (the usual  the limit requires the squeeze
theorem\footnote{\url{
https://en.wikipedia.org/wiki/Squeeze\_theorem}} applied to
the area of a circular sector
\footnote{\url{https://en.wikipedia.org/wiki/Circular\_sector}},
itself demanding to know the area of a circle. Unfortunately,
I'm not sure we can be simpler, given our definition of sine
at least).
\end{proof}

As for the other limit:
\begin{theorem}
\[
	\boxed{\lim_{\epsilon\rightarrow 0} \frac{\cos\epsilon-1}{\epsilon} = 0}
\]
\end{theorem}
\begin{proof} We will rely on the previous limit; this will be a "proof
with a trick" (multiplying by $a/a=1$; in the present context, $a$ will
always be non-zero). Note also at the end that we can
apply the product rules for limits given the existence of both sublimits.
\begin{equation*} \begin{aligned}
	\lim_{\epsilon\rightarrow 0} \frac{\cos\epsilon-1}{\epsilon}
		&=&& \lim_{\epsilon\rightarrow 0}\left(
			\frac{\cos\epsilon-1}{\epsilon}
			\times\frac{\cos\epsilon+1)}{\cos\epsilon+1)}
		\right) \\
	~ &=&& \lim_{\epsilon\rightarrow 0}\left(
			\frac{\cos^2\epsilon-1}{\epsilon(\cos\epsilon+1))}
		\right) \\
	~ &=&& -\lim_{\epsilon\rightarrow 0}\left(
			\frac{\sin^2\epsilon}{\epsilon(\cos\epsilon+1))}
		\right) \\
	~ &=&& -\lim_{\epsilon\rightarrow 0}\left(
			\frac{\sin\epsilon}{\epsilon}\times\frac{\sin\epsilon}{\cos\epsilon+1}
		\right) \\
	~ &=&& -\underbrace{\lim_{\epsilon\rightarrow 0}(
			\frac{\sin\epsilon}{\epsilon})}_{\rightarrow 1}
		\times\underbrace{
			\lim_{\epsilon\rightarrow 0}(\frac{\sin\epsilon}{\cos\epsilon+1})
		}_{\rightarrow 0/2=0} \\
	~ &=&& \boxed{0} \\
\end{aligned} \end{equation*}
\end{proof}

We now have everything to conclude: let's recapitulate all the
intermediate steps to compute $\sin'$:
\begin{theorem}[sine derivative]
\[
	(\forall x\in\mathbb{R}),\quad \boxed{\sin'(x) = \cos(x)}
\]
\end{theorem}

\begin{proof}
\begin{equation*} \begin{aligned}
	(\forall x \in\mathbb{R}),\quad\sin'(x) &\triangleq&& \lim_{\epsilon \rightarrow 0}
		\frac{\sin(x+\epsilon)-\sin x}{\epsilon} \\
	~ &=&& \lim_{\epsilon \rightarrow 0}
		\frac{\sin x\cos\epsilon+\cos x\sin\epsilon-\sin x}{\epsilon} \\
	~ &=&& \lim_{\epsilon \rightarrow 0}\left(
		\frac{\sin x(\cos\epsilon-1)}{\epsilon}+\frac{\cos x\sin\epsilon}{\epsilon}
		\right) \\
	~ &=&& \sin x\underbrace{
			\lim_{\epsilon \rightarrow 0}\frac{\cos\epsilon-1}{\epsilon}
		}_{=0}
		+\cos x\underbrace{
			\lim_{\epsilon\rightarrow 0}\frac{\sin\epsilon}{\epsilon}
		}_{=1} \\
	~ &=&& \boxed{\cos x}
\end{aligned} \end{equation*}
\end{proof}
\hr
$\bm{d\cos t/dt}$\,\\
If we apply our previous formulas \eqref{L02E05:thm:sinsum} and
\eqref{L02E05:thm:cossum} regarding respectively the sine and
cosine of a sum of two angles, in the case where one angle is
$\pi/2$, we have:
If one of those angle is $\pi/2$, we have:
\begin{equation*} \begin{aligned}
	(\forall x\in\mathbb{R}),\quad \sin(x+\frac{\pi}2) &&=\quad&
		\sin x\underbrace{\cos\frac{\pi}{2}}_{=0}
		+\cos x\underbrace{\sin \frac{\pi}{2}}_{=1} \\
	~ &&=\quad& \boxed{\cos x} \\
	(\forall x\in\mathbb{R}),\quad \cos(x+\frac{\pi}2) &&=\quad&
		\cos x\underbrace{\cos\frac{\pi}{2}}_{=0}
		-\sin x\underbrace{\sin \frac{\pi}{2}}_{=1} \\
	~ &&=\quad& \boxed{-\sin x}
\end{aligned} \end{equation*}

From there:
\begin{theorem}[cosine derivative]
\[
	(\forall x\in\mathbb{R}),\quad \boxed{\cos'(x) = -\sin(x)}
\]
\end{theorem}
\begin{proof}
\begin{equation*} \begin{aligned}
	(\forall x\in\mathbb{R}),\quad \cos'(x) &&=\quad&
		\sin'(x+\frac\pi2) \\
	~ &&=\quad& (\sin\circ(y \rightarrow y+\frac\pi2))'(x) \\
	~ &&=\quad& \cos(x+\frac\pi2) \\
	~ &&=\quad& \boxed{-\sin(x)}
\end{aligned} \end{equation*}
\end{proof}

\hr
$\bm{d e^t/dt}$\,\\
This one, as mentioned in the book, is "trivial" when we define the
exponential function to be precisely the function which is equal to
its derivative (and such as $e^0$ = 1). \\

And this is usually the way the exponential function will be first
introduced to students. You may want to have a look at other equivalent
characterisation of the function\footnote
{\url{https://en.wikipedia.org/wiki/Characterizations\_of\_the\_exponential\_function}}.
Trying to compute an exponential defined on a development in infinite
series carries a certain aesthetic for instance.

\[ \boxed{\frac{d}{dt}e^t \triangleq e^t} \]

\hr
$\bm{d\ln t/dt}$\,\\
As for the exponential, there can be some variety here depending
on how we \textit{characterize} the $\ln$ function
\footnote{\url{https://en.wikipedia.org/wiki/Natural\_logarithm\#Definitions}}.
Usually, it will be introduced as the \textit{inverse function} of
the exponential:
\begin{definition}[natural logarithm]
The natural logarithm function is defined as the function $\ln$ such that:
\[
	(\forall x\in\mathbb{R}),\quad \boxed{e^{\ln(x)} = x}
\]
\end{definition}
\begin{remark} To rigorously establish this definition, would have
needed to prove that the exponential is invertible.
% TODO?
%	Strictly positive => 1-to-1 => invertible
%	e-x * e^x = e^0 = 1 \forall x so e^x can't be zero \forall x
%		and e^0 = 1 => exp strictly positive
%	suffice to prove that e^x * e^y = e^x+y then
%	https://math.stackexchange.com/a/749179
\end{remark}

\begin{theorem}[natural logarithm derivative]
\[
	(\forall x\in\mathbb{R}),\quad \boxed{\ln'(x) = \frac1x}
\]
\end{theorem}
\begin{proof}
The proof develops from the previous definition of the logarithm
by integrating both side and then applying the chain rule:
\begin{equation*} \begin{aligned}
	~ && (\forall x\in\mathbb{R}),\quad e^{\ln(x)} &&=\quad& x \\
	\Leftrightarrow && \frac{d}{dx}e^{\ln(x)} &&=\quad& \frac{d}{dx}x \\
	\Leftrightarrow && \ln'(x)\underbrace{e^{\ln(x)}}_{=x} &&=\quad& 1 \\
	\Leftrightarrow && \ln'(x) &&=\quad& \boxed{\frac1x}
\end{aligned} \end{equation*}
\end{proof}

\begin{remark} For the sake of completeness, some authors\footnote{
\url{https://www.whitman.edu/mathematics/calculus\_late\_online/section09.02.html}},
will for instance start by defining the logarithm as an integral,
and then define the exponential as the inverse of the logarithm. From which
they can prove that the derivative of the exponential is the exponential.
\end{remark}
\end{document}
