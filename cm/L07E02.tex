\documentclass[solutions.tex]{subfiles}

\xtitle

\begin{document}
\maketitle
\begin{exercise}
Explain this conservation.
\end{exercise}
Let us recall that the referred conserved quantity is:
\[
	b p_1 + a p_2
\]
In the context of the following Lagrangian:
\begin{align}
	L = \frac{1}{2}(\dot{q_1}^2+\dot{q_2}^2) - V(a q_1-b q_2)
	\label{eqn:l07e02:lagrangian}
\end{align}

Because the question is a unclear, we'll make the conservation explicit
mathematically, and we'll try to understand the physical meaning
of such a quantity being conserved.

\hrr

As for the previous exercise, we can start by recalling
Euler-Lagrange's equations, for instance taken from
Equation $(13)$ of the previous chapter ("Lecture $6$:
The Principle of Least Action"):
\[
	\frac{d}{dt}\biggl(\frac{\partial}{\partial \dot{q_i}}L\biggr)
	= \frac{\partial}{\partial q_i}L
\]

Which was in the book followed by the definition of the
conjugate momentum $p_i$:
\[
	p_i = \frac{\partial}{\partial \dot{q_1}}L
\]

For our Lagrangian \eqref{eqn:l07e02:lagrangian}, we have
for the first half of Euler-Lagrange equations:
\begin{align}
	p_1 &\equiv \frac{\partial}{\partial \dot{q_1}}L = \dot{q_1} &
	p_2 &\equiv \frac{\partial}{\partial \dot{q_2}}L = \dot{q_2} \label{eqns:l07e02:p1} \\
	\frac{d}{dt}p_1 &= \dot{p_1} = \ddot{q_1} &
	\frac{d}{dt}p_2 &= \dot{p_2} = \ddot{q_2} \label{eqns:l07e02:p2}
\end{align}

Using the chain
rule\footnote{\url{https://en.wikipedia.org/wiki/Chain\_rule}}
for the other half, with $\varphi(q_i) = a q_1 - b q_2$, we get:
\begin{align}
	\frac{\partial}{\partial q_1}L &=
		-\frac{\partial}{\partial q_1}V(\varphi(q_1)) &
	\frac{\partial}{\partial q_2}L &=
		-\frac{\partial}{\partial q_2}V(\varphi(q_2)) \nonumber \\
	~ &= -\frac{\partial}{\partial q_1}\varphi(q_1)
		\frac{\partial}{\partial q_1}V(\varphi(q_1)) &
	~ &= -\frac{\partial}{\partial q_2}\varphi(q_2)
		\frac{\partial}{\partial q_2}V(\varphi(q_2)) \nonumber \\
	~ &= -(a\frac{\partial}{\partial q_1}V)(a q_1 - b q_2) &
	~ &= +(b\frac{\partial}{\partial q_2}V)(a q_1 - b q_2) \label{eqns:l07e02:p3}
\end{align}

As for the previous exercise, it seems that there a tacit assumption
of a symmetry within the potential $V$ so that we can write
$V' = \dfrac{\partial}{\partial q_i}V$; then, combining $\eqref{eqns:l07e02:p1}$,
$\eqref{eqns:l07e02:p2}$ and $\eqref{eqns:l07e02:p3}$:

\begin{align*}
	\dot{p_1} &= -aV'(a q_1 - b q_2) &
	\dot{p_2} &= +bV'(a q_1 - b q_2)
\end{align*}

As suggested, let's multiply the first equation by $b$, the second by $a$,
and sum the result:
\[
	b\dot{p_1} + a\dot{p_2} = -baV'(a q_1 - b q_2) +abV'(a q_1 - b q_2) = 0
\]
By linearity of the derivation, this is equivalent to say that:
\[
	\frac{d}{dt}(b p_1(t) + a p_2(t)) = 0
\]
Which indeed means that $b p_1(t) + a p_2(t) \in \mathbb{R}$ is a indeed
a constant over time, i.e. that it is conserved (over time).

\hrr

Now, let's see if we can understand what this means physically: essentially,
$a q_1 - b q_2$ means that we're scaling the "position" of the particles
respectively by $a$ and $b$, and make the potential depends on the resulting
distance. \\

The conserved quantity is the "conjugate" of this distance


\end{document}