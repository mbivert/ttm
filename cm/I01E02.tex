\documentclass[solutions.tex]{subfiles}

\xtitle

\begin{document}
\maketitle
\begin{exercise} Work out the rule for vector substraction.
\end{exercise}
This exercise is about getting a (visual) feel for vector
manipulation; it is \textit{not} about vector coordinates manipulation.
We were previously taught how to multiple vectors by a negative
scalar:
\begin{quote}
For example, $-2\vec{r}$ is the vector that is twice as long
as $\vec{r}$, but points in the opposite direction.
\end{quote}
And how to add vectors:
\begin{quote}
To add $\vec{A}$ and $\vec{B}$, place them as shown in Figure $13$
to form a quadrilateral (this way the directions of the vectors are
preserved). The sum of the vectors is the length and angle of the
diagonal
\end{quote}
So, by observing that (we'll use a \textbf{bold} font to denote
vectors instead of arrows, e.g. $\bm{v}$ is a vector):
\[
	\bm{u}-\bm{v} = \bm{u} + (-1\bm{v})
\]
We conclude that we first need to reverse the direction of the
vector to be substracted, and add this to the other vector.
Visually:
\begin{figure}[H]
	\centering
	\begin{tikzpicture}
		\tikzmath{
			\xmin = -2;
			\xmax = 3;
			\ymin = -3;
			\ymax = 3;
		}
		\coordinate (orig) at (0,0);
		\coordinate (u) at (3, 2);
		\coordinate (v) at (1, 3);
		\coordinate (mv) at (-1, -3);
		\coordinate (umv) at (2, -1);

		\draw[->] (\xmin-1, 0) -- (\xmax+1, 0) node[right] {$x$};
		\draw[->] (0, \ymin-1) -- (0, \ymax+1) node[above] {$y$};
		\draw[help lines, color=gray!30, dashed]
			(\xmin-1,\ymin-1) grid (\xmax+1,\ymax+1);
		\draw[->, black] (orig) -- (u) node() [below,midway,xshift=4]{$\bm{u}$};

		\draw[->] (orig) -- (v) node() [below,midway,xshift=4]{$\bm{v}$};
		\draw[->, gray] (orig) -- (mv) node() [below,midway,xshift=8]{$\bm{-v}$};
		\draw[->, gray, dashed] (u) -- (umv) node() [below,midway,xshift=8]{$-\bm{v}$};
		\draw[->, line width=2] (orig) -- (umv) node() [below,midway,yshift=-5]{$\bm{u}-\bm{v}$};
	\end{tikzpicture}
\end{figure}
\end{document}
