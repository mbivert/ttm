\documentclass[solutions.tex]{subfiles}

\xtitle

\begin{document}

\maketitle
\begin{exercise} Prove the sum rule (fairly easy), the
product rule (easy if you know the trick), and the chain rule
(fairly easy).
\end{exercise}
You may want to refer to a \textit{real} mathematical textbook for a
finer, more rigorous treatment of those exercises. We'll give reasonably
solid proofs, that should be sufficient in the context of an
introductory physics textbook. A interesting middle-ground, that we won't
explore here, would be to study the proofs that one can obtain from an
alternative (but equivalent) formulation of the derivative known as the
Carathéodory's derivative, which allows for simple proofs of such results. \\

Let's start by recalling how differentiation is defined.

\begin{definition} A function $\varphi : E \rightarrow \mathbb{R}$ is
said to be differentiable at a point $e\in E$ if the following limit exists:
\[
	\varphi'(e) = \frac{d}{dx}\varphi(e) =
		\boxed{\lim_{\epsilon \rightarrow 0}\frac{\varphi(e+\epsilon)-\varphi(e)}{\epsilon}}
\]
If this limit exists for all points $x$ of $E$ (we note, $(\forall x \in E)$), then
$\varphi$ is said to be differentiable on $E$, or simply differentiable. The function
which associate to each points of $E$ this limit is called the \textit{derivative}
of $\varphi$, and is called $\varphi$.
\end{definition}

\begin{theorem}[sum rule] Let $\varphi, \psi : E \rightarrow \mathbb{R}$,
both differentiable on $E$. Then,
\[
	\boxed{(\varphi+\psi)' = \varphi' + \psi'}
\]
\end{theorem}
\begin{proof} We have, by definition of the differentiation, and after
re-ordering the terms
\begin{equation*} \begin{aligned}
	(\forall x \in E),\quad(\varphi+\psi)'(x) &&=\quad&
		\lim_{\epsilon \rightarrow 0}\frac{(\varphi+\psi)(x+\epsilon)-(\varphi+\psi)(x)}{\epsilon} \\
	~ &&=\quad &
		\lim_{\epsilon \rightarrow 0}\frac{\varphi(x+\epsilon)+\psi(x+\epsilon)-\varphi(x)-\psi(x)}{\epsilon} \\
	~ &&=\quad &
		\lim_{\epsilon \rightarrow 0}\left(\frac{\varphi(x+\epsilon)-\varphi(x)}{\epsilon}+
			\frac{\psi(x+\epsilon)-\psi(x)}{\epsilon}\right) \\
	~ &&=\quad &
		\lim_{\epsilon \rightarrow 0}\frac{\varphi(x+\epsilon)-\varphi(x)}{\epsilon}
		+\lim_{\epsilon \rightarrow 0}\frac{\psi(x+\epsilon)-\psi(x)}{\epsilon} \\
	~ &&=\quad & \boxed{\varphi'(x)+\psi'(x)}
\end{aligned} \end{equation*}
\end{proof}
\begin{remark} As a rigorous proof is a bit tedious\footnote{if you want one,
have a look at \textit{Paul's Online Notes}:
\url{https://tutorial.math.lamar.edu/classes/calci/limitproofs.aspx}},
we \underline{assumed} for the last step that a limit of a sum is the
sum of the limits, \underline{when all the involved limits exist} (which
is the case here, because those limits are equivalent to saying our functions
are differentiable, which they are, per hypothesis)
\[
	\lim_{x \rightarrow a}\left(\varphi(x)+\psi(x)\right) =
		\lim_{x \rightarrow a}\varphi(x)
		+ \lim_{x \rightarrow a}\psi(x))
\]
\end{remark}

\begin{theorem}[product rule] Let $\varphi, \psi : E \rightarrow \mathbb{R}$,
both differentiable on $E$. Then,
\[
	\boxed{(\varphi\psi)' = \varphi'\psi + \varphi\psi'}
\]
\end{theorem}
\begin{proof} This is a simple and often used theorem, but unfortunately,
the proof of it is a bit "magic": if we start by applying the definition
of the differentiation to $(\varphi\psi)'$, we have to introduce a well-crafted
term (in the form $-a+a=0$) so as to factorize things to meet our goal. We would
furthermore be implicitly assuming that $(\varphi\psi)'$ exists, but we have
no guarantee of it. \\

We can solve those issues by starting from the definition of the
differentiation of $\varphi'\psi + \varphi\psi'$, but assuming we already
know the result we're trying to prove, is conceptually clumsy. Perhaps
calling this a "verification" rather than a proof would be more correct
then. \\

Using the former derivation, we have:
\begin{equation*} \begin{aligned}
	(\forall x \in E),\quad(\varphi\psi)'(x) &&=\quad&
		\lim_{\epsilon \rightarrow 0}\frac{(\varphi\psi)(x+\epsilon)-(\varphi\psi)(x)}{\epsilon} \\
	~ &&=\quad &
		\lim_{\epsilon \rightarrow 0}\frac{\varphi(x+\epsilon)\psi(x+\epsilon)-\varphi(x)\psi(x)}{\epsilon} \\
	~ &&=\quad &
		\lim_{\epsilon \rightarrow 0}\frac{
			\varphi(x+\epsilon)\psi(x+\epsilon)-\varphi(x)\psi(x)
			\overbrace{-\varphi(x+\epsilon)\psi(x)
			+\varphi(x+\epsilon)\psi(x)}^{=0}
		}{\epsilon} \\
	~ &&=\quad &
		\lim_{\epsilon \rightarrow 0}\frac{
			\varphi(x+\epsilon)(\psi(x+\epsilon)-\psi(x))
			+
			\psi(x)(\varphi(x+\epsilon)-\varphi(x))
		}{\epsilon} \\
	~ &&=\quad &
		\lim_{\epsilon \rightarrow 0}\varphi(x+\epsilon)\frac{
			(\psi(x+\epsilon)-\psi(x))
		}{\epsilon}
		+
		\psi(x)\lim_{\epsilon \rightarrow 0}\frac{
			\varphi(x+\epsilon)-\varphi(x)
		}{\epsilon} \\
	~ &&=\quad &
		\left(\lim_{\epsilon \rightarrow 0}\varphi(x+\epsilon)\right)
		\lim_{\epsilon \rightarrow 0}\frac{
			\psi(x+\epsilon)-\psi(x)
		}{\epsilon}
		+
		\psi(x)\lim_{\epsilon \rightarrow 0}\frac{
			\varphi(x+\epsilon)-\varphi(x)
		}{\epsilon} \\
	~ &&=\quad & \boxed{\varphi(x)\psi'(x) + \psi(x)\varphi'(x)}
\end{aligned} \end{equation*}
\end{proof}
\begin{remark} We \underline{assumed} another result on limits
\footnote{If you want a rigorous proof of it, you can refer to the
same resource as before:
\url{https://tutorial.math.lamar.edu/classes/calci/limitproofs.aspx}},
again assuming individual limits exists (and they do in our case, as
our functions are differentiable, and (thus) continuous):
\[
	\lim_{x \rightarrow a}\left(\varphi(x)\psi(x)\right) =
		\lim_{x \rightarrow a}\varphi(x)
		\times\lim_{x \rightarrow a}\psi(x))
\]

\end{remark}
\begin{theorem}[chain rule] Let $\varphi, \psi : E \rightarrow \mathbb{R}$,
both differentiable on $E$. Then,
\[
	\boxed{(\varphi\circ\psi)' = \psi'\times(\varphi'\circ\psi)}
\]
\end{theorem}
\begin{proof} Again, the proof is a bit "magical" in that we're going
to multiply by a well-crafted term, of the form $a/a=1$:
\begin{equation*} \begin{aligned}
	(\forall x \in E),\, (\varphi\circ\psi)(x)) &&=\quad &
		\lim_{\epsilon \rightarrow 0}\frac{
			(\varphi\circ\psi)(x+\epsilon)-(\varphi\circ\psi)(x))
		}{\epsilon} \\
	~ &&=\quad &
		\lim_{\epsilon \rightarrow 0}\left(\frac{
			\varphi(\psi(x+\epsilon))-\varphi(\psi(x)))
		}{\epsilon}\times\overbrace{
			\frac{\psi(x+\epsilon)-\psi(x)}{\psi(x+\epsilon)-\psi(x)}
		}^{=1}\right) \\
	~ &&=\quad &
		\lim_{\epsilon \rightarrow 0}\left(\frac{
			\varphi(\psi(x+\epsilon))-\varphi(\psi(x)))
		}{\psi(x+\epsilon)-\psi(x)}\times
			\frac{\psi(x+\epsilon)-\psi(x)}{\epsilon}
		\right) \\
	~ &&=\quad &
		\lim_{\epsilon \rightarrow 0}\frac{
			\varphi(\psi(x+\epsilon))-\varphi(\psi(x)))
		}{\psi(x+\epsilon)-\psi(x)}
		\times
		\underbrace{\lim_{\epsilon \rightarrow 0}
			\frac{\psi(x+\epsilon)-\psi(x)}{\epsilon}}_{\psi'(x)}\\
\end{aligned} \end{equation*}
Again for that last step, we've used the aforementioned rule on products
of existing limits. To conclude, we need to compute the first limit: let's
define $h=\psi(x+\epsilon)-\psi(x) \Leftrightarrow \psi(x+\epsilon) = \psi(x)+h$.
Note that $\epsilon \rightarrow 0 \Rightarrow h \rightarrow 0$. So the first
limit can be rewritten:
\[
	\lim_{h \rightarrow 0}\frac{
		\varphi(\psi(x)+h)-\varphi(\psi(x)))
	}{h} \triangleq \varphi'(\psi(x))
\]
\end{proof}
Now a problem with this previous proof is that it is invalid if $\psi$ is
a constant function for example, because $\psi(x+\epsilon)-\psi(x)$ is zero,
and we're dividing by zero when performing our magical "multiplication" by 1.
If you're interested, you can find an alternative proof using the other,
equivalent form of the derivative here: \url{https://www.youtube.com/watch?v=COLwYhEAt7Q}.
\end{document}
