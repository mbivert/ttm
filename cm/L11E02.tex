\documentclass[solutions.tex]{subfiles}

\xtitle

\begin{document}
\maketitle
\begin{exercise}
Prove Eq. $(4)$.
\end{exercise}
Where Eq. $(4)$ is the following, for $V$ a scalar field:
\[
	\vec{\nabla}\times\Bigl(\vec\nabla V(x)\Bigr) = 0
\]

If think we can agree that $V(x)$ is actually a $V(x, y, z)$. \\

And $\vec\nabla$ is the differentiation vector operator:
\[
	\vec\nabla = \begin{pmatrix}
		\dfrac{\partial}{\partial x} \\[1em]
		\dfrac{\partial}{\partial y} \\[1em]
		\dfrac{\partial}{\partial z}
	\end{pmatrix}
\]

By this definition,
\[
	\vec\nabla V(x, y, z) = \begin{pmatrix}
		\dfrac{\partial V}{\partial x} \\[1em]
		\dfrac{\partial V}{\partial y} \\[1em]
		\dfrac{\partial V}{\partial z}
	\end{pmatrix}
\]

We also have previously established that for a field
$F = (F_x, F_y, F_z)$:
\[
	\vec\nabla\times F = \begin{pmatrix}
		\dfrac{\partial F_z}{\partial y} - \dfrac{\partial F_y}{\partial z} \\[1em]
		\dfrac{\partial F_x}{\partial z} - \dfrac{\partial F_z}{\partial x} \\[1em]
		\dfrac{\partial F_y}{\partial x} - \dfrac{\partial F_x}{\partial y}
	\end{pmatrix}
\]

And so,
\[
	\vec\nabla\times (\vec\nabla V(x, y, z)) = \begin{pmatrix}
		\dfrac{\partial }{\partial y}\dfrac{\partial V}{\partial z}
			- \dfrac{\partial}{\partial z}\dfrac{\partial V}{\partial y} \\[1em]
		\dfrac{\partial}{\partial z}\dfrac{\partial V}{\partial x}
			- \dfrac{\partial}{\partial x}\dfrac{\partial V}{\partial z} \\[1em]
		\dfrac{\partial}{\partial x}\dfrac{\partial V}{\partial y}
			- \dfrac{\partial}{\partial y}\dfrac{\partial V}{\partial x}
	\end{pmatrix} = \vec{0}
\]

Where we can conclude because of Schwarz/Clairaut's theorem
\footnote{\url{https://en.wikipedia.org/wiki/Symmetry\_of\_second\_derivatives}}.
This means we consider $V$ to have continuous second partial derivatives on
its domain (or, at least in a neighborhood of a point $x$ of its domain), which
is often a reasonable assumption in Physics.

\end{document}
