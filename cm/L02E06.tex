\documentclass[solutions.tex]{subfiles}

\xtitle

\begin{document}
\maketitle
\begin{exercise} How long does it take for the oscillating
particle to go through one full cycle of motion?
\end{exercise}
We're in the case of a particle oscillating in one dimension.
Its motion, known as the \textit{simple harmonic motion},
is described by:
\[
	x(t) = \sin(\omega t)
\]
Essentially, $x(t)$ will correspond to the vertical component of
a point moving on the unit circle, located by an angle $\omega t$.
\begin{figure}[H]
	\centering
	\begin{tikzpicture}
		\tikzmath{
			\xmin = -5;
			\xmax = 5;
			\ymin = -5;
			\ymax = 5;
			\r = 4;
			\rd = 2*\r; % \r * 2cm, where 2cm is the default x length
			\t = 30;
			\px = \r * cos(\t);
			\py = \r * sin(\t);
		}
		\coordinate (O) at (0, 0);
		\coordinate (p) at (\t:\r);

		\draw[color=gray!30, dashed] (\xmin,\ymin) grid (\xmax,\ymax);
		% axises
		\draw[dashed,gray,->] (\xmin,0) -- (\xmax,0) node (xaxis) []{};
		\draw[dashed,gray,->] (0,\ymin) -- (0,\ymax) node (yaxis) [gray,right]{$\vec{x}$};

		% \rd = \r * 2cm (where 2cm is the default x length)
		\node[draw,circle, minimum size=\rd cm,inner sep=0,gray!40] (c) at (0,0){};
		\draw[-,gray!40] (O) -- (p) node (pvec) [midway,yshift=10,xshift=-2,gray]{$r=1$};

		\pic [draw,"$\omega t$",angle eccentricity=1.5,gray] {angle = xaxis--O--pvec};
		\draw[dashed,gray] (p) -- (0,\py) node () [xshift=-40,yshift=5,black]{$x(t) = \sin(\omega t)$};
		\filldraw[color=black!80, fill=black!5, thick] (0,\py) circle (0.2);
		\filldraw[fill=black!80] (p) circle (0.05);
	\end{tikzpicture}
\end{figure}

To fix things, consider the case of a particle starting at an
extreme position, say $x=1$ (at the top of the north hemisphere of
the unit circle). It will need to go down to $x=-1$, and then back
up to $x=1$. In the mean time, the corresponding point on the unit
circle would have walked a full circle, or $2\pi$ radians. \\

So we're looking for the time $T$ that it will take for us
to move by an angle $2\pi$, knowing that we move at a speed
of $\omega$ radians per unit of time (i.e. $\omega_{t=0}=0$,
$\omega_{t=1}=\omega$, $\omega_{t=2}=2\omega$, ...):
\[
	\omega T = 2\pi \Leftrightarrow \boxed{T = \frac{2\pi}\omega}
\]

\begin{remark}$T$ is commonly called the \textit{period} of motion.
\end{remark}

\end{document}
