\documentclass[solutions.tex]{subfiles}

\xtitle

\begin{document}
\maketitle
\begin{exercise} Can you explain why the dot product of
two vectors that are orthogonal is $0$?
\end{exercise}
The dot product between two vectors is defined in two ways
in the book; one of them involves the magnitudes of the vectors
and the angle between them:
\begin{equation*} \begin{aligned}
	\bm{u}\cdot\bm{v} &&=\quad& \norm{\bm{u}}\norm{\bm{v}}\cos\theta_{uv}
\end{aligned} \end{equation*}
By definition, two vectors are orthogonal if the angle between
them is $\pi/2$, or $90^\circ$. But, $\cos(\pi/2) = 0$, so it follows
from the previous dot product formula that, for two orthogonal
vectors, their dot product must be $0$.
\end{document}
