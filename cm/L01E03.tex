\documentclass[solutions.tex]{subfiles}

\xtitle

\begin{document}
\maketitle
\begin{exercise}
Determine which of the dynamical laws shown in Eq.s $(2)$ through
$(5)$ are allowable.
\end{exercise}
Let's recall that laws are said to be allowable if they are both
deterministic (i.e. the future behavior of the system is completely
determined by the initial state) and reversible (i.e. the law is
still deterministic even by reversing the direction of all the arrows). \\

Now, the dynamical laws referred to be the exercise are:
\begin{equation*} \begin{aligned}
	N(n+1) &&=&& N(n)-1 &\qquad\qquad \text{(2)} \\
	N(n+1) &&=&& N(n)+2 &\qquad\qquad \text{(3)} \\
	N(n+1) &&=&& N(n)^2 &\qquad\qquad \text{(4)} \\
	N(n+1) &&=&& -1^{N(n)}N(n) &\qquad\qquad \text{(5)} \\
\end{aligned} \end{equation*}

\hr

\begin{figure}[H]
	\centering
	\begin{tikzpicture}
		\node (qminf) [state]                   {...};
		\node (qm2)   [state, right = of qminf] {$-2$};
		\node (qm1)   [state, right = of qm2]   {$-1$};
		\node (q0)    [state, right = of qm1]   {$0$};
		\node (q1)    [state, right = of q0]    {$1$};
		\node (q2)    [state, right = of q1]    {$2$};
		\node (qinf)  [state, right = of q2]    {...};

		\path [-stealth, thick, <-]
			(qminf) edge (qm2)
			(qm2)   edge (qm1)
			(qm1)   edge (q0)
			(q0)    edge (q1)
			(q1)    edge (q2)
			(q2)    edge (qinf);
	\end{tikzpicture}
\end{figure}

$(2)$ is simply $(1)$ (described by $N(n+1)= N(n)+1$) with reversed
arrows; and $(1)$ has already be established as being allowed. Hence,
\fbox{(2) is allowed}.

\hr

\begin{figure}[H]
	\centering
	\begin{tikzpicture}
		\node (qminf) [state]                   {...};
		\node (qm2)   [state, right = of qminf] {$-2$};
		\node (q0)    [state, right = of qm2]   {$0$};
		\node (q2)    [state, right = of q0]    {$2$};
		\node (qinf)  [state, right = of q2]    {...};

		\node (qminf2) [state, below = of qminf]  {...};
		\node (qm3)    [state, right = of qminf2] {$-3$};
		\node (qm1)    [state, right = of qm3]    {$-1$};
		\node (q1)     [state, right = of qm1]    {$1$};
		\node (q3)     [state, right = of q1]     {$3$};
		\node (qinf2)  [state, right = of q3]     {...};

		\path [->]
			(qminf) edge (qm2)
			(qm2)   edge (q0)
			(q0)    edge (q2)
			(q2)    edge (qinf);

		\path [->]
			(qminf2) edge (qm3)
			(qm3)    edge (qm1)
			(qm1)    edge (q1)
			(q1)     edge (q3)
			(q3)     edge (qinf2);
	\end{tikzpicture}
\end{figure}

$(3)$ is very similar to $(1)$ or $(2)$, only, it has two infinite
cycles instead of one. Both are deterministic, and reversing
the arrows lead to deterministic cycles. Hence, \fbox{$(3)$ is allowed}.

\hr

\begin{figure}[H]
	\centering
	\begin{tikzpicture}
		\node (q4)    [state]                      {$4$};
		\node (qm2)   [state, above left = of q4]  {$-2$};
		\node (q2)    [state, above right = of q4] {$2$};
		\node (qinf)  [state, below = of q4]       {$...$};

		\path [-stealth, thick, ->]
			(qm2) edge (q4)
			(q2)  edge (q4)
			(q4)  edge (qinf);
	\end{tikzpicture}
\end{figure}

We didn't plot $(4)$ exhaustively, just enough to illustrate
its irreversibility: if we start at either $-2$ or $2$,
we end up on $4$. By reversing the arrows, we see that we can't decide
where we came from. Furthermore, states such as $-2$ cannot be
reached, because there's no number $n$ such as $-2 = n^2$). Actually,
even $2$ cannot be reached, because there's no \textit{integer} $n$ such
as $2 = n^2$. \\

Hence, \fbox{$(4)$ is \textbf{not} allowed}.

\hr

\begin{figure}[H]
	\centering
	\begin{tikzpicture}
		\node (qnp)    [state]    {$n \equiv 0\ [2]$};

		\node (qmp)    [state,below right = of qnp, align=center] {$m > 0$ \\ $m \equiv 1\ [2]$};
		\node (qmn)    [state,below left = of qnp, align=center]  {$m < 0$ \\ $m \equiv 1\ [2]$};

		\path [-stealth, thick, ->]
			(qnp) edge [loop above] (qnp);

		\path [-stealth, thick, ->]
			(qmp) edge [bend left] (qmn)
			(qmn) edge [bend left] (qmp);
	\end{tikzpicture}
\end{figure}

By sketching a few cycles for $(5)$, we see a pattern emerging, summed
up in the diagram above:
\begin{itemize}
	\item if we start on an even number, positive or negative, we'll loop
	on this number indefinitely;
	\item if we start with an odd number, positive or negative,
	we'll loop between the positive and negative version of
	that number;
\end{itemize}

Perhaps a mathematical subtlety would be in considering $-1$ raised to
a negative power. But,

\[ (\forall n \in \mathbb{N}^*),\quad -1^{-n} = (-1/1)^{-n} \equiv (1/-1)^n = (-1)^n \]

Hence, \fbox{$(5)$ is allowed}.

\end{document}
