\documentclass[solutions.tex]{subfiles}

\xtitle

\begin{document}
\maketitle
\begin{exercise}
Prove Eq. $(14)$
\end{exercise}

Eq. $(14)$ of the book refers to:
\[
	\{F(q,p),p_i\} = \frac{\partial F(q, p)}{\partial q_i}
\]

Where the brackets $\{ ., . \}$ are the Poisson Brackets: for
$A$ and $B$ each two functions of $2N$ variables $\{p_i\}_{1\leq i \leq N}$
and $\{q_i\}_{1\leq i \leq N}$, ($N\in\mathbb{N}$):
\[
	\{A, B\} = \sum_{i=1}^N
		\frac{\partial A}{\partial q_i}
			\frac{\partial B}{\partial p_i}
		-\frac{\partial A}{\partial p_i}
			\frac{\partial B}{\partial q_i}
\]

And $F(p, q)$ a function of $q$ and $p$. There's a bit of ambiguity
regarding what $p$ and $q$ are, which actually doesn't affect the derivation,
but let's make things clear anyway. In the
previous example, we proved that $\{q^n, p\}=nq^{n-1}$: in
this case, $N=1$ and we had a single $q$ and a single $p$. \\

But now we're asked to prove a result involving $F(p, q)$
partially derived according to $q_i$, which implies, for the result
not to be trivial, that $F$ is a function of $q_i$, and thus
that $p$ and $q$ are actually tuples of $N$ $p_i$ and $N$ $q_i$. \\

So, let's expand the Poisson brackets to be evaluated, using the definition
of the Poisson brackets:
\[
	\{F(q,p),p_i\} = \sum_{k=1}^N
		\frac{\partial}{\partial q_k}F(q_1, \cdots, q_N, p_1, \cdots, p_N)
		\frac{\partial p_i}{\partial p_k}
		- \frac{\partial}{\partial p_k}F(q_1, \cdots, q_N, p_1, \cdots, p_N)
		\frac{\partial p_i}{\partial q_k}
\]

Now because $p_i$ will never depends on $q_k$, as those are two distinct variables,
$\dfrac{\partial p_i}{\partial q_k} = 0$, and the previous expression shrinks
to:
\[
	\{F(q,p),p_i\} = \sum_{k=1}^N
		\frac{\partial}{\partial q_k}F(q_1, \cdots, q_N, p_1, \cdots, p_N)
		\frac{\partial p_i}{\partial p_k}
\]

For similar reasons, $\dfrac{\partial p_i}{\partial p_k} = \delta_i^k$,
and the previous expression shrinks again to:
\[
	\{F(q,p),p_i\} =
		\frac{\partial}{\partial q_i}F(q_1, \cdots, q_N, p_1, \cdots, p_N)
			= \boxed{\frac{\partial F(p, q)}{\partial q_i}}
\]

\begin{remark} Eq. $(15)$ of the book is to be proven as we did
for Eq. $(14)$.
\end{remark}

\begin{remark} Earlier in this section, the authors informally invited
us to verify the properties of the Poisson brackets (anti-symmetry, linearity,
product rule). I won't be doing it, because I think at this stage of the
book, this should be elementary: you just have to replace the brackets
by their definition, and re-arrange the terms, often using linearity of the
differentiation/partial differentiation, and then switch back to expressions
involving (the expected) Poisson brackets again. \\
\end{remark}

\end{document}
