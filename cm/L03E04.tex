\documentclass[solutions.tex]{subfiles}

\xtitle

% XXX/TODO: \bm doesn't work with bare mathjax, so we
% tweak things around here. \bm from the bm package seems
% to be the recommended solution for bold math fonts.
% (https://tex.stackexchange.com/a/596)
\renewcommand{\bm}[1]{\pmb{#1}}

% XXX/TODO: similar issue for \qed
%\renewcommand{\qed}{\,\boxed{\,}}
\renewcommand{\qed}{\,\Box}

% XXX/TODO: another tweak: redefining alignat* gives an
% OK pdf, but alignat*'s argument would appear
% in the HTML; Renaming alignat* to alignat2 with
% (Edit ,x/alignat\*/c/alignat2/) essentially
% isn't processed by pandoc, so just do it by hand:
% Edit ,x/\\begin{alignat\*}{.}/c/\begin{equation*} \begin{aligned}/
% Edit ,x/\\end{alignat\*}/c/\end{aligned} \end{equation*}/
%
% Using align* gives the same result as the current solution
% in the HTML, but there's some unpleasant extra-spacing in the
% pdf.
\iffalse
%\renewenvironment{alignat*}[1]{%
\newenvironment{alignat2}[1]{%
    \begin{equation*}%
    \begin{aligned}%
}{%
	\end{aligned}%
    \end{equation*}%
}
\fi

% XXX/TODO: another issue is that we can't use \Aboxed{} (which
% is like \boxed{}, except that you can box things that are spread
% across multiple columns in an aligned environment)

% XXX/TODO: references (\ref{...}) to labels (\label{...}) is an
% unimplemented feature:
%	https://github.com/jgm/pandoc/issues/1015

\begin{document}
\maketitle
\begin{exercise}
Show by differentiation that the general solution to Eq. $(6)$
is given in terms of two constants $A$ and $B$ by
\[
	x(t) = A\cos\omega t + B\sin\omega t
\]
Determine the initial position and velocity at time $t=0$ in terms
of $A$ and $B$.
\end{exercise}
We're in the case of a $1$-dimensional
\href{https://en.wikipedia.org/wiki/Harmonic\_oscillator}
{harmonic oscillator}, depicted below as a mass $m$ set on a frictionless
ground attached to a fixed element to the left by a horizontally positioned
spring of constant $k$.
\begin{figure}[H]
	\centering
	\begin{tikzpicture}
		\coordinate (orig) at (0,0);
		\coordinate (mass) at (5,0);

		% mass
		\filldraw[color=black!60, fill=black!5, thick] (mass) circle (0.5) node [above] {$m$};

		% x-axis
		\draw[dashed,gray,->] (orig) -- (7,0) node[gray,right]{$\vec{x}$};

		% frictionless ground
		\draw[very thin,lightgray] (0,-0.5) -- (7,-0.5);

		% spring
		\draw[decoration={aspect=0.3,segment length=3mm,amplitude=3mm,coil},decorate] (orig) -- (mass);

		% ground
		\draw[very thick] (0,1) -- (0,-1);
		\fill[pattern=north east lines] (-1,-1) rectangle (0,1);
	\end{tikzpicture}
\end{figure}
\begin{remark}
Almost equivalently, we could have worked in a vertical setting, which, on
one hand, would have avoided us the need for a frictionless ground, but
on the other hand, would make the equation slightly more complicated. \\

There are interesting, complementary aspects to both cases. Because
the differences are minute, we can explore both of them here.
\end{remark}

Harmonic oscillators play a major role in physics, as explained by Feynman
in \href{https://www.feynmanlectures.caltech.edu/I\_21.html}{Chapter $21$
of the first volume of his \textit{Lectures on Physics}}:

\begin{quote}
The harmonic oscillator, which we are about to study, has close analogs in many
other fields; although we start with a mechanical example of a weight on a spring,
or a pendulum with a small swing, or certain other mechanical devices, we are
really studying a certain differential equation. This equation appears again
and again in physics and in other sciences, and in fact it is a part of so
many phenomena that its close study is well worth our while.

Some of the phenomena involving this equation are the oscillations of a
mass on a spring; the oscillations of charge flowing back and forth in
an electrical circuit; the vibrations of a tuning fork which is generating
sound waves; the analogous vibrations of the electrons in an atom, which
generate light waves; the equations for the operation of a servosystem,
such as a thermostat trying to adjust a temperature; complicated
interactions in chemical reactions; the growth of a colony of bacteria
in interaction with the food supply and the poisons the bacteria produce;
foxes eating rabbits eating grass, and so on; all these phenomena follow
equations which are very similar to one another, and this is the reason
why we study the mechanical oscillator in such detail.
\end{quote}

Thus, we'll take the time to analyze it more precisely than what
is expected here: this is the occasion to present some aspects not
covered, or not covered as explicitly in the book. \\

Despite everything happening in one-dimension, we'll use a
vector position $\bm{x}$, which is a function of time. Hence, the vectors
for speed and acceleration will respectively be $\bm{v}=\dot{\bm{x}} =
\dfrac{d}{dt}\bm{x}(t)$ and $\bm{a}=\ddot{\bm{x}} = \dfrac{d^2}{dt^2}\bm{x}(t)$.
Note that we're representing vectors with bold font instead of
arrows; bold font is rather common in physics. This should be the only
additional thing needed to follow through for someone having read the
book up to this stage.

\hr
\textbf{Contact force, normal force, friction force} \\
The \href{https://en.wikipedia.org/wiki/Contact\_force}{contact force}
is the force resulting from the contact of two objects. It is generally decomposed
into vertical and horizontal components. The horizontal component perhaps
is the most intuitive: it corresponds to
\href{https://en.wikipedia.org/wiki/Friction}{friction forces}. For
instance, after having given a gentle push to an object an on ordinary table,
it will only move up to a certain point: it is progressively stopped by
the friction between the object and the table. You may think that this friction
is caused at least in part because at a microscopic scale, both the object
and the surface are far, far from being perfectly plane, and even more
so at an atomic scale. \\

But this is false. As a counter-example, consider
\href{https://en.wikipedia.org/wiki/Gauge\_block}{gauge blocks},
manufactured to be extremely flat: their surfaces, far from being
frictionless to each other, cling! \\

Because a rigorous understanding of friction forces requires advanced
mechanics knowledge, we can satisfy ourselves with Feynman's vulgarization,
from \href{https://www.feynmanlectures.caltech.edu/I\_12.html}{Chapter
$12$ of the first volume of his \textit{Lectures on Physics}}:

\begin{quote}
There is another kind of friction, called dry friction or sliding friction,
which occurs when one solid body slides on another. In this case a force is
needed to maintain motion. This is called a frictional force, and its origin,
also, is a very complicated matter. Both surfaces of contact are irregular, on
an atomic level. There are many points of contact where the atoms seem to cling
together, and then, as the sliding body is pulled along, the atoms snap apart
and vibration ensues; something like that has to happen. Formerly the mechanism
of this friction was thought to be very simple, that the surfaces were merely
full of irregularities and the friction originated in lifting the slider over
the bumps; but this cannot be, for there is no loss of energy in that process,
whereas power is in fact consumed. The mechanism of power loss is that as the
slider snaps over the bumps, the bumps deform and then generate waves and
atomic motions and, after a while, heat, in the two bodies. \\

[...] \\

It was pointed out above that attempts to measure $\mu$ by
sliding pure substances such as copper on copper will lead to
spurious results, because the surfaces in contact are not pure
copper, but are mixtures of oxides and other impurities. If we
try to get absolutely pure copper, if we clean and polish the
surfaces, outgas the materials in a vacuum, and take every
conceivable precaution, we still do not get $\mu$. For if we
tilt the apparatus even to a vertical position, the slider will
not fall off—the two pieces of copper stick together! The coefficient
$\mu$, which is ordinarily less than unity for reasonably hard
surfaces, becomes several times unity! The reason for this
unexpected behavior is that when the atoms in contact are all
of the same kind, there is no way for the atoms to “know” that
they are in different pieces of copper. When there are other
atoms, in the oxides and greases and more complicated thin surface
layers of contaminants in between, the atoms “know” when they are
not on the same part. When we consider that it is forces between
atoms that hold the copper together as a solid, it should become
clear that it is impossible to get the right coefficient of
friction for pure metals.
\end{quote}

The \href{https://en.wikipedia.org/wiki/Normal\_force}{normal force}
is the vertical component of the contact force. An object on a table
is affected by the Earth gravity, but naturally resists going through the
table, unless of course, the object is really massive and/or the table
very weak. There's a bunch of complicated interactions at the atomic
level from which this situation occurs: at a macroscopic scale, we
simplify things and wrap this complexity by abstracting it as
a (macroscopic) force.

\begin{remark} In the case of the vertical setup, with a mass attached
to a vertical spring, the mass is not in contact with a ground surface,
so there would have been no need to discuss contact forces.
\end{remark}

\begin{remark} A static object on a flat surface corresponds to a
special case where the horizontal component of the contact force (friction)
is null. Hence, the contact force and the normal force are, in such a special
case, one and the same. The exact same thing would happen in the
case of a frictionless surface.
\end{remark}

\hr
\textbf{Newton's laws} \\
Let us start by recalling
\href{https://en.wikipedia.org/wiki/Newton\%27s\_laws\_of\_motion}
{Newton's laws of motion}:

\begin{enumerate}
	\item \textbf{Principle of inertia}: Every body continues in
	its state of rest, or of uniform motion in a straight line,
	unless it is compelled to change that state by forces
	impressed upon it;
	\item "$\bm{F} = m\bm{a}$": The change of motion
	[\textit{momentum}] of an object is proportional to the force
	impressed; and is made in the direction of the straight line in
	which the force is impressed;
	\item "action $\Rightarrow$ reaction":
	To every action there is always opposed an equal reaction;
	or, the mutual actions of two bodies upon each other are always
	equal, and directed to contrary parts.
\end{enumerate}

\begin{remark} A force always is applied from an object, to another object;
with the restriction that an object cannot apply a force to itself.
\end{remark}

\begin{remark}\label{rem:L03E04:fma} The second law is a actually a statement
about momentum, from which we can derive $\bm{F}=m\bm{a}$.
More precisely, the momentum, called "motion" by Newton,
is defined as $\bm{p} = m\bm{v}$, which literally captures the idea of
the speed $\bm{v}$ of a certain amount of matter $m$. Then, it follows
that the general case, $\bm{p}$ is as $\bm{v}$ a function of time.

The "change of motion" then refers to an infinitesimal change of
the momentum over time, mathematically captured by the time derivative
$\dfrac{d}{dt}\bm{p}$. The law can then
be progressively written, assuming the mass $m$ is constant over time:
\[
	\bm{F} = \frac{d}{dt}\bm{p}
	= \frac{d}{dt}m\bm{v}
	= m\frac{d}{dt}\bm{v}
	= m\bm{a}\qed
\]
\end{remark}

\begin{remark}\label{rem:L03E04:relmass} It may seem weird to mention
in the previous Remark \ref{rem:L03E04:fma} that we assume the mass $m$
to be constant over time. In the current context, this will be the case,
but in general, likely contrary to what any reasonable human would expect,
they are exceptions. More precisely,
in special relativity, the mass is a function of the velocity $v$:
\[
	m_{\text{rel}} = \frac{m}{\sqrt{1-\dfrac{v^2}{c^2}}}
\]
We won't delve into the details here, but note that if $v=\norm{\bm{v}}$
is much smaller that $c$, the speed of light, then the relative
mass $m_{\text{rel}}$ and $m$ are very close.\\

To say it otherwise, this assumption doesn't hold at high velocities.
\end{remark}

\begin{remark} In the second law, the force impressed ("$\bm{F}$") is
actually the force resulting from all the other forced applied to an
object, $\sum_i\bm{F}_i$, which is sometimes called $\bm{F}_\text{net}$. \\

For instance, if you push a cart, $\bm{F}_\text{net}$ should contain
at least the force you are exerting, the gravity exerted by Earth, and
the contact force generated by the ground on the cart's wheel.
The sum of \textit{all the forces applied to the cart} that will
ultimately influence its motion. \\

As a result, in practice, when analyzing a situation so as to establish
the equation of motion of an object, one should start by identifying
all the external forces applied to this object.
\end{remark}

\begin{remark} A common special case of the second law is to consider
objects that, within a certain frame of reference, do not move. Hence,
by definition, their speed $\bm{v}$ would be $0$, and so would be their
acceleration. That is, $\sum_i\bm{F}_i = \bm{0}$. \\

Another special case for the second law is when things
move at constant speed. Again, the acceleration would then be $\bm{0}$
and thus still $\sum_i\bm{F}_i = \bm{0}$.
\end{remark}

\begin{remark}\label{rem:L03E04:mass} It is interesting to spend a moment to ponder
what the mass $m$ of an object really "is". Intuitively, for a non-physicist
mass is another word for weight, but not to a physicist. \\

Instead, in the context of the second law, physicists will observe
that the more mass there is, the more force will be needed to accelerate
an object. Conversely, the less mass there is, the less force will be needed to
alter the motion of an object. Thus, they would conceptualize the mass
as a \textbf{measure of resistance to acceleration} (change of movement). \\

This is the second time we meet a subtlety regarding masses, the
first one being Remark \ref{rem:L03E04:relmass} about how mass
is a function of the velocity. We'll come back later once again in a
moment at some more subtlety regarding mass.
\end{remark}

\begin{remark} In the case of a static object laying on a flat surface, one
may think that the gravitational force exerted by the Earth on the object and
the normal force/contact force arising from the contact of the object with
the surface are two opposite forces, in the sense of Newton's third law: after
all, they are indeed of equal intensity and opposite directions. \\

But this is \textit{incorrect}: in such a situation, we actually
have two pairs of opposite forces, in the sense of Newton's third law:
\begin{enumerate}
	\item $\bm{F}_{o,e}$ and $\bm{F}_{e,o}$, respectively the gravitational
	force exerted by the object on Earth, and the gravitational force
	exerted by Earth on the object;
	\item $\bm{F}_{o,t}$ and $\bm{F}_{t,o}$, respectively the contact
	force exerted by the object on the table, and the contact force
	exerted by the table on the object.
\end{enumerate}

Now to be more complete, there's even a third pair of forces, which would
be in almost all practical situations negligible, namely, the gravitational
forces exerted by the surface on the object, and by the object on the
surface. \\

A good exercise to check one understanding of this topic would be to try
to identify forces in the case of a hand pressed on a (static) wall, or
in the case of two hands pressed against each other.
\end{remark}

\begin{remark} Newton's third law is actually a consequence of the
principle of \href{https://en.wikipedia.org/wiki/Conservation\_of\_momentum}
{conservation of momentum}: essentially, in a closed system, the total
momentum is a constant over time. \\

This is a key idea in modern physics, and will be key later in the
book.
\end{remark}

\hr
\textbf{Hooke's law} \\
\href{https://en.wikipedia.org/wiki/Hooke\%27s\_law}{Hooke's law},
as Newton's laws, is an \textit{empirical law}, that is, which has been found
by repeated experimentation. Essentially, it states that the force needed to
extend or compress a spring by a given distance varies linearly with the
distance during which the spring is extended/compressed. That linear factor
$k$ \textit{is} the spring's constant. This force is thus a function
of that distance/displacement $\bm{\delta}$:
\[
	\bm{F}_s(\bm{\delta}) = k \bm{\delta}
\]
The equation could be read as something like:
\begin{itemize}
	\item If I don't compress/stretch the spring ($\bm{\delta}=\bm{0}$),
	no force is exerted;
	\item The more I compress/stretch the spring, the greater the force,
	where the relation between both varies linearly.
\end{itemize}

\begin{remark} In most practical cases, $\bm{\delta}$ will be a
function of time $\bm{\delta}(t)$; thus, the force itself will
be a function of time: $\bm{F}_s(\bm{\delta}(t))$.
\end{remark}

\begin{remark} By Newton's third law, if there's an object
exerting a force on the spring, then the spring also exerts a force
on the object, of equal intensity, but in the opposite direction. Hence,
equivalently, Hooke's law could be reformulated in the form presented
in the book:
\[
	\bm{F}_s(\bm{\delta}) = - k \bm{\delta}
\]
\end{remark}

\begin{remark}\label{rem:L03E04:origin} You may sometimes find that last
version of the law, as is the case in the book, written as
$\bm{F}_s(\bm{x}) = -k\bm{x}$.
This is a convenient choice that physicists can make, where, by
identifying the origin of the coordinate system as the rest position
of the mass, the measure of displacement $\bm{\delta}$ will indeed
always match the position vector $\bm{x}$. \\

This can be summarized by saying that $bm{F}_s$ is the the force pulling
the mass back to its rest position. \\
\end{remark}

\begin{remark} If you have a spring or two around (architect lamps are
a good source) and some regular weight (small tableware, such as iron tea
cups would do), you should be able to easily verify this law experimentally.
In this case, the objects' weight will be the source of a gravitational
force (we'll come back to the gravitational force in a moment): doubling
the mass (attaching two cups) should double the deformation of the spring.

\begin{figure}[H]
	\centering
	\begin{tikzpicture}
		\tikzmath{
			% ground's width
			\gw = 0.5;
		}
		\coordinate (borig)  at (-2,0);
		\coordinate (rbmass) at (0,-2);
		\coordinate (bmass)  at ($(borig)+(rbmass)$);

		\coordinate (orig)  at (0,0);
		\coordinate (rmass) at (0,-4.5);
		\coordinate (mass)  at ($(orig)+(rmass)$);

		\coordinate (aorig)  at (2,0);
		\coordinate (ramass) at (0,-7);
		\coordinate (amass)  at ($(aorig)+(ramass)$);

		% ground start/end
		\coordinate (gstart) at ($(borig)+(-1,0)$);
		\coordinate (gend)   at ($(aorig)+(1,0)$);

		% mass, amass
		\filldraw[color=black!60, fill=black!5, thick] (mass)  circle (0.5) node {$m$};
		\filldraw[color=black!60, fill=black!5, thick] (amass) circle (0.7) node {$2m$};

		% ground
		\draw[very thick] (gstart) -- (gend);
		\fill[pattern=north east lines] ($(gstart)+(0,\gw)$) rectangle (gend);

		% springs
		\draw[lightgray,decoration={aspect=0.3,segment length=1mm,amplitude=3mm,coil},decorate] (borig) -- (bmass);
		\draw[lightgray,decoration={aspect=0.3,segment length=3mm,amplitude=3mm,coil},decorate] (orig)  -- (mass);
		\draw[lightgray,decoration={aspect=0.3,segment length=6mm,amplitude=3mm,coil},decorate] (aorig) -- (amass);

		% length annotations
		\draw[decorate,gray,decoration={brace,amplitude=2mm,mirror}]
			(bmass) -- ($(borig)+(rmass)$)
			node[gray, midway, xshift=-10] {$l$};
		\draw[decorate,gray,decoration={brace,amplitude=2mm,mirror}]
			($(borig)+(rmass)$) -- ($(borig)+(ramass)$)
			node[gray, midway, xshift=-10] {$l$};

		% horizontal axis to increase length annotation readability
		\draw[dashed,lightgray] ($(rmass)+(borig)+(-1,0)$) -- ($(rmass)+(aorig)+(1,0)$);
		\draw[dashed,lightgray] ($(ramass)+(borig)+(-1,0)$) -- ($(ramass)+(aorig)+(1,0)$);

	\end{tikzpicture}
\end{figure}
\end{remark}

\begin{remark} Were you at least to imagine the previous experiment,
you would realize that there must so some limits to this law: the spring
could break or be irremediably deformed. Hooke's law is actually
a \href{https://en.wikipedia.org/wiki/Taylor\_series}
{first-order linear approximation}, that works reasonably
well in plenty of practical cases. There are more advanced models who
generalize Hooke's law, but they are of no use to us in this context.
\end{remark}

\hr
\textbf{Forces diagram} \\
The goal of a force diagram, or a
\href{https://en.wikipedia.org/wiki/Free\_body\_diagram}
{free body diagram}, is to identify the forces acting on an object,
with the intent of applying Newton's second law, so as to later
determine an equation of movement for that object: we want to identify
the sum of all the forces \textbf{acting on the object},
$\bm{F}_\text{net}=\sum_i\bm{F}_i$, who are likely to contribute
to the shape of its trajectory. Let's complete the diagram of
our horizontal spring setup with:

\begin{description}
	\item[$\bm{F}_{e,o}$] the gravitational force exerted by the earth on
	our object. Note that we haven't talked about gravitation yet, and
	as you'll see in a moment, we don't need to, but this will be necessary
	for the vertical spring setup;
	\item[$\bm{F}_{t,o}$] the normal force exerted by the table on our
	object. Because the surface is frictionless, the normal force
	\textit{is} the contact force;
	\item[$\bm{F}_{s,o}$] the force exerted by the spring on our object,
	which can be described by Hooke's law.
\end{description}

\begin{figure}[H]
	\centering
	\begin{tikzpicture}
		\coordinate (orig) at (0,0);
		\coordinate (mass) at (5,0);

		% mass's center, where Feo will be applied; slightly
		% off-center (+0.1) so we can show Fto
		\coordinate (mcenter) at ($(mass)+(0.05,0)$);
		\coordinate (Feo) at (0,-2); % relative to mcenter

		\coordinate (mtcenter) at ($(mass)+(-0.05,-0.5)$);
		\coordinate (Fto) at (0,2); % relative to mtcenter

		\coordinate (mscenter) at ($(mass)+(-0.05,0)$);
		\coordinate (Fso) at (-3,0); % relative to mscenter

		% mass
		\filldraw[color=black!60, fill=black!5, thick] (mass) circle (0.5) node [above right] {$m$};

		% x-axis
		\draw[dashed,gray,->] (orig) -- (7,0) node[gray,right]{$\vec{x}$};

		% frictionless ground
		\draw[very thin,lightgray] (0,-0.5) -- (7,-0.5);

		% spring
		\draw[lightgray,decoration={aspect=0.3,segment length=3mm,amplitude=3mm,coil},decorate] (orig) -- (mass);

		% ground
		\draw[very thick] (0,1) -- (0,-1);
		\fill[pattern=north east lines] (-1,-1) rectangle (0,1);

		% gravitational force
		\draw[very thick,->] (mcenter) to ["$\bm{F}_{e,o}$"] ($(mcenter)+(Feo)$);

		% normal/contact force
		\draw[very thick,->] (mtcenter) to ["$\bm{F}_{t,o}$"] ($(mtcenter)+(Fto)$);

		% spring's force
		\draw[very thick,->] (mscenter) to ["$\bm{F}_{s,o}$"] ($(mscenter)+(Fso)$);

	\end{tikzpicture}
\end{figure}

\textbf{Placement of the origin of the $\vec{x}$-axis} As we've
already observed before, it can be interesting to identify the origin
of our coordinate system to the position where the system will be at
equilibrium/rest. \\

\textbf{Intensities and signs} Thanks the previous convention, as discussed
in Remark \ref{rem:L03E04:origin}, Hooke's
law can be used to determine the force exerted by the spring on
the object, and can be written as $\bm{F}_{s,o} = -k\bm{x}$. \\

We know, from Newton's second law, that the contact/normal
force exerted by the table on the object is equal to $-\bm{F}_{o,t}$,
where $\bm{F}_{o,t}$ is the force exerted by the object on the table,
which itself results from the gravitational force $\bm{F}_{e,o}$, i.e.
\[
	\bm{F}_{t,o} = -\bm{F}_{o,t} = -\bm{F}_{e,o}
\]

As we'll see quickly, we don't actually need to compute anything
more, in this situation. We'll nevertheless show how to compute the
gravitational force in the next section, when considering the
similar case of a mass attached to a vertical spring. \\

\textbf{Applying Newton's second law}  We can now use Newton's second
law, $\sum_i \bm{F}_i = m\bm{a} = m\ddot{\bm{x}}$ to deduce an
equation of motion:

\begin{equation*} \begin{aligned}
	~ && \bm{F}_{e,o} + \bm{F}_{t,o} + \bm{F}_{s,o} &= m\ddot{\bm{x}} \\
	\Leftrightarrow && \bm{F}_{e,o} - \bm{F}_{e,o} + \bm{F}_{s,o} &= m\ddot{\bm{x}} \\
	\Leftrightarrow && \bm{F}_{s,o} &= m\ddot{\bm{x}} \\
	\Leftrightarrow && -k\bm{x} &= m\ddot{\bm{x}} \qed
\end{aligned} \end{equation*}

Which is a \href{https://en.wikipedia.org/wiki/Linear\_differential\_equation\#Second\-order\_case}
{second order, linear differential equation}, homogeneous, with constant
coefficients. It's often rewritten, as is the case in the book,
by defining $\omega^2 = \dfrac{k}{m}$:

\[ \boxed{\ddot{\bm{x}} = -\omega^2\bm{x}} \]

\hr
\textbf{Gravitational force} \\
To handle the case of a vertical spring, we won't be able to brush
the gravitational force away as we did previously, and we'll need to
compute it. \\

There are a few different ways to reach a formula for the
gravitational force, but perhaps the simplest one, in the context of
studying a mass attached to a vertical spring, would be to empirically
study the fall of an object of mass $m$ in Earth's gravitational
field. \\

One could proceed for instance by video-taping a vertically falling
object of known mass $m$. The video can then be discretized into a series
of images, on which we could measure the evolution of the position
of the mass over time. From there, we could compute the speed, as the
variation of position, and the acceleration, as the variation of
speed. \\

Then, neglecting the air resistance (well, we \textit{could} perform
the experiment in a  \href{https://en.wikipedia.org/wiki/Vacuum}{vacuum}),
we can pretend that there's a single force acting on the mass, which
\textit{is} our gravitational force. \\

The experiment could obviously be performed multiple times to
improve the accuracy. In the end, physicists have found that
Earth's gravity is an acceleration vector $\bm{g}$ of intensity
$9.81\,m.s^{-2}$, directed towards the center of the Earth, which
experimentally gives a gravitational force of:
\[ \bm{F}_{e,o} = m\bm{g} \]

\begin{remark}\label{rem:test} A home-made reenactment of the experiment would be
difficult, especially if there's a need for strong precision, which
would involve the presence of a vacuum. At least, more difficult
than experimentally validating Hooke's law on a single spring. \\

Historically, a simpler and more primitive setup was employed, for
instance by Galileo, which would involved measuring the speed of
a ball rolling on an inclined plane. A great way to lower the speed
of the "fall" and thus to get more accurate time measurements.
\end{remark}

\begin{remark} 1) Note that $\bm{g}$ captures not only
the gravitational force exerted by the Earth, but also captures
the contribution of the Earth's rotation (centrifugal force). \\

2) Because the Earth is not a perfect sphere, the value of
$\norm{\bm{g}}$ is not uniform everywhere on Earth; there are
noticeable differences from city to city for instance.
\end{remark}

\begin{remark} In Remark \ref{rem:L03E04:mass}, we conceptualized the
mass as a measure of resistance to acceleration. Physicists call
this the \textit{inertial mass}. And they conceptually distinguish
it from what they call the \textit{gravitational mass}, which
is the property of matter that qualify how an object will interact
with a gravitational field. \\

We tacitly assumed that both are the same: this is
the \href{https://en.wikipedia.org/wiki/Equivalence_principle}
{equivalence principle}, a cornerstone of modern theories
of gravities. So far, physicists haven't been able to distinguish
the two masses experimentally either. But conceptually, they refer
to two distinct notions. \\

Mathematically, this means that in the case of a falling object
subject to a single force exerted by gravitation, by applying
Newton's second law, we have:
\begin{equation*} \begin{aligned}
	~ && \bm{F}_\text{net} = \bm{F}_{e,o} &= m_i\bm{a} \\
	\Leftrightarrow && m_g\bm{g} &= m_i\bm{a} \\
	\Leftrightarrow && \bm{g} &= \bm{a}
\end{aligned} \end{equation*}

% TODO: maybe later
\iffalse
Incidentally, the vertical spring device that we're about to
study can be used to compute as an experimental setup to try
to measure a difference between the two masses, which we'll
present at some point later. However, to get there, we will
need to find the equation of motion for the mass attached
to the spring first.
\fi
\end{remark}

\begin{remark} 1) Another approach could have been to rely on
\href{https://en.wikipedia.org/wiki/Newton\%27s\_law\_of\_universal\_gravitation}
{Newton's law of universal gravitation}.
\[
	\bm{F}_{e,o} = -G\frac{m_e m_o}{\norm{\bm{r}_{e,o}}^2}\hat{\bm{r}}_{e,o}
\]
But not only is this law also empirical, but it would involve knowing
the mass of the Earth $m_e$, the gravitational constant $G$, etc. \\

2) Similarly we could delve into
\href{https://en.wikipedia.org/wiki/General\_relativity}
{Einstein's theory of general relativity}, but the prerequisites
would again have been more sophisticated than exposing a basic
experiment.
\end{remark}

\hr
\textbf{Vertical spring} \\
We know have all we need to study a variant of our previous setup,
slightly more complicated mathematically, where a mass $m$ is
attached to the end side of a vertical spring of constant $k$, itself
firmly attached to a fixed element at its top:
\begin{figure}[H]
	\centering
	\begin{tikzpicture}
		\coordinate (orig) at (0,0);
		\coordinate (mass) at (0,-5);

		\coordinate (Feo) at (0,-2); % relative to mass
		\coordinate (Fso) at (0,3); % relative to mass

		% mass
		\filldraw[color=black!60, fill=black!5, thick] (mass) circle (0.5) node [right] {$m$};

		% x-axis
		\draw[dashed,gray,->] (orig) -- ($(mass)+(0,-2)$) node[gray,right]{$\vec{z}$};

		% ground
		\draw[very thick] (-1,0) -- (1,0);
		\fill[pattern=north east lines] (-1,0.5) rectangle (1,0);

		% spring
		\draw[lightgray,decoration={aspect=0.3,segment length=3mm,amplitude=3mm,coil},decorate] (orig) -- (mass);

		% small visual dot to make forces clearer
		\filldraw[black] (mass) circle (0.05);

		% gravitational force
		\draw[very thick,->] (mass) to ["$\bm{F}_{e,o}$"] ($(mass)+(Feo)$);

		% spring's force
		\draw[very thick,->] (mass) to ["$\bm{F}_{s,o}$"] ($(mass)+(Fso)$);

	\end{tikzpicture}
\end{figure}
We'll choose the same convention regarding the origin of our coordinate system
(see Remark \ref{rem:L03E04:origin}). Contrary to the previous case, there's
no other force to counterbalance the gravitational force $\bm{F}_{e,o}$;
applying Newton's second law gives:
\begin{equation*} \begin{aligned}
	~ && \sum_i \bm{F}_i = \bm{F}_{e,o} + \bm{F}_{s,o} &= m\ddot{\bm{x}} \\
	\Leftrightarrow && m\bm{g} - k\bm{x} &= m\ddot{\bm{x}} \\
	\Leftrightarrow && \ddot{\bm{x}} &= -\omega^2\bm{x}+\bm{g}
\end{aligned} \end{equation*}
Again, with $\omega^2 = \dfrac{k}{m}$; so the only difference is
that we get an additional constant $\bm{g}$, but this still is a
second-order linear differential equation with constant coefficients.

\begin{remark} However, this differential equation is said to be
\textit{non-homogeneous} because of the non-zero constant $\bm{g}$, by
opposition with the previous \textit{homogeneous} equation obtained
from the horizontal spring setup.
\end{remark}
\hr
\textbf{Verifying the solution} \\
Solving differential equations, mathematically, can be rather
involved. For such simple equations, mathematicians themselves
may find it sufficient to make an educated guess by tweaking
an exponential-based function, despite having explored the
subject with much more depth. \\

We won't discuss the details here, and will pragmatically
satisfy ourselves with verifying that the proposed solution actually
solves the equation:
\[x(t) = A\cos\omega t + B\sin\omega t \]
\begin{remark} We've been before expressing all our equations with
vectors so far, but because we're working in one-dimension, we can
divide everything by $\bm{\hat{x}}$ and work from the resulting
scalar equations.
\end{remark}
And indeed, the proposed solution does solve the equation
for the horizontal spring setup.
\begin{align*}
	\ddot x(t) &= -A\omega^2\cos\omega t - B\omega^2\sin\omega t \\
	~ &= -\omega^2 (A\cos\omega t + B\sin\omega t) \\
	~ &= -\omega^2 x(t) \qed
\end{align*}

But it doesn't solve the vertical setup; we'll get there quickly,
this isn't much more complicated. You may want to try by yourself
to see if you can tweak the solution of the homogeneous equation
to solve the non-homogeneous one.

\hr
\textbf{Initial conditions} \\
Let's start by computing the expression of the velocity:
\[ \dot x(t) = -A\omega\sin\omega t + B\omega\cos\omega t \]
Then, at $t=0$, we have:
\begin{equation*} \begin{aligned}
	x(0) = && A\cos(0) + B\sin(0) &= \boxed{B} \\
	\dot x(0) = && -A\omega\sin(0) + B\omega\cos(0) &= \boxed{A\omega}\qed
\end{aligned} \end{equation*}

\hr
\textbf{Equations and initial conditions for the vertical setup} \\
There are general mathematical methods for reaching a solution to a
non-homogeneous linear second-order differential equation with
constant coefficient from a homogeneous, but again, suffice for us
to verify that the following would work:
\[x(t) = A\cos\omega t + B\sin\omega t + \frac{g}{\omega^2} \]
Indeed:
\begin{align*}
	\ddot x(t) &= -A\omega^2\cos\omega t - B\omega^2\sin\omega t \\
	~ &= -A\omega^2\cos\omega t - B\omega^2\sin\omega t -\frac{\omega^2}{\omega^2}g + g\\
	~ &= -\omega^2 (A\cos\omega t + B\sin\omega t + \frac{g}{\omega^2}) + g \\
	~ &= -\omega^2 x(t) +g\qed
\end{align*}
From there, we can again express the initial position and speed
as functions of $A$ and $B$, first by computing the expression of the speed:
\[ \dot x(t) = -A\omega\sin\omega t + B\omega\cos\omega t \]
And then by looking at what's happening at $t=0$:
\begin{equation*} \begin{aligned}
	x(0) = && A\cos(0) + B\sin(0)+\frac{g}{\omega^2} &= \boxed{B+\frac{g}{\omega^2}} \\
	\dot x(0) = && -A\omega\sin(0) + B\omega\cos(0) &= \boxed{A\omega}\qed
\end{aligned} \end{equation*}

\iffalse
\hr
\textbf{Inertial mass vs. gravitational mass}
%https://www.youtube.com/watch?v=Ws3yB3QsKY4
\fi

\end{document}
