\documentclass[solutions.tex]{subfiles}

\xtitle

\begin{document}
\maketitle
\begin{exercise} Prove Eq. $(3)$. \textit{Hint: Use the product rule
for differentiation.}
\end{exercise}
This is a very simple differentiation exercise. I'm going to be very slow,
because using a "normal" pace would essentially mean stating the result,
as it's been done in the book, hence negating the need for this exercise.
If you can't compute this kind of derivative in your head at this stage,
I would recommend practising until it becomes second nature. For instance,
\textit{Paul's Online Notes}\footnote{
\url{https://tutorial.math.lamar.edu/Problems/CalcI/DerivativeIntro.aspx}}
has plenty of corrected exercises on differentiation. \\

Let's recall the context a little bit: we're in the case of a single
particle moving along the $\vec{x}-axis$ under the influence of a force
$F(x)$. The \textit{kinetic energy} of that particle, that is, the energy
that the particle has because of its motion, is noted $T$ and is "defined"
as:
\[
	T = \frac12mv^2
\]
Eq. $(3)$ refers to the time derivative of $v^2$, in the context of
its usage in the formulation of kinetic energy. There are (at least)
two ways to evaluate this derivative; let's start by using the
authors' hint regarding the product rule; let's make the time
dependency more obvious:
\[
	\frac{d}{dt}v(t)^2 = \frac{d}{dt}v(t)v(t)
\]
Now let's recall the \textit{product rule} is, for $\varphi$ and
$\psi$ two real-valued functions of $t$:
\[
	(\varphi\psi)' = \varphi'\psi + \varphi\psi'
\]
Hence, in the case of $\varphi=\psi=v$, we have:
\begin{equation*} \begin{aligned}
	\frac{d}{dt}v(t)^2 &&=\quad& \frac{d}{dt}v(t)v(t) \\
	~ &&=\quad&\dot{v}(t)v(t)+v(t)\dot{v}(t) \\
	~ &&=\quad&\boxed{2v\dot{v}(t)} \qed
\end{aligned} \end{equation*}
\begin{remark} We could have also used the \textit{chain rule};
for $\varphi$ a function of $t$, and $\psi$ and function whose domain (input)
is the codomain (output) of $\varphi$:
\[
	(\psi\circ\varphi)'(t) = (\psi(\varphi(t))' = \varphi'(t)\psi'(\varphi(t))
\]
In this case, $\psi=(x\mapsto x^2)$ and $\varphi=v$, so:
\begin{equation*} \begin{aligned}
	\frac{d}{dt}v(t)^2 &&=\quad& \frac{d}{dt}(x\mapsto x^2)(v(t)) \\
	~ &&=\quad&\dot{v}(t)(x\mapsto x^2)'(v(t)) \\
	~ &&=\quad&\dot{v}(t)(x\mapsto 2x)(v(t)) \\
	~ &&=\quad&\boxed{2v\dot{v}(t)} \qed
\end{aligned} \end{equation*}
\end{remark}
\begin{remark} As a reminder, both the product rule and the chain rule have
been proved in
\href{https://github.com/mbivert/ttm/blob/master/cm/L02E04.pdf}{L02E04}. \\
\end{remark}
\end{document}
