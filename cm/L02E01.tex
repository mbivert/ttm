\documentclass[solutions.tex]{subfiles}

\xtitle

\begin{document}
\maketitle
\begin{exercise}
Calculate the derivatives of each of these functions.

\begin{equation*} \begin{aligned}
	f(t) &&=\quad& t^4 + 3t^3 - 12t^2 + t - 6 \\
	g(x) &&=\quad& \sin x - \cos x \\
	\theta(\alpha) &&=\quad& e^\alpha + \alpha\ln\alpha \\
	x(t) &&=\quad& \sin^2t - \cos t
\end{aligned} \end{equation*}
\end{exercise}
\begin{remark} Those are exactly the functions graphed
in\href{https://github.com/mbivert/ttm/blob/master/cm/I01E01.pdf}{I01E01}
\end{remark}
\hr
$\bm{f(t) = t^4 + 3t^3 - 12t^2 + t - 6}$ \\, \\
To find the derivative of $f$, we need to apply four rules that
were mentioned (the first one was proved) in the book:
\begin{enumerate}
	\item the formula for the derivative of a general power;
	for $n\in\mathbb{N}$:
	\[
		\frac{d}{dt}(t^n) = n t^{n-1}
	\]
	\item the fact that the derivative of a constant is zero;
	for $c\in\mathbb{R}$:
	\[
		\frac{d}{dt}c = 0
	\]
	\item the fact that the derivative of a constant times
	a function is the same as the constant times the derivative
	of the function; for $c\in\mathbb{R}$ and $\varphi$ a function
	of $t$:
	\[
		\frac{d}{dt}(c\varphi) = c\frac{d}{dt}\varphi
	\]
	\item the \textit{sum rule} (also referred to as the
	\textit{linearity of differentiation} in mathematics); for
	both $\varphi$ and $\psi$ functions of $t$:
	\[
		\frac{d}{dt}(\varphi+\psi) = \frac{d}{dt}\varphi+\frac{d}{dt}\psi
	\]
\end{enumerate}
\begin{remark} Two simpler notations are common to denote the derivative
of a function of a single variable:
\[
	\frac{d}{dt}\varphi = \varphi' = \dot\varphi
\]
While the first one is used mostly in mathematics, for abstract functions,
the second one is used almost exclusively in physics, to denote
a \textit{time} derivative. We'll use them both in such ways from
now on. For instance, as this exercise is rather mathematical, we'll
use the prime notation. \\

While we're here, bear in mind that the prime notation can also be used
to denote the derivative of a more or less complex expression, wrapped
in parenthesis, e.g.:
\[
	(\varphi(x)+\psi(x)+\cos^2 x)' = \frac{d}{dx}(\varphi(x)+\psi(x)+\cos^2 x)
\]
\end{remark}
By application of the \textit{sum rule}, the rule regarding the derivative
of a constant, and the rule regarding a function multiplied by a constant:
\[
	f'(t) = (t^4)' + 3(t^3)'-12(t^2)'+(t)'-0
\]
Then, it's just a matter of applying the formula for the derivative of a general
power to each individual term:
\[
	\boxed{f'(t) = 4t^3 + 9t^2-24t+1}
\]

\hr
$\bm{g(x) = \sin x - \cos x}$ \\, \\
In addition the the previously mentioned \textit{sum rule}, we will also
need two more rules to compute the derivative of $g$, also presented in
the book, regarding the derivative of $\sin$ and $\cos$:
\begin{equation*} \begin{aligned}
	\cos'(t) &&=\,& -\sin(t); &&&
	\sin'(t) &&=\,& \cos(t)
\end{aligned} \end{equation*}
We then have successively:
\begin{equation*} \begin{aligned}
	g'(x) &&=\,& \sin'(x)-\cos'(x) \\
	~ &&=\,& \boxed{\cos(x)+\sin(x)}
\end{aligned} \end{equation*}

\hr
$\bm{\theta(\alpha) = e^\alpha + \alpha\ln\alpha}$ \\, \\
To compute $\theta'$, in addition to the \textit{sum rule}
and the formula for a general power (with $n=1$), we need two
additional rules pertaining to the derivatives of both the
exponential and the log:
\begin{equation*} \begin{aligned}
	(e^t)' &&=\quad& e^t; &&&
	\ln'(t) &&=\quad& \frac{1}{t}
\end{aligned} \end{equation*}
And the \textit{product rule}; for $\varphi$ and $\psi$ functions
of $t$:
\[
	\frac{d}{dt}(\varphi\psi) = \varphi'\psi + \varphi\psi'
\]
Where all those rules were mentioned in the book. We then obtain:
\begin{equation*} \begin{aligned}
	\theta'(\alpha) &&=\,& (e^\alpha)'+(\alpha\ln(\alpha))' \\
	~ &&=\,& e^\alpha+(\frac{d}{d\alpha}\alpha)\ln(\alpha)+\alpha \ln'(\alpha) \\
	~ &&=\,& \boxed{e^\alpha+\ln(\alpha)+1}
\end{aligned} \end{equation*}

\hr
$\bm{x(t) = (\sin t)^2 - \cos t}$ \\, \\
Besides the \textit{sum rule}, the rules regarding the derivative
of $\cos$ and $\sin$, and the formula for the derivative of a general
power, we will only need a single new rule to compute $x'$, the \textit{chain rule};
for $\varphi$ a function of $t$, and $\psi$ and function whose domain (input)
is the codomain (output) of $\varphi$:
\[
	\frac{d}{dt}(\psi\circ\varphi) = \frac{d}{dt}(\psi(\varphi(t)) = \varphi'(t)\psi'(\varphi(t))
\]
Then,
\begin{equation*} \begin{aligned}
	x'(t) &&=\,& ((\sin t)^2)'-(\cos t)' \\
	~ &&=\,& (\sin t)'(u \mapsto u^2)'(\sin t) + \sin t \\
	~ &&=\,& \cos t (u \mapsto 2u)(\sin t) + \sin t \\
	~ &&=\,& 2\cos t\sin t + \sin t \\
	~ &&=\,& \boxed{(1+2\cos t)\sin t} \\
\end{aligned} \end{equation*}
\begin{remark} We used a bit of mathematical notation to avoid us the need
to explicitly name the function which squares its argument. More explicitly,
to compute the derivative of $\mu(t) = \sin^2t$, we could have define
$\nu(u) = u^2$. Then, $\mu(t) = \nu(\sin(t))$, and $\nu'(u) = 2u$;
by the \textit{chain rule}
\[
	(\sin^2 t)' = \mu'(t) = (\sin t)'\nu'(\sin(t)) = 2\cos t\sin t
\]
\end{remark}
\begin{remark} Instead of using the \textit{chain rule} to compute the
derivative of $\sin^2t$, we could instead have used the \textit{product rule}:
\[
	(\sin^2t)' = (\sin t \times \sin t)' = (\sin t)' \sin t + \sin t(\sin t)'
		= 2\sin t(\sin t)' = 2\sin t\cos t
\]
\end{remark}
\end{document}
