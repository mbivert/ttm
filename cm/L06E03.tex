\documentclass[solutions.tex]{subfiles}

\xtitle

\begin{document}
\maketitle
\begin{exercise}
Use the Euler-Lagrange equations to derive the equations of motions
from the Lagrangian in Eq. $(12)$.
\end{exercise}
Again, let us recall the general form of Euler-Lagrange equations
for a configuration space of size $n\in\mathbb{N}$: $(\forall i \in\llbracket 1, n\rrbracket)$,
\begin{align}
	\frac{d}{dt}\left(\frac{\partial}{\partial\dot{x_i}}L\right)
		= \frac{\partial}{\partial x_i}L
	\label{eqn:l06e03:euler-lagrange}
\end{align}

In the case of this exercise, the Lagrangian $L$ is defined in Eq. $(12)$ as:
\begin{align}
	L = \frac{m}{2}(\dot{X}^2+\dot{Y}^2)
	  + \frac{m\omega^2}{2}(X^2+Y^2)
	  + m\omega(\dot{X}Y-\dot{Y}X) \nonumber
\end{align}

Let's compute the partial derivatives of $L$ on $\dot{X}$, $X$, $\dot{Y}$ and $Y$:
\begin{align}
	\frac{\partial}{\partial\dot{X}}L &=
		\frac{\partial}{\partial\dot{X}}
			\left(\frac{m}{2}\dot{X}^2 + m\omega\dot{X}Y\right)
	& \frac{\partial}{\partial X}L &=
		\frac{\partial}{\partial X}
			\left(\frac{m\omega^2}{2}X^2 - m\omega\dot{Y}X\right)
		\nonumber \\
	~ &= m\dot{X} + m\omega Y & ~ &= m\omega^2 X - m\omega\dot{Y} \nonumber\\
	~ & ~ & ~ & \nonumber \\
	\frac{\partial}{\partial\dot{Y}}L &=
		\frac{\partial}{\partial\dot{Y}}
			\left(\frac{m}{2}\dot{Y}^2 - m\omega\dot{Y}X\right)
	& \frac{\partial}{\partial Y}L &=
		\frac{\partial}{\partial Y}
			\left(\frac{m\omega^2}{2}Y^2 + m\omega\dot{X}Y\right)
		\nonumber \\
	~ &= m\dot{Y} - m\omega X & ~ &= m\omega^2 Y + m\omega\dot{X}
	\label{eqn:l06e03:euler-lagrange-split}
\end{align}

Finally, by plugging \eqref{eqn:l06e03:euler-lagrange-split} into
\eqref{eqn:l06e03:euler-lagrange}, we obtain:
\begin{align*}
	~ & \frac{d}{dt}\left(m\dot{X} + m\omega Y\right) = m\omega^2 X - m\omega\dot{Y}
	& ~ & \frac{d}{dt}\left(m\dot{Y} - m\omega X\right) = m\omega^2 Y + m\omega\dot{X}\\
	\Leftrightarrow & \boxed{m\ddot{X} = m\omega^2X-2m\omega\dot{Y}}
	& ~ \Leftrightarrow & \boxed{m\ddot{Y} = m\omega^2Y+2m\omega\dot{X}} \qed
\end{align*}
\begin{remark} Those results indeed matches the equations proposed
in the book just slightly before this exercise.
\end{remark}

\end{document}
