\documentclass[solutions.tex]{subfiles}

\xtitle

\begin{document}
\maketitle
\begin{exercise}
Can you think of a general way to classify the laws
that are possible for a six-state system?
\end{exercise}
From what precedes, we can think of classifying systems
by their number of cycles. We could furthermore refine
the classification, for instance by considering the
distribution of states per cycle. \\

The official solutions\footnote{\url{http://www.madscitech.org/tm/slns/l1e2.pdf}}
mention an interesting fact, not directly apparent from the book's
examples: there could be cycles within the "main" cycles (and cycles
within the "secondary" cycles, etc.), which can also be used to refine
the classification.

\end{document}