\documentclass[solutions.tex]{subfiles}

\title{L07E01}

\begin{document}
\maketitle
\begin{exercise}
Derive Equations $(2)$ and explain the sign difference.
\end{exercise}

Let us recall Equations $(2)$:
\begin{align*}
	\dot{p_1} &= -V'(q_1-q_2) &
	\dot{p_2} &= +V'(q_1-q_2)
\end{align*}

We have to derive them from the Lagrangian given in Equation $(1)$,
which represents a system of two generalized coordinates $q_1$
and $q_2$:
\begin{align}
	L = \frac{1}{2}(\dot{q_1}^2+\dot{q_2}^2) - V(q_1-q_2)
	\label{eqn:l07e01:lagrangian}
\end{align}

To retrieve the equations of motions from a Lagrangian, we need
to use Euler-Lagrange's equations, for instance recalled as
Equation $(13)$ of the previous chapter ("Lecture $6$:
The Principle of Least Action"):
\[
	\frac{d}{dt}\biggl(\frac{\partial}{\partial \dot{q_i}}L\biggr)
	= \frac{\partial}{\partial q_i}L
\]

Let us also recall, again from previous chapter, right after
Equation $(13)$, that the conjugate momentum is defined by
\[
	p_i = \frac{\partial}{\partial \dot{q_1}}L
\]

For our Lagrangian \eqref{eqn:l07e01:lagrangian}, we have
for the first half of Euler-Lagrange equations:
\begin{align}
	p_1 &\equiv \frac{\partial}{\partial \dot{q_1}}L = \dot{q_1} &
	p_2 &\equiv \frac{\partial}{\partial \dot{q_2}}L = \dot{q_2} \label{eqn:l07e01:p1} \\
	\frac{d}{dt}p_1 &= \dot{p_1} = \ddot{q_1} &
	\frac{d}{dt}p_2 &= \dot{p_2} = \ddot{q_2} \label{eqn:l07e01:p2}
\end{align}
Using the chain
rule\footnote{\url{https://en.wikipedia.org/wiki/Chain\_rule}}
for the other half, with $\varphi(q_i) = q_1-q_2$, we get:
\begin{align}
	\frac{\partial}{\partial q_1}L &=
		-\frac{\partial}{\partial q_1}V(\varphi(q_1)) &
	\frac{\partial}{\partial q_2}L &=
		-\frac{\partial}{\partial q_2}V(\varphi(q_2)) \nonumber \\
	~ &= -\frac{\partial}{\partial q_1}\varphi(q_1)
		\frac{\partial}{\partial q_1}V(\varphi(q_1)) &
	~ &= -\frac{\partial}{\partial q_2}\varphi(q_2)
		\frac{\partial}{\partial q_2}V(\varphi(q_2)) \nonumber \\
	~ &= -(\frac{\partial}{\partial q_1}V)(q_1-q_2) &
	~ &= +(\frac{\partial}{\partial q_2}V)(q_1-q_2) \label{eqn:l07e01:p3}
\end{align}

By noting $V' = \dfrac{\partial}{\partial q_i}V$, and combining equations
$\eqref{eqn:l07e01:p1}$, $\eqref{eqn:l07e01:p2}$ and $\eqref{eqn:l07e01:p3}$,
we indeed obtain the expected equations of motion $\qed$.

\begin{remark}
That is, assuming, $\dfrac{\partial}{\partial q_1}V(q_1) =
\dfrac{\partial}{\partial q_2}V(q_2)$: for all the energy
potential presented earlier in the book, there's indeed such
a symmetry, e.g.

\begin{align*}
	V &= \frac{1}{2}k(x^2+y^2),& p103 \\
	V &= \frac{1}{2}\frac{k}{x^2+y^2},& p103 \\
	V &= -m \omega^2 (X^2+Y^2),& p120 \\
\end{align*}

A similar tacit assumption seems to exists in Herbert Goldstein's
\textit{Classical Mechanics}\footnote{\url{https://physics.stackexchange.com/a/107141}}.

\end{remark}

Mathematically, the sign difference comes from the fact that the
potential depends on one side from $q_1$ and on the other from $-q_2$,
which will persist when differentiating the potential $V$. Physically,
it reflects that there's an order relation between the two "positions"
$q_1$ and $q_2$: one will come before the other, and our potential $V$
depends on this ordering.

\end{document}
