\documentclass[solutions.tex]{subfiles}

\xtitle

\begin{document}
\maketitle
\begin{exercise} The derivative of a derivative is called the
second derivative and is written $\dfrac{d^2 f(t)}{dt^2}$. Take
the second derivative of each of the functions listed above.
\end{exercise}
\begin{remark} As for the first derivative, there are
two common notations for second derivatives:
\[
	\frac{d^2}{dt^2}\varphi = \varphi'' = \ddot\varphi
\]
Again, the dot notation is used in physics to denote time differentiation,
while the prime notation is often used in mathematics, in a more abstract context.
\end{remark}
Let's start by recalling the functions:
\begin{equation*} \begin{aligned}
	f(t) &&=\quad& t^4 + 3t^3 - 12t^2 + t - 6 \\
	g(x) &&=\quad& \sin x - \cos x \\
	\theta(\alpha) &&=\quad& e^\alpha + \alpha\ln\alpha \\
	x(t) &&=\quad& \sin^2t - \cos t
\end{aligned} \end{equation*}

And their derivatives, computed in the previous exercise:

\begin{equation*} \begin{aligned}
	f'(t)  &&=\quad& 4t^3 + 9t^2-24t+1 \\
	g'(x) &&=\quad& \cos(x)+\sin(x) \\
	\theta'(\alpha) &&=\quad& e^\alpha+\ln(\alpha)+1 \\
	x'(t) &&=\quad& (1+2\cos t)\sin t
\end{aligned} \end{equation*}

While in the previous exercise
(\href{https://github.com/mbivert/ttm/blob/master/cm/L02E01.pdf}{L02E01})
the derivation were rather slow and detailed, because the process
is essentially the same, we're going to go (much) faster here, the most
difficult part being in the application of the product rule for $x''$.
\begin{equation*} \begin{aligned}
	f''(t)  &&=\quad& \boxed{12t^2 + 18t-24} \\
	g''(x) &&=\quad& \boxed{\cos(x)-\sin(x) = -g(x)} \\
	\theta''(\alpha) &&=\quad& \boxed{e^\alpha+\frac{1}\alpha} \\
	x''(t) &&=\quad& (1+2\cos t)\cos t -2\sin t\sin t \\
	~ &&=\quad& \boxed{\cos t +2(\cos^2 t - \sin^2t)}
\end{aligned} \end{equation*}
\begin{remark} $x''$ could be slightly improved by using
the trigonometric identity $\cos^2x-\sin^2x = \cos2x$, which hasn't
been introduced in the book.
\end{remark}

\end{document}
