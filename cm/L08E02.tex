\documentclass[solutions.tex]{subfiles}

\xtitle

\begin{document}
\maketitle
\begin{exercise}
Starting with Eq. $(14)$, calculate the Hamiltonian in terms of $p$
and $q$.
\end{exercise}

Again Eq. $(14)$ of the book refers to the following Lagrangian:
\[
	L = \frac{1}{2\omega}\dot{q}^2 - \frac{\omega}{2}q^2
\]

There are two different ways to proceed here, depending on
what we take as our definition of the Hamiltonian. If we
consider $H$ to be the energy of the system, i.e. $K+V$,
the sum of the kinetic and potential energies, then we need
to identify them in the Lagrangian, typically defined as $K-V$. \\

Now we "know" the kinetic/potential energy formulation for
a basic harmonic oscillator, and the previous change of variable
we performed to get the Lagrangian doesn't really affect them,
fundamentally, so we know that the first term of the Lagrangian
is the kinetic energy, while the second is the potential energy. \\

So this gives us first:
\[
	H = \frac{1}{2\omega}\dot{q}^2 + \frac{\omega}{2}q^2
\]

Then, we need to recall that $p$ is defined by:

\[
	p = \frac{\partial L}{\partial\dot{q}} = \frac{1}\omega\dot{q}
\]

Hence, $\dot{q} = \omega p$, which we can inject in the previous
version of $H$ to get:

\[
	H = \frac{1}{2\omega}(\omega p)^2 + \frac{\omega}{2}q^2
		= \boxed{\frac12\omega(p^2+q^2)}
\]

As stated earlier, there's another approach, if we were to start
from the definition of the Hamiltonian given by Eq. $(4)$ of this chapter:
\[
	H = \sum_i (p_i \dot{q_i})-L = p\dot{q}-L
		= p\dot{q}-\frac{1}{2\omega}\dot{q}^2 + \frac{\omega}{2}q^2
\]

Then we can again inject $\dot{q} = \omega p$ to get:
\[
	H = p (p \omega) - \frac{1}{2\omega}(\omega p)^2+\frac\omega2q^2
		= \omega p^2 - \frac12\omega p^2+\frac\omega2q^2
		= \frac\omega2 p^2 + \frac\omega2q^2 = \boxed{\frac\omega2(p^2+q^2)}
\]

\end{document}
