\documentclass[solutions.tex]{subfiles}

\xtitle

\begin{document}
\maketitle
\begin{exercise} Show that the position and velocity
vectors from Eq.s $(3)$ are orthogonal.
\end{exercise}
Actually, Eq.s $(3)$ refers to velocity and acceleration,
not position. Because of this ambiguity, let's look for
which pair of vectors are orthogonal among the three.
Let's start by recalling how they have been defined (knowing
that velocity and acceleration are obtained by differentiating
position respectively once and twice):
\begin{equation*} \begin{aligned}
	r_x(t) &=& R\cos(\omega t)&;\quad& r_y(t) &=& R\sin(\omega t) \\
	v_x(t) &=& -R\omega\sin(\omega t);&\quad& v_y(t) &=& R\omega\cos(\omega t) \\
	a_x(t) &=& -R\omega^2\cos(\omega t);&\quad& a_y(t) &=& -R\omega^2\sin(\omega t)
\end{aligned} \end{equation*}

We've established in
\href{https://github.com/mbivert/ttm/blob/master/cm/I01E06.pdf}{I01E06}
that two vectors are orthogonals if their dot product is zero; where
the dot product has been defined as:
\begin{equation*} \begin{aligned}
	\bm{u}\cdot\bm{v} &&=\quad& u_xv_x + u_yv_y + u_zv_z \\
	~ &&=\quad& \norm{\bm{u}}\norm{\bm{v}}\cos\theta_{uv}
\end{aligned} \end{equation*}

Then, let's compute a few dot products (we're in the plane, so the
 $z$ components must be zero):
\begin{equation*} \begin{aligned}
	\bm{r}\cdot\bm{v} &&=\quad& r_xv_x + r_yv_y \\
	~ &&=\quad& R\cos(\omega t)\times(-R\omega\sin(\omega t))
		+ R\sin(\omega t)\times R\omega\cos(\omega t) \\
	~ &&=\quad& \boxed{0} \\
	\bm{r}\cdot\bm{a} &&=\quad& r_xa_x + r_ya_y \\
	~ &&=\quad& R\cos(\omega t)\times(-R\omega^2\cos(\omega t))
		+ R\sin(\omega t)\times(-R\omega^2\sin(\omega t)) \\
	~ &&=\quad& -R^2\omega^2\underbrace{
		(\cos^2(\omega t)+\sin^2(\omega t))
	}_{=1} \\
	~ &&=\quad& \boxed{-(R\omega)^2 \neq 0} \\
	\bm{v}\cdot\bm{a} &&=\quad& v_xa_x + v_ya_y \\
	~ &&=\quad& (-R\omega\sin(\omega t))\times(-R\omega^2\cos(\omega t))
		+ R\omega\cos(\omega t)\times(-R\omega^2\sin(\omega t)) \\
	~ &&=\quad& \boxed{0}
\end{aligned} \end{equation*}

Hence both position and acceleration are orthogonals with velocity.

\begin{remark} Regarding $\bm{r}\cdot\bm{a}$, we could also have
observed that $\bm{a} = -\omega^2\bm{r}$: the vectors are colinear,
so they simply can't be orthogonals. From there, were we to
already have established $\bm{r}\cdot\bm{v}=0$, we could have
inferred $\bm{v}\cdot\bm{a}=-\omega^2\bm{v}\cdot\bm{r}=0$, using
the (bi)linearity of the dot product.
\end{remark}
\end{document}