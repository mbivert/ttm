\documentclass[solutions.tex]{subfiles}

\xtitle

\begin{document}
\maketitle
\begin{exercise}
Since the notion is so important to theoretical physics, think
about what a closed system is, and speculate on whether closed
systems can actually exist. What assumptions are implicit in
establishing a closed system? What is an open system?
\end{exercise}

The book defines a closed system as "a collection of objects [...]
that is either the entire universe $(1)$ or is so isolated from everything
else that it behaves as if nothing else exists $(2)$". From everyday
experience, we \textit{know} there's a lot going on we're not aware of.
Thus, we know we can't even hope to \textit{truly} consider the entire
universe, but in an abstract sense, at best. \\

Furthermore, to establish with absolute certainty $(2)$,
it is necessary to have a full understanding of everything that
exists. For, even were we to build a system isolated from everything
we know that could affect it, by the previous assumption, there may be
some elements, unknown to us, affecting the system in various ways. \\

Hence, as for a lot of things in physics, a closed system is but a
conceptual tool, a (very) convenient approximation of some aspects
of what we think we experience of reality.

\hrr

To establish our closed system, we must generally assume that it is
closed \textit{relatively} to what we are trying to measure/observe.
That is, there will be no external factors noticeably/unexpectedly
affecting the result of a a given observation/consideration. \\

\hrr

As we can only define a closed system as an approximation, we won't
bother trying to define an \textit{open system} in absolute terms
either. A opened system, is by contrast with a closed
system, a collection of objects that isn't neither the entire Universe,
nor so isolated from everything else that it behaves as if nothing
else exists. More simply, it's a collection of objects affected
by its environment.

\end{document}
