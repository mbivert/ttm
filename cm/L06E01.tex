\documentclass[solutions.tex]{subfiles}

\title{The Theoretical Minimum \\
	{\Large Classical Mechanics - Solutions} \\
	{\large L06E01}
}

\begin{document}
\maketitle
\begin{exercise}
Show that Eq. $(4)$ is just another form of Newton's
equation of motion $F = ma$.
\end{exercise}
Where Eq. $(4)$ are the freshly derived Euler-Lagrange
equations of motions:
\begin{align}
	\frac{d}{dt}\frac{\partial}{\partial\dot{x}}L - \frac{\partial}{\partial x}L = 0
	\label{eqn:l06e01:euler-lagrange}
\end{align}
In the context of a single particle moving in one dimension, with
kinetic and potential energy given by:
\begin{align*}
	T &= \frac{1}{2} m\dot{x}^2 \\
	V &= V(x)
\end{align*}
From which results the Lagrangian:
\begin{align}
	L &= T-V \nonumber\\
	~ &= \frac{1}{2}m\dot{x}^2-V(x)
	\label{eqn:l06e01:lagrangian}
\end{align}

Let us recall that we also have the \textit{potential energy
principle}, stated in one-dimension as Eq. $(1)$ of the previous
chapter, \textit{Lecture 5: Energy}:
\begin{align}
	F(x) = -\frac{d}{dx}V(x)
	\label{eqn:l06e01:potential}
\end{align}

Which is also stated more generally in that same chapter, for an
abstract configuration space $\{x\} = \{x_i\}$, as Eq. $(5)$:
\[
	F_i(\{x\}) = -\frac{\partial}{\partial x_i}V(\{x\})
\]

Thus, deriving each part of \eqref{eqn:l06e01:euler-lagrange} with
our Lagrangian \eqref{eqn:l06e01:lagrangian}, and considering the
\textit{definition} of a potential energy $V(x)$ \eqref{eqn:l06e01:potential}
yields:
\begin{align*}
	\frac{d}{dt}\frac{\partial}{\partial\dot{x}}L &=
		\frac{d}{dt}m\dot{x} &
	\frac{\partial}{\partial x}L &=
		\frac{\partial}{\partial x}V(x) \\
	~ &= m\ddot{x} & ~ &= -F
\end{align*}

Then indeed, Euler-Lagrange equations become equivalent to Newton's law of motion:
\begin{align*}
	~ & \frac{d}{dt}\frac{\partial}{\partial\dot{x}}L - \frac{\partial}{\partial x}L = 0 \\
	\Leftrightarrow & m\ddot{x}-(-F) = 0 \\
	\Leftrightarrow & \boxed{F = m\ddot{x} = ma} \qed
\end{align*}
\end{document}