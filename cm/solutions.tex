\documentclass[a4paper]{article}

% TODO: improve solutions.tex exercice title formatting

\author{M. Bivert}
\title{The Theoretical Minimum \\
	{\Large Classical Mechanics - Solutions}
}

\usepackage{subfiles}

\usepackage[margin=1in]{geometry}
\usepackage{hyperref}

\usepackage[utf8]{inputenc}

% \noindent everywhere
\setlength\parindent{0pt}

\usepackage{mathtools}
\usepackage{amsmath}
\usepackage{amsthm}
\usepackage{amsfonts}
\usepackage{bm}

\usepackage{stmaryrd}

\usepackage{float} % \begin{figure}[H]

\usepackage{titlesec}
\usepackage{titletoc}

\usepackage{tikz}
\usetikzlibrary{snakes,calc,patterns,angles,quotes,decorations.pathmorphing,math,decorations.pathreplacing}

\newtheorem{exercise}{Exercise}
\newtheorem{axiom}{Axiom}
\newtheorem{remark}{Remark}

\newcommand{\hr}{\noindent\rule{\textwidth}{0.4pt} \\}
\newcommand{\hrr}{\begin{center}\noindent\rule{0.5\textwidth}{0.4pt} \\\end{center}}

\newcommand{\bra}[1]{\langle #1|}
\newcommand{\ket}[1]{|#1\rangle}
\newcommand{\braket}[2]{\left\langle #1 \middle| #2 \right\rangle}
\newcommand\omicron{o}
\newcommand\Omicron{O}
\newcommand{\norm}[1]{\lVert #1 \rVert}

% \title for subfiles
\newcommand{\xtitle}[1]{\title{The Theoretical Minimum \\
	{\Large Classical Mechanics - Solutions} \\
	{\large #1}
}}

% Essentially, https://tex.stackexchange.com/a/329928
\newcounter{interludes}
\renewcommand{\theinterludes}{\arabic{interludes}}
\newcommand{\interlude}[1]{\refstepcounter{interludes}
	\section*{Interlude \theinterludes: #1}
	\addcontentsline{toc}{section}{Interlude \theinterludes: #1}
}

% Awful way to alter both the ToC and *section formats.
\newcounter{lectures}
\renewcommand{\thelectures}{\arabic{lectures}}
\newcommand{\lecture}[1]{\refstepcounter{lectures}
	\section*{Lecture \thelectures: #1}
	\addcontentsline{toc}{section}{Lecture \thelectures: #1}
}

\newcounter{sublectures}
\renewcommand{\thesublectures}{\arabic{sublectures}}
\newcommand{\sublecture}[1]{\refstepcounter{sublectures}
	\subsection*{#1}
	\addcontentsline{toc}{subsection}{#1}
}

\newcounter{subsublectures}
\renewcommand{\thesubsublectures}{\arabic{subsublectures}}
\newcommand{\subsublecture}[1]{\refstepcounter{subsublectures}
	\subsubsection*{#1}
	\addcontentsline{toc}{subsubsection}{#1}
}

\begin{document}
\maketitle
\begin{abstract}
Below are solution proposals to the exercises of
\textit{The Theoretical Minimum - Classical Mechanics}, written
by Leonard Susskind and George Hrabovsky. An effort has been
so as to recall from the book all the referenced equations,
and to be rather verbose regarding mathematical details, rather
in line with the general tone of the serie.
\end{abstract}

\tableofcontents

% --------------------------------------------------------
\lecture{The Nature of Classical Physics}

% --------------------------------------------------------
\interlude{Spaces, Trigonometry, and Vectors}

% --------------------------------------------------------
\lecture{Motion}

% --------------------------------------------------------
\interlude{Integral Calculus}

% --------------------------------------------------------
\lecture{Dynamics}
\sublecture{Aristotle's Law of Motion}
\subsublecture{Exercise 1/4}
\subfile{L03E01.tex}
\sublecture{Mass, Acceleration and Force}
\sublecture{Some Simple Examples of Solving Newton's Equations}
\subsublecture{Exercise 2/4}
\subfile{L03E02.tex}
\subsublecture{Exercise 3/4}
\subfile{L03E03.tex}
\subsublecture{Exercise 4/4}
\subfile{L03E04.tex}

% --------------------------------------------------------
\interlude{Partial Differentiation}

% --------------------------------------------------------
\lecture{Systems of More Than One Particle}

% --------------------------------------------------------
\lecture{Energy}
\sublecture{More than one dimension}
\subsublecture{Exercise 2/3}
\subfile{L05E02.tex}
\subsublecture{Exercise 3/3}
%\subfile{L05E03.tex}

% --------------------------------------------------------
\lecture{The Principle of Least Action}
\sublecture{The Transition to Advanced Mechanics}
\sublecture{Action and the Lagrangian}

\sublecture{Derivation of the Euler-Lagrange Equation}
\subsublecture{Exercise 1/6}
\subfile{L06E01.tex}

\sublecture{More Particles and More dimensions}
\subsublecture{Exercise 2/6}
\subfile{L06E02.tex}

\sublecture{What's Good about Least Action?}
\subsublecture{Exercise 3/6}
\subfile{L06E03.tex}
\subsublecture{Exercise 4/6}
\subfile{L06E04.tex}

\sublecture{Generalized Coordinates and Momenta}
\subsublecture{Exercise 5/6}
%\subfile{L06E05.tex}
\sublecture{Cyclic coordinates}
\subsublecture{Exercise 6/6}
\subfile{L06E06.tex}

% --------------------------------------------------------
\lecture{Symmetries and Conservation Laws}
\sublecture{Preliminaries}
\subsublecture{Exercise 1/7}
\subfile{L07E01.tex}
\subsublecture{Exercise 2/7}
\subfile{L07E02.tex}
\sublecture{Examples of symmetries}
\subsublecture{Exercise 3/7}
\subfile{L07E03.tex}
\subsublecture{Exercise 4/7}
\subfile{L07E04.tex}
\sublecture{Back to examples}
\subsublecture{Exercise 5/7}
%\subfile{L07E05.tex}
%\subsublecture{Exercise 6/7}
%\subfile{L07E06.tex}
%\subsublecture{Exercise 7/7}
%\subfile{L07E07.tex}

\end{document}
