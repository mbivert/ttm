\documentclass[a4paper]{article}

\newcommand{\vtitle}{Special Relativity and Classical Field Theory}

% define \vtitle{} (volume title) before \input{}ing

\author{M. Bivert}
\title{The Theoretical Minimum \\
	{\Large \vtitle{} - Solutions} \\
	{\footnotesize Last version: %
		\href{https://tales.mbivert.com/on-the-theoretical-minimum-solutions/}%
		{tales.mbivert.com/on-the-theoretical-minimum-solutions/} or %
		\href{https://github.com/mbivert/ttm}{github.com/mbivert/ttm}}
}

\usepackage{subfiles}

\usepackage[margin=1in]{geometry}
\usepackage{hyperref}

% \noindent everywhere
\setlength\parindent{0pt}

\usepackage{mathtools}
\usepackage{amsmath}
\usepackage{amsthm}
\usepackage{amsfonts}
\usepackage{bm}
\usepackage{amssymb}

\usepackage{stmaryrd}

\usepackage{float} % \begin{figure}[H]

\usepackage{tikz}
\usetikzlibrary{
	snakes,calc,patterns,angles,quotes,
	decorations.pathmorphing,math,
	decorations.pathreplacing,automata,
	arrows.meta,positioning,external
}
\usepackage{pgfplots}
\tikzexternalize[prefix=tikz/]
\pgfplotsset{compat=newest}

\usepackage{bashful}
\usepackage{listings}

\newcommand{\hr}{\noindent\rule{\textwidth}{0.4pt} \\}
\newcommand{\hrr}{\begin{center}\noindent\rule{0.5\textwidth}{0.4pt} \\\end{center}}

\newtheorem{axiom}{Axiom}
\newtheorem{definition}{Definition}
\newtheorem{remark}{Remark}
\newtheorem{theorem}{Theorem}
\newtheorem{lemma}{Lemma}
\newtheorem{exercise}{Exercise}
\newtheorem{example}{Example}

\newcommand{\bra}[1]{\langle #1|}
\newcommand{\ket}[1]{|#1\rangle}
\newcommand{\brah}[1]{\{#1|}
\newcommand{\keit}[1]{|#1\}}
\newcommand{\braket}[2]{\left\langle #1 \middle| #2 \right\rangle}
\newcommand{\ketbra}[2]{\left| #1 \right\rangle\left\langle #2 \right|}
\newcommand{\innerprod}[2]{\left\langle #1 , #2 \right\rangle}
\newcommand{\avg}[1]{\left\langle #1 \right\rangle}
\newcommand{\Tr}{\text{Tr}}
\newcommand\omicron{o}
\newcommand\Omicron{O}
\newcommand{\uvec}[1]{\bm{\hat{#1}}}
\newcommand{\norm}[1]{\lVert #1 \rVert}
\newcommand{\abs}[1]{\lvert #1 \rvert}
\newcommand{\harpoon}{\overset{\rightharpoonup}}
\newcommand{\intint}[1]{\llbracket #1 \rrbracket}

% \title for subfiles
\newcommand{\xtitle}{\title{The Theoretical Minimum \\
	{\Large \vtitle{} - Solutions} \\
	{\large \jobname} \\
	{\footnotesize Last version: %
		\href{https://tales.mbivert.com/on-the-theoretical-minimum-solutions/}%
		{tales.mbivert.com/on-the-theoretical-minimum-solutions/} or %
		\href{https://github.com/mbivert/ttm}{github.com/mbivert/ttm}}
}}


\begin{document}
\maketitle
\begin{abstract}
Below are solution proposals to the exercises of
\textit{The Theoretical Minimum - \vtitle{}}, written
by Leonard Susskind and Art Friedman. An effort has been
so as to recall from the book all the referenced equations,
and to be rather verbose regarding mathematical details,
hopefully in line with the general tone of the series.
\end{abstract}

\tableofcontents

\section{The Lorentz Transformation}
\subsection{Reference Frames}
\subsection{Inertial Reference Frames}
\subsubsection{Newtonian (Pre-SR) Frames}
\subfile{L01E00.tex}
\subsubsection{SR Frames}
\paragraph{Synchronizing Our Clocks}
\paragraph{Units and Dimensions: A Quick Detour}
\paragraph{Setting Up Our Coordinates --- Again!}
\paragraph{Back to the Main Road}
\paragraph{Finding the $x'$ Axis}
\paragraph{Spacetime}

\paragraph{Lorentz Transformations}
\subsubsection{Historical Aside}
\subsubsection{Back to the Equations}
\paragraph{Switching to Conventional Units}
\paragraph{The Other Two Axes}
\subsubsection{Nothing Moves Faster than Light}
\subsection{General Lorentz Transformation}
\subsection{Length Contraction and Time Dilation}
\paragraph{Length Contraction}
%\subfile{L01E01.tex}
\paragraph{Time Dilation}
\paragraph{The Twin Paradox}
%\subfile{L01E02.tex}
\paragraph{The Stretch Limo and the Bug}
\subsection{Minkowski's World}
\subsubsection{Minkowski and the Light Cone}
\subsubsection{The Physical Meaning of Proper Time}
\subsubsection{Spacetime Interval}
\subsubsection{Timelike, Spacelike, and Lightlike Separations}
\paragraph{Timelike Separation}
\paragraph{Spacelike Separation}
\paragraph{Lightlike Separation}
\subsection{Historical Perspective}
\subsubsection{Einstein}
\subsubsection{Lorentz}

\end{document}
