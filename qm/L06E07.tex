\documentclass[solutions.tex]{subfiles}

\xtitle

\begin{document}
\maketitle
\begin{exercise} Next, Charlie prepares the spins in a different
state, called $\ket{T_1}$, where
\[
	\ket{T_1} = \frac1{\sqrt2}\left(\ket{ud}+\ket{du}\right)
\]
In these examples, $T$ stands for $\textit{triplet}$. These triplet
states are completely different from the states in the coin and die
examples. What are the expectation values of the operators $\sigma_z\tau_z$,
$\sigma_x\tau_x$, and $\sigma_y\tau_y$? \\

What a difference a sign can make!
\end{exercise}
This is the same kind of computations there were done in the
\href{https://github.com/mbivert/ttm/blob/master/qm/L06E06.pdf}{previous exercise},
and earlier in the book. As usual, recall the Pauli matrices:
\[
	\tau_x = \sigma_x = \begin{pmatrix}
		0 & 1 \\
		1 & 0 \\
	\end{pmatrix};\qquad \tau_y = \sigma_y = \begin{pmatrix}
		0 & -i \\
		i & 0 \\
	\end{pmatrix};\qquad \tau_z = \sigma_z = \begin{pmatrix}
		1 & 0 \\
		0 & -1 \\
	\end{pmatrix}
\]
Also recall, from
\href{https://github.com/mbivert/ttm/blob/master/qm/L06E04.pdf}{L06E04},
the rules for acting on composite state vectors\footnote{You have the same
in the book's appendix}:
\begin{equation*}\begin{aligned}
	\sigma_z\ket{uu} &=&\ket{uu};  &&& \tau_z\ket{uu} &=&& \ket{uu} \\
	\sigma_z\ket{ud} &=&\ket{ud};  &&& \tau_z\ket{ud} &=&& -\ket{ud} \\
	\sigma_z\ket{du} &=&-\ket{du}; &&& \tau_z\ket{du} &=&& \ket{du} \\
	\sigma_z\ket{dd} &=&-\ket{dd}; &&& \tau_z\ket{dd} &=&& -\ket{dd} \\
	 \cline{3-6}
	\sigma_x\ket{uu} &=&\ket{du}; &&& \tau_x\ket{uu} &=&& \ket{ud} \\
	\sigma_x\ket{ud} &=&\ket{dd}; &&& \tau_x\ket{ud} &=&& \ket{uu} \\
	\sigma_x\ket{du} &=&\ket{uu}; &&& \tau_x\ket{du} &=&& \ket{dd} \\
	\sigma_x\ket{dd} &=&\ket{ud}; &&& \tau_x\ket{dd} &=&& \ket{du} \\
	 \cline{3-6}
	\sigma_y\ket{uu} &=&i\ket{du};  &&& \tau_y\ket{uu} &=&& i\ket{ud} \\
	\sigma_y\ket{ud} &=&i\ket{dd};  &&& \tau_y\ket{ud} &=&& -i\ket{uu} \\
	\sigma_y\ket{du} &=&-i\ket{uu}; &&& \tau_y\ket{du} &=&& i\ket{dd} \\
	\sigma_y\ket{dd} &=&-i\ket{ud}; &&& \tau_y\ket{dd} &=&& -i\ket{du} \\
\end{aligned}\end{equation*}

We now have everything we need to compute the expectation values.

\hrr
\begin{equation*}\begin{aligned}
	\avg{\sigma_z\tau_z} &:=&& \bra{T_1}\sigma_z\tau_z\ket{T_1} \\
	~ &=&& \frac1{\sqrt2}\bra{T_1}\sigma_z\tau_z\left(\ket{ud}+\ket{du}\right) \\
	~ &=&& \frac1{\sqrt2}\bra{T_1}\sigma_z\left(-\ket{ud}+\ket{du}\right) \\
	~ &=&& -\frac1{\sqrt2}\bra{T_1}\left(\ket{ud}+\ket{du}\right) \\
	~ &=&& -\frac12(\bra{ud}+\bra{du})(\ket{ud}+\ket{du}) \\
	~ &=&& -\frac12\left(
		\underbrace{\braket{ud}{ud}}_{1}
		+\underbrace{\braket{ud}{du}}_{0}
		+\underbrace{\braket{du}{ud}}_{0}
		+\underbrace{\braket{du}{du}}_{1}
	\right) \\
	~ &=&&\boxed{-1}
\end{aligned}\end{equation*}

For the last step, remember, as for the previous exercise, that $\ket{du}$
and $\ket{ud}$ are orthonormal basis vectors.

\hrr

\begin{equation*}\begin{aligned}
	\avg{\sigma_x\tau_x} &:=&& \bra{T_1}\sigma_x\tau_x\ket{T_1} \\
	~ &=&& \frac1{\sqrt2}\bra{T_1}\sigma_x\tau_x\left(\ket{ud}+\ket{du}\right) \\
	~ &=&& \frac1{\sqrt2}\bra{T_1}\sigma_x\left(\ket{uu}+\ket{dd}\right) \\
	~ &=&& \frac1{\sqrt2}\bra{T_1}\left(\ket{du}+\ket{ud}\right) \\
	~ &=&& \frac12(\bra{ud}+\bra{du})(\ket{du}+\ket{ud}) \\
	~ &=&& -\frac12\left(
		\underbrace{\braket{ud}{du}}_{0}
		+\underbrace{\braket{ud}{ud}}_{1}
		+\underbrace{\braket{du}{du}}_{1}
		+\underbrace{\braket{du}{ud}}_{0}
	\right) \\
	~ &=&&\boxed{+1}
\end{aligned}\end{equation*}

\hrr

\begin{equation*}\begin{aligned}
	\avg{\sigma_y\tau_y} &:=&& \bra{T_1}\sigma_y\tau_y\ket{T_1} \\
	~ &=&& \frac1{\sqrt2}\bra{T_1}\sigma_y\tau_y\left(\ket{ud}+\ket{du}\right) \\
	~ &=&& \frac1{\sqrt2}\bra{T_1}\sigma_y\left(-i\ket{uu}+i\ket{dd}\right) \\
	~ &=&& \frac{i}{\sqrt2}\bra{T_1}\left(-i\ket{du}-i\ket{ud}\right) \\
	~ &=&& \frac12(\bra{ud}+\bra{du})(\ket{du}+\ket{ud}) \\
	~ &=&& -\frac12\left(
		\underbrace{\braket{ud}{du}}_{0}
		+\underbrace{\braket{ud}{ud}}_{1}
		+\underbrace{\braket{du}{du}}_{1}
		+\underbrace{\braket{du}{ud}}_{0}
	\right) \\
	~ &=&&\boxed{+1}
\end{aligned}\end{equation*}

\end{document}
