\documentclass[a4paper]{article}

\author{M. Bivert}
\title{TTM - QM - Solutions}

\usepackage{subfiles}

\usepackage[margin=1in]{geometry}
\usepackage{hyperref}

% \noindent everywhere
\setlength\parindent{0pt}

\usepackage{mathtools}
\usepackage{amsmath}
\usepackage{amsthm}   % \qed
\usepackage{amsfonts} % \mathbb

\newtheorem{exercise}{Exercise}
\newtheorem{axiom}{Axiom}
\newtheorem{remark}{Remark}

\newcommand{\hr}{\noindent\rule{\textwidth}{0.4pt} \\}
\newcommand{\hrr}{\begin{center}\noindent\rule{0.5\textwidth}{0.4pt} \\\end{center}}

\newcommand{\bra}[1]{\langle #1|}
\newcommand{\ket}[1]{|#1\rangle}
\newcommand{\braket}[2]{\left\langle #1 \middle| #2 \right\rangle}
\newcommand\omicron{o}
\newcommand\Omicron{O}

\begin{document}
\maketitle
\begin{abstract}
Below are solution proposals to the exercises of
\textit{The Theoretical Minimum - Quantum Mechanics}, written
by Leonard Susskind and Art Friedman. An effort has been
so as to recall from the book all the referenced equations,
and to be rather verbose regarding mathematical details, rather
in line with the general tone of the serie.
\end{abstract}

\tableofcontents

\section{Systems and Experiments}
\subsection{Inner Products}
\subfile{L01E01.tex}
\subfile{L01E02.tex}
\section{Quantum States}
\subsection{Along the $x$ Axis}
\subfile{L02E01.tex}
\subsection{Along the $y$ Axis}
\subfile{L02E02.tex}
\subfile{L02E03.tex}
\section{Principles of Quantum Mechanics}
\subsection{Mathematical Interlude: Linear Operators}
\subsubsection{Hermitian Operators and Orthonormal Bases}
%\subfile{L03E01.tex}
\end{document}
