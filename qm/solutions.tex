\documentclass[a4paper]{article}

\author{M. Bivert}
\title{The Theoretical Minimum \\
	{\Large Quantum Mechanics - Solutions} \\
	{\footnotesize Last version: %
		\href{https://tales.mbivert.com/on-the-theoretical-minimum-solutions/}%
		{tales.mbivert.com/on-the-theoretical-minimum-solutions/} or %
		\href{https://github.com/mbivert/ttm}{github.com/mbivert/ttm}}
}

\usepackage{subfiles}

\usepackage[margin=1in]{geometry}
\usepackage{hyperref}

% \noindent everywhere
\setlength\parindent{0pt}

\usepackage{mathtools}
\usepackage{amsmath}
\usepackage{amsthm}
\usepackage{amsfonts}
\usepackage{bm}
\usepackage{amssymb}

\usepackage{stmaryrd}

\usepackage{float} % \begin{figure}[H]

\usepackage{tikz}
\usetikzlibrary{snakes,calc,patterns,angles,quotes,decorations.pathmorphing,math,decorations.pathreplacing,automata,arrows.meta,positioning,external}
\tikzexternalize[prefix=tikz/]

\newcommand{\hr}{\noindent\rule{\textwidth}{0.4pt} \\}
\newcommand{\hrr}{\begin{center}\noindent\rule{0.5\textwidth}{0.4pt} \\\end{center}}

\newtheorem{axiom}{Axiom}
\newtheorem{definition}{Definition}
\newtheorem{remark}{Remark}
\newtheorem{theorem}{Theorem}
\newtheorem{lemma}{Lemma}
\newtheorem{exercise}{Exercise}
\newtheorem{example}{Example}

\newcommand{\bra}[1]{\langle #1|}
\newcommand{\ket}[1]{|#1\rangle}
\newcommand{\brah}[1]{\{#1|}
\newcommand{\keit}[1]{|#1\}}
\newcommand{\braket}[2]{\left\langle #1 \middle| #2 \right\rangle}
\newcommand{\innerprod}[2]{\left\langle #1 , #2 \right\rangle}
\newcommand{\avg}[1]{\left\langle #1 \right\rangle}
\newcommand{\Tr}{\text{Tr}}
\newcommand\omicron{o}
\newcommand\Omicron{O}
\newcommand{\uvec}[1]{\bm{\hat{#1}}}
\newcommand{\norm}[1]{\lVert #1 \rVert}
\newcommand{\abs}[1]{\lvert #1 \rvert}
\newcommand{\harpoon}{\overset{\rightharpoonup}}
\newcommand{\intint}[1]{\llbracket #1 \rrbracket}

% \title for subfiles
\newcommand{\xtitle}{\title{The Theoretical Minimum \\
	{\Large Quantum Mechanics - Solutions} \\
	{\large \jobname} \\
	{\footnotesize Last version: %
		\href{https://tales.mbivert.com/on-the-theoretical-minimum-solutions/}%
		{tales.mbivert.com/on-the-theoretical-minimum-solutions/} or %
		\href{https://github.com/mbivert/ttm}{github.com/mbivert/ttm}}
}}

\begin{document}
\maketitle
\begin{abstract}
Below are solution proposals to the exercises of
\textit{The Theoretical Minimum - Quantum Mechanics}, written
by Leonard Susskind and Art Friedman. An effort has been
so as to recall from the book all the referenced equations,
and to be rather verbose regarding mathematical details, rather
in line with the general tone of the series.
\end{abstract}

\tableofcontents

\section{Systems and Experiments}
\subsection{Inner Products}
\subfile{L01E01.tex}
\subfile{L01E02.tex}
\section{Quantum States}
\subsection{Along the $x$ Axis}
\subfile{L02E01.tex}
\subsection{Along the $y$ Axis}
\subfile{L02E02.tex}
\subfile{L02E03.tex}
\section{Principles of Quantum Mechanics}
\subsection{Mathematical Interlude: Linear Operators}
\subsubsection{Hermitian Operators and Orthonormal Bases}
\subfile{L03E01.tex}
\subsubsection{The Gram-Schmidt Procedure}
\subsection{The Principles}
\subsection{An Example: Spin Operators}
\subsection{Constructing Spin Operators}
\subfile{L03E02.tex}
\subsection{A Common Misconception}
\subsection{$3$-Vector Operators Revisited}
\subsection{Reaping the Results}
\subfile{L03E03.tex}
\subfile{L03E04.tex}
\subfile{L03E05.tex}
\subsection{The Spin-Polarization Principle}
\section{Time and Change}
\subsection{A Classical Reminder}
\subsection{Unitarity}
\subsection{Determinism in Quantum Mechanics}
\subsection{A Closer Look at $U(t)$}
\subfile{L04E01.tex}
\subsection{The Hamiltonian}
\subsection{What Ever Happened to $\hbar$?}
\subsection{Expectation Values}
\subsection{Ignoring the Phase-Factor}
\subsection{Connections to Classical Mechanics}
\subfile{L04E02.tex}
\subfile{L04E03.tex}
\subsection{Conservation of Energy}
\subsection{Spin in a Magnetic Field}
\subfile{L04E04.tex}
\subsection{Solving the Schr\"odinger Equation}
\subfile{L04E05.tex}
\subsection{Recipe for a Schr\"odinger Ket}
\subfile{L04E06.tex}
\subsection{Collapse}
\section{Uncertainty and Time Dependence}
\subsection{Mathematical Interlude: Complete Sets of Commuting Variables}
\subsubsection{States That Depend On More Than One Measurable}
\subsubsection{Wave Functions}
\subsubsection{A Note About Terminology}
\subsection{Measurement}
\subfile{L05E01.tex}
\subsection{The Uncertainty Principle}
\subsection{The Meaning of Uncertainty}
\subsection{Cauchy-Schwarz Inequality}
\subsection{The Triangle Inequality and the Cauchy-Schwarz Inequality}
\subsection{The General Uncertainty Principle}
\subfile{L05E02.tex}
\section{Combining Systems: Entanglement}
\subsection{Mathematical Interlude: Tensor Products}
\subsubsection{Meet Alice and Bob}
\subsubsection{Representing the Combined System}
\subsection{Classical Correlation}
\subfile{L06E01.tex}
\subsection{Combining Quantum Systems}
\subsection{Two Spins}
\subsection{Product States}
\subfile{L06E02.tex}
\subsection{Counting Parameters for the Product State}
\subsection{Entangled States}
\subfile{L06E03.tex}
\subsection{Alice and Bob's Observables}
\subfile{L06E04.tex}
\subfile{L06E05.tex}
\subsection{Composite Observables}
\subfile{L06E06.tex}
\subfile{L06E07.tex}
\subfile{L06E08.tex}
\subfile{L06E09.tex}
\subfile{L06E10.tex}
\section{More on Entanglement}
\subsection{Mathematical Interlude: Tensor Products in Component Form}
\subsubsection{Building Tensor Product Matrices from Basic Principles}
\subsubsection{Building Tensor Product Matrices from Component Matrices}
\subfile{L07E01.tex}
\subfile{L07E02.tex}
\subfile{L07E03.tex}
\subsection{Mathematical Interlude: Outer Products}
\subsection{Density Matrices: A New Tool}
\subsection{Entanglement and Density Matrices}
\subsection{Entanglement for Two Spins}
\subfile{L07E04.tex}
\subfile{L07E05.tex}
\subfile{L07E06.tex}
\subsection{A Concrete Example: Calculating Alice's Density Matrix}
\subfile{L07E07.tex}
\subfile{L07E08.tex}
\subsection{Tests for Entanglement}
\subsubsection{The Correlation Test for Entanglement}
%\subfile{L07E09.tex}
\subsubsection{The Density Matrix Test for Entanglement}
\subsection{The Process of Measurement}
%\subfile{L07E10.tex}
\subsection{Entanglement and Locality}
\subsection{The Quantum Sim: An Introduction to Bell's Theorem}
\subsection{Entanglement Summary}
%\subfile{L07E11.tex}
%\subfile{L07E12.tex}
\section{Particles and Waves}
\end{document}
