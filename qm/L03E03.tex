\documentclass[solutions.tex]{subfiles}

\xtitle

\begin{document}
\maketitle
\begin{exercise}
Calculate the eigenvectors and eigenvalues of $\sigma_n$. \textit{Hint:}
Assume the eigenvector $\lambda_1$ has the form:
\[
	\begin{pmatrix}
		\cos\alpha \\
		\sin\alpha \\
	\end{pmatrix},
\]
where $\alpha$ is an unknown parameter. Plug this vector into the eigenvalue
equation and solve for $\alpha$ in terms of $\theta$. Why did we use a single
parameter $\alpha$? Notice that our suggested column vector must have unit
length.
\end{exercise}
Let's recall the context: we're trying to build an operator that allows
us to measure the spin of a particle. We've started by building the
components of such an operator, each representing our ability to measure
the spin along any of the $3D$ axes: $\sigma_x$, $\sigma_y$ and $\sigma_z$.
Each of them was built from the behavior of the spin we "measured": we
extracted from the observed behavior a set of constraints, which allowed
us to determine the components of the spin operator: \\

\[
	\sigma_x = \begin{pmatrix}
		0 & 1 \\
		1 & 0 \\
	\end{pmatrix};\qquad \sigma_y = \begin{pmatrix}
		0 & -i \\
		i & 0 \\
	\end{pmatrix};\qquad \sigma_z = \begin{pmatrix}
		1 & 0 \\
		0 & -1 \\
	\end{pmatrix}
\]

Those are individually fine to measure the spin components
along the $3$ main axis, but we'd like to measure spin components
along an arbitrary axis $\hat{n}$. Such a measure can be performed
by an operator constructed as a linear combination of the previous
three matrices:
\[
	\sigma_n = \bm{\sigma} \cdot \hat{n} = \begin{pmatrix}
		\sigma_x \\
		\sigma_y \\
		\sigma_z \\
	\end{pmatrix} \cdot \begin{pmatrix}
		n_x \\
		n_y \\
		n_z \\
	\end{pmatrix}
	= n_x\sigma_x + n_y\sigma_y + n_z\sigma_z
\]

\begin{remark} Remember from your linear algebra courses that
matrices can be added and scaled: they form a vector space.
\end{remark}

The present exercise involves an arbitrary spin vector, that is, a
linear combination of $\sigma_x$, $\sigma_y$ and $\sigma_z$ that is of the form:
\begin{equation*}\begin{aligned}
	\sigma_n &&=&&& \sin\theta\sigma_x + \cos\theta\sigma_z \\
	~        &&=&&& \sin\theta\begin{pmatrix}
		0 & 1 \\
		1 & 0 \\
	\end{pmatrix} + \cos\theta\begin{pmatrix}
		1 & 0 \\
		0 & -1 \\
	\end{pmatrix} \\
	~        &&=&&& \begin{pmatrix}
		\cos\theta & \sin\theta \\
		\sin\theta & -\cos\theta \\
	\end{pmatrix}
\end{aligned}\end{equation*}

We're then asked to look for the eigenvalues/eigenvectors of that matrix,
that is, we want to understand what kind of spin (states) can be encoded
by such a matrix, and which values they can take.

\hrr

Let's recall that to find the eigenvalues/eigenvectors,
we need to diagonalize the matrix: assuming it can be diagonalized, it
means that there's a basis where it can be expressed as a diagonal matrix;
the change of basis is encoded by a linear map, thus a matrix, and so
we must be able to find an invertible matrix $P$ and a diagonal matrix
$D$ such that:\footnote{This is "basic" linear algebra; the authors assume
that you're already familiar with it to some degree (e.g. matrix product);
don't hesitate to refer to a more thorough course on the subject for more.
I'll quickly review here how diagonalization works}
\[
	\sigma_n = P D P^{-1}
	\quad\Leftrightarrow\quad
	\sigma_n P = PD
\]
\[
	\Leftrightarrow
	\begin{pmatrix}
		\cos\theta & \sin\theta \\
		\sin\theta & -\cos\theta \\
	\end{pmatrix}
	\begin{pmatrix}
		a & b \\
		c & d \\
	\end{pmatrix} =
	\begin{pmatrix}
		a & b \\
		c & d \\
	\end{pmatrix}
	\begin{pmatrix}
		\lambda_1 & 0         \\
		0         & \lambda_2 \\
	\end{pmatrix} = 	\begin{pmatrix}
		\lambda_1 a & \lambda_2 b \\
		\lambda_1 c & \lambda_2 d \\
	\end{pmatrix}
\]

Where $\lambda_1$ and $\lambda_2$ would be the eigenvalues, associated
to the two eigenvectors:
\[
	\ket{\lambda_1} = \begin{pmatrix}
		a \\
		c \\
	\end{pmatrix};\qquad
	\ket{\lambda_2} = \begin{pmatrix}
		b \\
		d \\
	\end{pmatrix}
\]

Note that the previous equation implies that we must have:
\[
	(\forall i \in \{1,2\}),\ \sigma_n\ket{\lambda_i} = \lambda_i\ket{\lambda_i}
\]
Which is equivalent to saying, where $0_2$ is the zero $2\times 2$ matrix,
and $I_2$ the $2\times2$ identity matrix:
\[
	 \sigma_n\ket{\lambda_i} - \lambda_i\ket{\lambda_i} = 0_2
	 \quad\Leftrightarrow\quad
	 	(\sigma_n - I_2\lambda_i)\ket{\lambda_i} = 0_2
\]
If we want a non-trivial solution (i.e. $\ket{\lambda_i} \neq \bm{0}$), then
it follows that we must have:
\[
	\sigma_n - I_2\lambda_i = 0_2
\]
This means that the matrix $\sigma_n - I_2\lambda_i$ cannot be invertible
(for otherwise multiplying it by its inverse would yield, by the rule
of invertibility $I_2$, but  on the other side, from the matrix's definition,
it would yield $0_2$, hence a contradiction, hence it's not invertible). \\

Non-invertibility of a matrix translates to their determinant being zero,
which means the $\lambda_i$ solves the following equation for $\lambda$:
\begin{equation*}\begin{aligned}
	\det(\sigma_n - I_2\lambda) = 0
	&&\Leftrightarrow&&&
	\begin{vmatrix}
		\cos\theta-\lambda & \sin\theta \\
		\sin\theta & -\cos\theta-\lambda \\
	\end{vmatrix} = 0 \\
	&&\Leftrightarrow&&&
		-(\cos\theta-\lambda)(\cos\theta+\lambda) - \sin^2\theta = 0 \\
	&&\Leftrightarrow&&&
		-(\cos^2\theta-\lambda^2) - \sin^2\theta = 0 \\
	&&\Leftrightarrow&&&
		\lambda^2 - \underbrace{(\sin^2\theta+\cos^2\theta)}_{=1} = 0 \\
	&&\Leftrightarrow&&&
		\lambda^2 = 1 \\
	&&\Leftrightarrow&&&
		\lambda = \begin{cases}
		1  &=\lambda_1 \\
		-1 &=\lambda_2 \\
	\end{cases}
\end{aligned}\end{equation*}
Now that we have our eigenvalues, we can use them to determine the
associated eigenvectors, as, remember, they are linked by:
\[
	(\forall i \in \{1,2\}),\ \sigma_n\ket{\lambda_i} = \lambda_i\ket{\lambda_i}
\]
And so:
\begin{equation*}\begin{aligned}
	\sigma_n\ket{\lambda_1} = \lambda_1\ket{\lambda_1}
	&&\Leftrightarrow&&&
	\begin{pmatrix}
		\cos\theta & \sin\theta \\
		\sin\theta & -\cos\theta \\
	\end{pmatrix}\begin{pmatrix}
		a \\
		c \\
	\end{pmatrix} = \begin{pmatrix}
		a \\
		c \\
	\end{pmatrix} \\
	&&\Leftrightarrow&&&
	\begin{cases}
		a\cos\theta + c\sin\theta & = a \\
		a\sin\theta - c\cos\theta & = c \\
	\end{cases} \\
	&&\Leftrightarrow&&&
	\begin{cases}
		a(\cos\theta - 1) + c\sin\theta & = 0 \\
		a\sin\theta + c(-\cos\theta-1) & = 0 \\
	\end{cases}
\end{aligned}\end{equation*}

Consider the first equation of this system: we're left
with two main choices, depending on whether
$\cos\theta = 1$ or not. If it is, let's take $\theta=0$
for instance, but this would true modulo $\pi$, then
we must have $\sin\theta=0$, and the first equations gives
us nothing of value. The second then simplifies to $c=0$, thus $a=0$. \\

Let's now consider the case where $\cos\theta \neq 1$.
The system can be rewritten as:
\begin{equation*}\begin{aligned}
	\begin{cases}
		a &= \dfrac{-c\sin\theta}{\cos\theta - 1}\\
		a\sin\theta + c(-\cos\theta-1) & = 0 \\
	\end{cases} \\
\end{aligned}\end{equation*}

We can inject the first equation in the second to yield:

\begin{equation*}\begin{aligned}
	\frac{-c\sin\theta}{\cos\theta - 1}\sin\theta + c(-\cos\theta-1) = 0
	&&\Leftrightarrow&&&
	\frac{-c\sin\theta}{\cos\theta - 1}\sin\theta + \frac{\cos\theta-1}{\cos\theta-1}c(-\cos\theta-1) = 0 \\
	&&\Leftrightarrow&&&
	\frac{c\bigl(-\sin^2\theta - (\cos\theta-1)(\cos\theta+1)\bigr)}{\cos\theta-1} = 0 \\
	&&\Rightarrow&&&
		c(-\sin^2\theta - (\cos^2\theta-1)) = 0 \\
	&&\Rightarrow&&&
		c(-\underbrace{(\sin^2\theta+\cos^2\theta)}_{=1}-1) = 0 \\
	&&\Rightarrow&&&
		c = 0 \Rightarrow a = 0
\end{aligned}\end{equation*}

That's a struggle; we don't seem to be able to extract anything
but the trivial solution; maybe there's some trigonometric trick to
find the general
solution\footnote{There definitely is one, see for instance \url{https://www.wolframalpha.com/input?i=diagonalize+\%7B\%7Bcos+x\%2C+sin+x\%7D\%2C\%7Bsin+x\%2C-cos+x\%7D\%7D}}). \\

\hrr

Instead, let's try to use and understand the authors' hint,
which is to look for eigenvectors of the form:
\[
	\begin{pmatrix}
		\cos\alpha \\
		\sin\alpha \\
	\end{pmatrix}
\]

\textit{Why} is this a reasonable choice? Let's start by answering
why we need a single parameter $\alpha$: it corresponds to the single
degree of freedom we have in this case. Let's recall the two equivalent
ways of counting the number of degree of freedom that were given
in subsection $2.5$:

\begin{enumerate}
	\item First, point the apparatus in any direction in the $xz$-plane
	(remember for comparison that in subsection $2.5$, we were allowed
	to take a direction in the $xyz$-space). A single angle is sufficient
	to encode this single direction ($2$ were needed in the $xyz$ space).
	Furthermore, note that this angle would have has its coordinate in
	the $xz$-plane $\cos\alpha$ and $\sin\alpha$, respectively in the
	$x$ and $z$ directions. \\

	Note that we're really capturing \textit{directions}: a point
	in $\mathbb{R}^2$ contains too much information, as we want
	to identify all the points which share the same direction;
	\item The second approach was to say that the general form of the
	spin state in $xyz$-space was given by a (complex) linear combination
	$\alpha_u\ket{u}+\alpha_d\ket{d}$. But, recall the definition
	of $\ket{l}$ and $\ket{r}$, the vectors associated with the $x$-direction:
	\[
		\ket{r} = \frac1{\sqrt2}\ket{u} + \frac1{\sqrt2}\ket{d};\qquad
		\ket{l} = \frac1{\sqrt2}\ket{u} - \frac1{\sqrt2}\ket{d}
	\]
	They didn't involved complex numbers. We started to need, and have
	proven in exercise
	\href{https://github.com/mbivert/ttm/blob/master/qm/L02E03.pdf}{L02E03}
	that this was mandatory once we had
	enough constraints to cover the three spatial directions (i.e.,
	when dealing with $\ket{i}$ and $\ket{o}$, after having already
	established the two other pairs of orthogonal vectors). \\

	That's to say, we don't need complex numbers when we only have
	two directions, so actually, the general form of a spin
	in a plane is a \textit{real} linear combination, which cuts
	down the number of degrees of freedom to $2$. \\

	Normalization adds yet another constraint, which cuts us down
	to a single degree of freedom. But, shouldn't the phase
	ambiguity brings us to $\ldots$ \textit{zero} degree of freedom?
	What are we missing? \\

	Well, the idea of phase ambiguity was that we could multiply
	the vectors by a $\exp(i\theta) = cos\theta+i\sin\theta$, for
	$\theta\in\mathbb{R}$. But we saw that we actually don't need
	complex numbers when we're in a $2D$-plane, which means
	$\sin\theta = 0$, and thus forces $\cos\theta = 1$, so the
	phase ambiguity doesn't impact the number of degrees
	of freedom;

	\item Here's a third argument that we'll re-use in the next
	exercise\footnote{Source: \url{https://physics.stackexchange.com/a/720025}}.
	Consider as a first guess an eigenvector of the form
	\[
		\begin{pmatrix}
			z_1 \\
			z_2 \\
		\end{pmatrix};\qquad (z_1, z_2)\in\mathbb{C}^2
	\]
	We can put both complex numbers in exponential form:
	\[
		\begin{pmatrix}
			r_1\exp(i\phi_1) \\
			r_2\exp(i\phi_2 \\
		\end{pmatrix} = \exp(i\phi_1)\begin{pmatrix}
			r_1 \\
			r_2\exp(i(\phi_2-\phi_1)) \\
		\end{pmatrix};\qquad (r_1, r_2, \phi_1, \phi_2)\in\mathbb{R}^4
	\]
	We can then ignore the general phase factor $\exp(i\phi_1)$, e.g.
	choose $\phi_1$ = 0. Furthermore, we'll want the (eigen)vector to be
	normalized (remember, the eigenvector associated to the eigenvalues
	of of a Hermitian operator make an orthonormal basis), i.e.:
	\[
		|r_1|^2 + |r_2\exp(i\phi_2)|^2 = 1
		\Leftrightarrow |r_1|^2 + |r_2|^2 = 1
	\]
	But we're then losing a degree of freedom, meaning, $r_1$ and $r_2$
	are not independent from each other: we can express them both in term of
	a single parameter, as long as the previous equation is satisfied.
	We can choose, as it'll make computation easier,
	$r_1 = \cos\alpha$, $r_2=\sin\alpha$, with $\alpha\in\mathbb{R}$.
	Which brings us to:
	\[
		\begin{pmatrix}
			\cos\alpha \\
			\exp(i\phi_2)\sin\alpha \\
		\end{pmatrix}
	\]
	If $\phi_2$ varies, then our eigenvector isn't restricted to
	a plane. But, because our eigenvector will be a eigenvector
	of a Hermitian matrix, we know by the real spectral theorem\footnote{
	\href{https://github.com/mbivert/ttm/blob/master/qm/L03E01.pdf}{L03E01.pdf}}
	that it must be a (an orthonormal basis) vector of the $xz$-plane.
	So we can choose $\phi_2 = 0$ to restrict it to a plane.
\end{enumerate}

Note that the form of this vector is naturally normalized ($\cos^2\alpha
+\sin^2\alpha = 1$). Recall that it \textit{must} be normalized because
this column vector actually corresponds to:
\[
	\begin{pmatrix}
		\cos\alpha \\
		\sin\alpha \\
	\end{pmatrix} = \cos\alpha\begin{pmatrix}
		1 \\
		0 \\
	\end{pmatrix} + \sin\alpha\begin{pmatrix}
		0 \\
		1 \\
	\end{pmatrix} = \cos\alpha\ket{u} + \sin\alpha\ket{d}
\]
And the square of the magnitude of $\cos\alpha$ encodes the
probability for the measured value to correspond to $\ket{u}$
while the square of the magnitude of $\sin\alpha$ encodes the
probability of the system to be measured in state $\ket{d}$,
and both states are orthogonal: the total probability must be $1$. \\

Alright, let's get to actually finding the eigenvectors
associated to our eigenvalues. We can use the same trick as
in the previous exercise
\href{https://github.com/mbivert/ttm/blob/master/qm/L03E02.pdf}{L03E02.pdf}:
because of the diagonalization process, we have the following relation:
\[
	\sigma_n = PDP^{-1} \Leftrightarrow \sigma_n P
		= PD(\underbrace{PP^{-1}}_{:=I_2}) = PD = P\begin{pmatrix}
		1 & 0 \\
		0 & -1 \\
	\end{pmatrix}
\]
\[
	\Leftrightarrow \underbrace{\begin{pmatrix}
		\cos\theta & \sin\theta \\
		\sin\theta & -\cos\theta \\
	\end{pmatrix}}_{=\sigma_n}\underbrace{\begin{pmatrix}
		\cos\alpha & \cos\beta \\
		\sin\alpha & \sin\beta \\
	\end{pmatrix}}_{=P} =\begin{pmatrix}
		\cos\alpha & \cos\beta \\
		\sin\alpha & \sin\beta \\
	\end{pmatrix}\begin{pmatrix}
		1 & 0 \\
		0 & -1 \\
	\end{pmatrix} = \begin{pmatrix}
		\cos\alpha & -\cos\beta \\
		\cos\alpha & -\sin\beta \\
	\end{pmatrix}
\]

Where the columns of $P$ are the eigenvectors associated
to the eigenvalues $1$ and $-1$. Both have the same "form", as
previously explained. We could have used the same approach as in
the book (see the previous exercise), but you'll get with the
same (kind?) of system in the end. Let's perform the
matrix multiplication on the left and extract two equations from the
four we can get by identifying the matrix components:
\[
	\begin{pmatrix}
		\cos\theta\cos\alpha + \sin\theta\sin\alpha &
			\cos\theta\cos\beta + \sin\theta\sin\beta \\
		\sin\theta\cos\alpha - \cos\theta\sin\alpha &
			\sin\theta\cos\beta - \cos\theta\sin\beta \\
	\end{pmatrix} = \begin{pmatrix}
		\cos\alpha & -\cos\beta \\
		\cos\alpha & -\sin\beta \\
	\end{pmatrix}
\]
\[
	\Leftrightarrow\begin{cases}
		\cos\theta\cos\alpha + \sin\theta\sin\alpha = \cos\alpha \\
		\cos\theta\cos\beta  + \sin\theta\sin\beta  = -\cos\beta \\
	\end{cases}
\]

\begin{remark} Strictly speaking, we don't really know if this is
equivalent so far, as we're just extracting two equations from
potentially four distinct equations. For correctness' sake, we could
(I won't out of laziness) verify that the solution we find for
those two equations also solve the two other remaining equations.
\end{remark}

The following trigonometric identities\footnote{Look around
for the proofs if needed; formulas can be found on
\href{https://en.wikipedia.org/wiki/List\_of\_trigonometric\_identities\#Angle\_sum\_and\_difference\_identities}{Wikipedia}}:
\[
	\cos\theta\cos\alpha = \frac12(\cos(\theta-\alpha) + \cos(\theta+\alpha));\qquad
	\sin\theta\sin\alpha = \frac12(\cos(\theta-\alpha) - \cos(\theta+\alpha))
\]
\[
	\cos(\alpha-\pi) = -\cos\alpha
\]
Allows us to rewrite the previous system as
\[
	\Leftrightarrow\begin{cases}
		\frac12\Bigl(
			\bigl(\cos(\theta-\alpha)+\cos(\theta+\alpha)\bigr) +
			\bigl(\cos(\theta-\alpha)-\cos(\theta+\alpha)\bigr)
		\Bigr) = \cos\alpha \\
				\frac12\Bigl(
			\bigl(\cos(\theta-\beta)+\cos(\theta+\beta)\bigr) +
			\bigl(\cos(\theta-\beta)-\cos(\theta+\beta)\bigr)
		\Bigr) = \cos(\beta-\pi) \\
	\end{cases}
\]
\[
	\Leftrightarrow\begin{cases}
		\cos(\theta-\alpha) = \cos\alpha \\
		\cos(\theta-\beta)  = \cos(\beta-\pi) \\
	\end{cases}
\]
And with the following identities:
\[
	\cos(\alpha+\frac\pi2) = -\sin\alpha;\qquad
	\sin(\alpha+\frac\pi2) = \cos\alpha
\]
We reach:
\[
	\Rightarrow\begin{cases}
		\theta-\alpha = \alpha \\
		\theta-\beta = \beta-\pi \\
	\end{cases}\Rightarrow\begin{cases}
		\alpha = \frac\theta2 \\
		\beta  = \frac12(\theta+\pi) \\
	\end{cases}\Rightarrow\boxed{\begin{cases}
		\ket{+1} = \begin{pmatrix}
			\cos\alpha \\
			\sin\alpha \\
		\end{pmatrix} = \begin{pmatrix}
			\cos(\theta/2) \\
			\sin(\theta/2) \\
		\end{pmatrix} \\
		\ket{-1} = \begin{pmatrix}
			\cos\beta \\
			\sin\beta \\
		\end{pmatrix} = \begin{pmatrix}
			\cos(\theta/2 + \pi/2) \\
			\sin(\theta/2 + \pi/2) \\
		\end{pmatrix} = \begin{pmatrix}
			-\sin(\theta/2) \\
			\cos(\theta/2) \\
		\end{pmatrix}\\
	\end{cases}}
\]

\end{document}
