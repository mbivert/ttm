\documentclass[solutions.tex]{subfiles}

\xtitle

\begin{document}
\maketitle
\begin{exercise} Do the same for the other two entangled triplet
states,
\[
	\ket{T_2} = \frac1{\sqrt2}\left(\ket{uu}+\ket{dd}\right)
\]
\[
	\ket{T_3} = \frac1{\sqrt2}\left(\ket{uu}-\ket{dd}\right)
\]
\end{exercise}
As for
\href{https://github.com/mbivert/ttm/blob/master/qm/L06E07.pdf}{previous exercise},
this is just about crunching numbers. We won't be using the Pauli matrices
explicitly here; instead, we'll use the multiplication table from
\href{https://github.com/mbivert/ttm/blob/master/qm/L06E04.pdf}{L06E04}

\begin{equation*}\begin{aligned}
	\sigma_z\ket{uu} &=&\ket{uu};  &&& \tau_z\ket{uu} &=&& \ket{uu} \\
	\sigma_z\ket{ud} &=&\ket{ud};  &&& \tau_z\ket{ud} &=&& -\ket{ud} \\
	\sigma_z\ket{du} &=&-\ket{du}; &&& \tau_z\ket{du} &=&& \ket{du} \\
	\sigma_z\ket{dd} &=&-\ket{dd}; &&& \tau_z\ket{dd} &=&& -\ket{dd} \\
	 \cline{3-6}
	\sigma_x\ket{uu} &=&\ket{du}; &&& \tau_x\ket{uu} &=&& \ket{ud} \\
	\sigma_x\ket{ud} &=&\ket{dd}; &&& \tau_x\ket{ud} &=&& \ket{uu} \\
	\sigma_x\ket{du} &=&\ket{uu}; &&& \tau_x\ket{du} &=&& \ket{dd} \\
	\sigma_x\ket{dd} &=&\ket{ud}; &&& \tau_x\ket{dd} &=&& \ket{du} \\
	 \cline{3-6}
	\sigma_y\ket{uu} &=&i\ket{du};  &&& \tau_y\ket{uu} &=&& i\ket{ud} \\
	\sigma_y\ket{ud} &=&i\ket{dd};  &&& \tau_y\ket{ud} &=&& -i\ket{uu} \\
	\sigma_y\ket{du} &=&-i\ket{uu}; &&& \tau_y\ket{du} &=&& i\ket{dd} \\
	\sigma_y\ket{dd} &=&-i\ket{ud}; &&& \tau_y\ket{dd} &=&& -i\ket{du} \\
\end{aligned}\end{equation*}

\hr

As the computations are fairly similar, and to save space, I'll be computing
the expectation values for $T_2$ and $T_3$ in parallel, distinguishing them by
a subscript number. \\

Let's start with $\avg{\sigma_z\tau_z}$:
\begin{equation*}\begin{aligned}
	\avg{\sigma_z\tau_z}_2 &:=&& \bra{T_2}\sigma_z\tau_z\ket{T_2}
	& \avg{\sigma_z\tau_z}_3 &:=&& \bra{T_3}\sigma_z\tau_z\ket{T_3} \\
	%
	~ &=&& \frac1{\sqrt2}\bra{T_2}\sigma_z\tau_z\left(\ket{uu}+\ket{dd}\right)
	& ~ &=&& \frac1{\sqrt2}\bra{T_3}\sigma_z\tau_z\left(\ket{uu}-\ket{dd}\right) \\
	%
	~ &=&& \frac1{\sqrt2}\bra{T_2}\sigma_z\left(\ket{uu}-\ket{dd}\right)
	& ~ &=&& \frac1{\sqrt2}\bra{T_3}\sigma_z\left(\ket{uu}+\ket{dd}\right) \\
	%
	~ &=&& \frac1{\sqrt2}\bra{T_2}\left(\ket{uu}+\ket{dd}\right)
	& ~ &=&& \frac1{\sqrt2}\bra{T_3}\left(\ket{uu}-\ket{dd}\right) \\
	%
	~ &=&& \frac12(\bra{uu}+\bra{dd})(\ket{uu}+\ket{dd})
	& ~ &=&& \frac12(\bra{uu}-\bra{dd})(\ket{uu}-\ket{dd}) \\
	%
	~ &=&& \frac12\left(
		\underbrace{\braket{uu}{uu}}_{1}
		+\underbrace{\braket{uu}{dd}}_{0}
		+\underbrace{\braket{dd}{uu}}_{0}
		+\underbrace{\braket{dd}{dd}}_{1}
	\right)
	& ~ &=&& \frac12\left(
		\underbrace{\braket{uu}{uu}}_{1}
		-\underbrace{\braket{uu}{dd}}_{0}
		-\underbrace{\braket{dd}{uu}}_{0}
		+\underbrace{\braket{dd}{dd}}_{1}
	\right) \\
	%
	~ &=&& \boxed{+1}
	& ~ &=&& \boxed{+1}
\end{aligned}\end{equation*}

Moving on to $\avg{\sigma_x\tau_x}$:
\begin{equation*}\begin{aligned}
	\avg{\sigma_x\tau_x}_2 &:=&& \bra{T_2}\sigma_x\tau_x\ket{T_2}
	& \avg{\sigma_x\tau_x}_3 &:=&& \bra{T_3}\sigma_x\tau_x\ket{T_3} \\
	%
	~ &=&& \frac1{\sqrt2}\bra{T_2}\sigma_x\tau_x\left(\ket{uu}+\ket{dd}\right)
	& ~ &=&& \frac1{\sqrt2}\bra{T_3}\sigma_x\tau_x\left(\ket{uu}-\ket{dd}\right) \\
	%
	~ &=&& \frac1{\sqrt2}\bra{T_2}\sigma_x\left(\ket{ud}+\ket{du}\right)
	& ~ &=&& \frac1{\sqrt2}\bra{T_3}\sigma_x\left(\ket{ud}-\ket{du}\right) \\
	%
	~ &=&& \frac1{\sqrt2}\bra{T_2}\left(\ket{dd}+\ket{uu}\right)
	& ~ &=&& \frac1{\sqrt2}\bra{T_3}\left(\ket{dd}-\ket{uu}\right) \\
	%
	~ &=&& \frac12(\bra{uu}+\bra{dd})(\ket{dd}+\ket{uu})
	& ~ &=&& \frac12(\bra{uu}-\bra{dd})(\ket{dd}-\ket{uu}) \\
	%
	~ &=&& \frac12\left(
		\underbrace{\braket{uu}{dd}}_{0}
		+\underbrace{\braket{uu}{uu}}_{1}
		+\underbrace{\braket{dd}{dd}}_{1}
		+\underbrace{\braket{dd}{uu}}_{0}
	\right)
	& ~ &=&& \frac12\left(
		\underbrace{\braket{uu}{dd}}_{0}
		-\underbrace{\braket{uu}{uu}}_{1}
		-\underbrace{\braket{dd}{dd}}_{1}
		+\underbrace{\braket{dd}{uu}}_{0}
	\right) \\
	%
	~ &=&& \boxed{+1}
	& ~ &=&& \boxed{-1}
\end{aligned}\end{equation*}

Finally for $\avg{\sigma_y\tau_y}$:
\begin{equation*}\begin{aligned}
	\avg{\sigma_y\tau_y}_2 &:=&& \bra{T_2}\sigma_y\tau_y\ket{T_2}
	& \avg{\sigma_y\tau_y}_3 &:=&& \bra{T_3}\sigma_y\tau_y\ket{T_3} \\
	%
	~ &=&& \frac1{\sqrt2}\bra{T_2}\sigma_y\tau_y\left(\ket{uu}+\ket{dd}\right)
	& ~ &=&& \frac1{\sqrt2}\bra{T_3}\sigma_y\tau_y\left(\ket{uu}-\ket{dd}\right) \\
	%
	~ &=&& \frac1{\sqrt2}\bra{T_2}\sigma_y\left(i\ket{ud}-i\ket{du}\right)
	& ~ &=&& \frac1{\sqrt2}\bra{T_3}\sigma_y\left(i\ket{ud}+i\ket{du}\right) \\
	%
	~ &=&& \frac{i}{\sqrt2}\bra{T_2}\left(i\ket{dd}+i\ket{uu}\right)
	& ~ &=&& \frac{i}{\sqrt2}\bra{T_3}\left(i\ket{dd}-i\ket{uu}\right) \\
	%
	~ &=&& -\frac12(\bra{uu}+\bra{dd})(\ket{dd}+\ket{uu})
	& ~ &=&& -\frac12(\bra{uu}-\bra{dd})(\ket{dd}-\ket{uu}) \\
	%
	~ &=&& \frac{-1}2\left(
		\underbrace{\braket{uu}{dd}}_{0}
		+\underbrace{\braket{uu}{uu}}_{1}
		+\underbrace{\braket{dd}{dd}}_{1}
		+\underbrace{\braket{dd}{uu}}_{0}
	\right)
	& ~ &=&& \frac{-1}2\left(
		\underbrace{\braket{uu}{dd}}_{0}
		-\underbrace{\braket{uu}{uu}}_{1}
		-\underbrace{\braket{dd}{dd}}_{1}
		+\underbrace{\braket{dd}{uu}}_{0}
	\right) \\
	%
	~ &=&& \boxed{-1}
	& ~ &=&& \boxed{+1}
\end{aligned}\end{equation*}

We can conclude, from those expectation values alone, that whenever:
\begin{itemize}
	\item The expectation value is $-1$, Bob and Alice measure
	a spin pointing in different directions;
	\item The expectation value is $+1$, Bob and Alice measure
	a spin pointing in the same direction.
\end{itemize}

I just want to spend a few more lines to make something clear. Recall
the definition of $\ket{\text{sing}}$:
\[
	\ket{\text{sing}} = \frac1{\sqrt2}(\ket{ud}-\ket{du})
\]
The argument of the authors was that, the reason for $\avg{\tau_z\sigma_z}$
to be $-1$ was that $\ket{\text{sing}}$ is built from two spins,
one of which is always up while the other is down, and we're measuring
both spin alongside the axis on which they are either up or down. \\

However, in the case of e.g. $\avg{\tau_x\sigma_x}$, the answer was
not as obviously, because we're in this case measuring the spins alongside
the $x$-axis, and it's not immediate from the expression of
$\ket{\text{sing}}$ what kind of balance we have alongside the $x$-axis. \\

Let's do a little experiment. Recall the definition of the "basis
vectors" for the $x$-axis, left and right:
\[
	\ket{r} = \frac1{\sqrt2}(\ket{u}+\ket{d});\qquad
	\ket{l} = \frac1{\sqrt2}(\ket{u}-\ket{d})
\]
We want to express, say, $T_3$ in terms of $\ket{l}$ and $\ket{r}$,
to see if indeed, when expressed as such, $T_3$ is created from
two spins such that when one is left, the other is right, which
would be concordant with the idea that $\avg{\sigma_x\tau_x}_3 = -1$.
Let's start by rewriting $\ket{u}$ and $\ket{d}$ in terms of $\ket{r}$
and $\ket{l}$:
\[
	\begin{cases}
		\ket{r} = \frac1{\sqrt2}(\ket{u}+\ket{d}) \\
		\ket{l} = \frac1{\sqrt2}(\ket{u}-\ket{d})
	\end{cases}\Leftrightarrow\begin{cases}
		\ket{u} = \sqrt2\ket{r}-\ket{d} \\
		\ket{d} = -\sqrt2\ket{l}+\ket{u} \\
	\end{cases}\Leftrightarrow\begin{cases}
		\ket{u} = \frac{\sqrt2}2(\ket{r}+\ket{l}) \\
		\ket{d} = \frac{\sqrt2}2(\ket{r}-\ket{l}) \\
	\end{cases}
\]

Let's now rewrite $T_3$ in the $\ket{r}$, $\ket{l}$ basis:
\begin{equation*}\begin{aligned}
	\ket{T_3} &=&& \frac1{\sqrt2}\left(\ket{uu}-\ket{dd}\right) \\
	~ &=&& \frac1{\sqrt2}\left( \keit{u}\otimes\ket{u}-\keit{d}\otimes\ket{d}\right) \\
	~ &=&& \frac1{\sqrt2}\left(
		\frac24(\keit{r}+\keit{l})(\ket{r}+\ket{l})
		-\frac24(\keit{r}-\keit{l})(\ket{r}-\ket{l})
	\right) \\
	~ &=&& \frac1{2\sqrt2}\left(
		\ket{rr}+\ket{rl}+\ket{lr}+\ket{ll}
		-(\ket{rr}-\ket{rl}-\ket{lr}+\ket{ll})
	\right) \\
	~ &=&&\frac1{\sqrt2}(\ket{rl}+\ket{lr})
\end{aligned}\end{equation*}

And indeed, as expected, $T_3$ is built from two spins such that when
one is left, the other is right.
\end{document}