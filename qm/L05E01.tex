\documentclass[solutions.tex]{subfiles}

\xtitle

\begin{document}
\maketitle
\begin{exercise} Verify this claim.
\end{exercise}
The claim being that any $2\times2$ Hermitian matrix
can be represented as a linear combination of:
\[
	I = \begin{pmatrix}
		1 & 0 \\
		0 & 1 \\
	\end{pmatrix};\quad
	\sigma_x = \begin{pmatrix}
		0 & 1 \\
		1 & 0 \\
	\end{pmatrix};\quad
	\sigma_y = \begin{pmatrix}
		0 & -i \\
		i & 0  \\
	\end{pmatrix};\quad
	\sigma_z = \begin{pmatrix}
		1 & 0 \\
		0 & -1 \\
	\end{pmatrix}
\]

The general form of a $2\times2$ Hermitian matrix is:
\[
	(\forall (r, r', w)\in\mathbb{R}^2\times\mathbb{C}),\quad
	\begin{pmatrix}
		r & w \\
		w^* & r' \\
	\end{pmatrix}
\]
Recall indeed that because for a Hermitian matrix $L$ we have
$L = L^\dagger := (L^*)^T$, hence the diagonal elements must be real. \\

Compare then with the general form for a linear combination of the four
matrices above:
\[
	(\forall (a, b, c, d)\in\mathbb{R}^4),\quad
	a\sigma_x + b\sigma_y + c\sigma_z + dI = \begin{pmatrix}
		c+d  & a-ib \\
		a+ib & c-d  \\
	\end{pmatrix}
\]

Clearly we can identify $w\in\mathbb{C}$ with $a-ib$: this is a general
form for a complex number, and this naturally identifies $w^*$ with
$a+ib$, as expected. \\

Regarding the remaining parameters, we have on one side two
real parameters, and on the other side, two non-equivalent equations
involving two parameters, meaning, two degrees of freedom on both sides.
So there's room to identify $r$ with $c+d$ and $r'$ with $c-d$. More
precisely, given two arbitrary $(r, r')\in\mathbb{R}^2$, we can always
find $(c, d)\in\mathbb{R}^2$ such that $r = c+d$ and $r' = c-d$:
\[
	\begin{cases}
		r = c+d \\
		r' = c-d \\
	\end{cases}\Leftrightarrow
	\begin{cases}
		c = r-d \\
		d = c-r' \\
	\end{cases}\Leftrightarrow
	\begin{cases}
		c = r-(c-r') \\
		d = (r-d)-r' \\
	\end{cases}\Leftrightarrow
	\begin{cases}
		c = \frac{r+r'}2 \\
		d = \frac{r-r'}2 \\
	\end{cases}
\]

\begin{remark} Note that (real) linear combinations of those 4 matrices
are isomorphic to  $\mathbb{Q}$\footnote{
\url{https://en.wikipedia.org/wiki/Quaternion}}.
\end{remark}

\end{document}
