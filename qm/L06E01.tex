\documentclass[solutions.tex]{subfiles}

\xtitle

\begin{document}
\maketitle
\begin{exercise} Prove that if $P(a, b)$ factorizes, then
the correlation between $a$ and $b$ is zero.
\end{exercise}
Let's assume that $P(a, b)$ factorizes, meaning, let's assume
than there are two functions $P_A$ and $P_B$ such that:
\[
	P(a, b) = P_A(a)P_B(b)
\]
Recall that the authors have defined the (statistical) correlation
the quantity\footnote{Precise mathematical formulations are
more involved, see for instance
\url{https://en.wikipedia.org/wiki/Correlation}}:
\[
	\avg{\sigma_A\sigma_B}-\avg{\sigma_A}\avg{\sigma_B}
\]
Where $\avg{\sigma_C}$ is the average value of $C$'s observations,
also known as the expected
value\footnote{\url{https://en.wikipedia.org/wiki/Expected\_value}},
and was defined earlier as:
\[
	\avg{\sigma_C} = \sum_c cP(c)
\]
How should we understand $\avg{\sigma_A\sigma_B}$? We're trying to find
a way to express it as we just did for $\avg{\sigma_C}$. It's defined
as the average of the product of $\sigma_A$ and $\sigma_B$, meaning,
the sum of all possible products of $a$ and $b$, weighted by some probability
distribution, but which one? Well, we don't really know its form specifically,
but if for $\avg{\sigma_C}$ it was a function of $c$, then we can guess it
must now be a function of $a$ and $b$: this is the $P(a, b)$ from the exercise
statement:
\[
	\avg{\sigma_A\sigma_B} = \sum_a\left(\sum_b ab P(a, b)\right)
\]

From there, it's just a matter of developing the computation,
using our assumption that $P(a, b)$ factorizes:
\begin{equation*}\begin{aligned}
	\avg{\sigma_A\sigma_B} &=&& \sum_a\left(\sum_b ab P(a, b)\right) \\
	~ &=&& \sum_a\sum_b \left(ab P_A(a)P_B(b)\right) \\
	~ &=&& \sum_a\sum_b \left((a P_A(a))(bP_B(b))\right) \\
	~ &=&& \left(\sum_a aP_A(a)\right)\left(\sum_bbP_B(b)\right) \\
	~ &=&& \avg{\sigma_A}\avg{\sigma_B}
\end{aligned}\end{equation*}
\[
	\Leftrightarrow
		\boxed{\avg{\sigma_A\sigma_B} - \avg{\sigma_A}\avg{\sigma_B} = 0}\qed
\]

\begin{remark} If you're uncertain about the $\sum$ manipulations, you
may want to rewrite them as explicit sum over a small number of terms
to convince you of their correctness.
\end{remark}

\end{document}
