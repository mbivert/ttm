\documentclass[solutions.tex]{subfiles}

\xtitle

\begin{document}
\maketitle
\begin{exercise} Verify that the state-vector in $7.30$ represents a completely
untangled state.
\end{exercise}
Let's recall the state-vector from $7.30$, and let's call it $\ket{\Psi}$.
\[
	\ket{\Psi} = \alpha_u\ket{u,b} + \alpha_d\ket{d,b}
\]
As I've found this confusing, let me start by recalling
a bit of vocabulary\footnote{See for instance: \url{https://www.researchgate.net/post/What-is-difference-between-mixed-state-and-mixed-entangled-state}}.
A quantum state can be either \textbf{pure} or \textbf{mixed}: either its
a single state, or a \textit{convex combination}\footnote{A fancy term
you may find here and there: a linear combination of elements, where the
scalars factors sums to $1$; see \url{https://en.wikipedia.org/wiki/Convex_combination}}
of pure states. This is true for a "regular" state space, as for a state space built
via a tensor products of two (or finitely many, by induction) other state
spaces. \\

Now there's a second qualification, that is only applicable for states
which are taken from a state space made by glueing two (again, or finitely many)
other state spaces: \textbf{entangled} states, and \textbf{disentangled} states. \\

\textit{Mixed} and \textit{entangled} are definitely not synonymous: you can have
a non-mixed (i.e. pure) entangled state for example. \\

\begin{example} The state vector from $7.30$ is a pure state: this is \textit{not}
a convex combination of states. But this tells us absolutely nothing regarding
whether it's an entangled state. We know however that it makes sense to talk about
it being entangled or not, as we're dealing with a combined system involving
$(i)$ an apparatus and $(ii)$ a spin to be measured by said apparatus. \\

We could test this purity by computing the density matrix $\rho$, and checking
whether $\rho^2 = \rho$ or $\Tr(\rho) = 1$.
\end{example}

Let's clarify the vocabulary one step further: a \textbf{completely untangled state}
\textit{is} a \textbf{product state}: that's a state where measurements on one subsystem
affect in no ways the other subsystem(s). \\

From there, we have a few different ways of proceeding.

\hr

The simplest approach is to remember that a state is a product state when it can
be expressed via two components (well, or more, but we're in the case where
there are two subsystems here: the apparatus, and the spin to be measured with the
apparatus), one for each subsystem. Recall that $\ket{a,\alpha}$ really is a shortcut
for $\ket{a}\otimes\keit{\alpha}$. This means, the prepared state really is:
\[
	\alpha_u\ket{u}\otimes\keit{b} + \alpha_d\ket{d}\otimes\keit{b}
\]
But the tensor product distributes\footnote{As is common in most Physics-centered
introduction to Quantum Mechanics, the tensor product introduction is a bit
hand-wavy. For a more rigorous development, see for instance
\href{https://www.youtube.com/watch?v=OZ1WCyJmjgo&list=PLPH7f_7ZlzxQVx5jRjbfRGEzWY_upS5K6&index=14&pp=iAQB}{this video} by F. Schuller. Some subtleties such as the
fact that the equivalence classes respect addition and scalar multiplication have
been left as homework; there's a set of
\href{https://drive.google.com/file/d/1I7rIH7Rtm0cCKVuLNeWfFMdKurX123x5/view}{notes}
which contains the "missing" proofs.}, hence this simplifies as:
\[
	\left(\alpha_u\ket{u}+\alpha_d\ket{d}\right)\otimes\keit{b}
\]

As the combined state is normalized, we must have
$\sqrt{\alpha_u^2+\alpha_d^2}=1$, which implies that the sub-state
corresponding to the spin is also normalized. Trivially, the sub-state
corresponding to the apparatus is also normalized. Hence, we've expressed
our combined state as a tensor product of two normalized state,
one for each subsystems: $\fbox{\text{this is a product state}}$. \\

\hr

A slightly more involved (calculus-wise) variant of this approach would be to
rely on the general form of the product state\footnote{p164, section $6.5$ -
\textit{Product states}} and to evaluate whether our state vector can be expressed
in such a way. The general form can be computed, again using the distributive
nature of the tensor product:

\begin{equation*}\begin{aligned}
	\ket{\text{product state}} &=&&
		\biggl\{\alpha_u\ket{u}+\alpha_d\ket{d}\biggr\}
		\otimes
		\biggl\{\beta_b\keit{b}+\beta_+\keit{+1}+\beta_-\keit{-1}\biggr\} \\
	~ &=&&
		\alpha_u\ket{u}\otimes\biggl\{
			\beta_b\keit{b}+\beta_+\keit{+1}+\beta_-\keit{-1}\biggr\}
		+\alpha_d\ket{d}\biggl\{
			\beta_b\keit{b}+\beta_+\keit{+1}+\beta_-\keit{-1}\biggr\} \\
	~ &=&&
		\alpha_u\beta_d\ket{u,b}+
		\alpha_u\beta_+\ket{u,+1}+
		\alpha_u\beta_-\ket{u,-1}+
		\alpha_d\beta_d\ket{d,b}+
		\alpha_d\beta_+\ket{d,+1}+
		\alpha_d\beta_-\ket{d,-1}
\end{aligned}\end{equation*}

By setting:
\[
	\beta_d = 1; \beta_+ = \beta_- = 0
\]

We found back the state vector from $7.30$, retrospectively justifying the
notation for $\alpha_u$ and $\alpha_d$. Because the subsystem states, must
be normalized, the resulting combined state is also normalized.

\hr

However, and perhaps this is more in line with the author's intent, we've
just saw\footnote{p$212$ and onward, section $7.7$ \textit{Tests for Entanglement}}
two tests to check whether the state corresponding to a given a wave-function
for a composite system is entangled or not. \\

\hrr

For the first criteria, we'd need to take any two arbitrary observables
from each subsystem, say observable $\bm{A}$ and $\bm{B}$, and prove that
their correlation $C(\bm{A}, \bm{B})$ is zero. But essentially, the proof
will end up relying on the tensor product distributivity, rely on the
density matrix (see just after), or essentially mimick the proofs of
\href{https://github.com/mbivert/ttm/blob/master/qm/L07E09.pdf}{L07E09}. \\

I don't think there's added value to develop it further here. \\

\hrr

The second technique is slightly more original: the idea is that, for any product
state, the density matrix has exactly one non-zero eigenvalue, and that
eigenvalue is exactly $1$. \\

Recall that in
\href{https://github.com/mbivert/ttm/blob/master/qm/L07E04.pdf}{L07E04}
we've already determined Alice's density matrix:
\[
	\rho = \begin{pmatrix}
		\alpha_u^*\alpha_u & \alpha_u^*\alpha_d \\
		\alpha_d^*\alpha_u & \alpha_d^*\alpha_d \\
	\end{pmatrix}
\]
Let's diagonalize it: as usual, we have the eigenvector/eigenvalue
relationship:
\[
	\rho\ket{\lambda} = \lambda\ket{\lambda}
	\Leftrightarrow
	(\rho-I_2\lambda)\ket{\lambda} = 0
\]
Which implies that $\rho-I_2\lambda$ isn't invertible\footnote{Multiply
both side of the equation by the inverse, and use the fact that
an eigenvector cannot be the zero vector}, which translates
to its determinant being equal to zero:
\begin{equation*}\begin{aligned}
	\begin{vmatrix}
		\alpha_u^*\alpha_u-\lambda & \alpha_u^*\alpha_d \\
		\alpha_d^*\alpha_u & \alpha_d^*\alpha_d-\lambda \\
	\end{vmatrix} &=&& \biggl(
		(\alpha_u^*\alpha_u-\lambda)(\alpha_d^*\alpha_d-\lambda)
	\biggr)
	-
	\biggl(
		\alpha_u^*\alpha_d\alpha_d^*\alpha_u
	\biggr) \\
	~ &=&& \biggl(
		\alpha_u^*\alpha_u\alpha_d^*\alpha_d -\lambda(
			\underbrace{\alpha_u^*\alpha_u+\alpha_d^*\alpha_d}_{%
			=\braket{\Psi}{\Psi}=1}
		) +\lambda^2
	\biggr)-\alpha_u^*\alpha_d\alpha_d^*\alpha_u \\
	~ &=&& \lambda(1-\lambda)
\end{aligned}\end{equation*}
Clearly, we have one non-zero eigenvalue which is exactly one: the criteria
indeed applies, and the state must be non-entangled.
\end{document}
