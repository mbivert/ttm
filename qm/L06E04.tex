\documentclass[solutions.tex]{subfiles}

\xtitle

\begin{document}
\maketitle
\begin{exercise} Use the matrix forms of $\sigma_z$, $\sigma_x$,
and $\sigma_y$ and the column vectors for $\keit{u}$ and $\keit{d}$
to verify Eqs. $6.6$. Then, use Eqs. $6.6$ and $6.7$ to write the
equations that were left out of Eqs. $6.8$. Use the appendix to
check your answers.
\end{exercise}
As usual, let's recall our Pauli matrices:

\[
	\sigma_x = \begin{pmatrix}
		0 & 1 \\
		1 & 0 \\
	\end{pmatrix};\qquad \sigma_y = \begin{pmatrix}
		0 & -i \\
		i & 0 \\
	\end{pmatrix};\qquad \sigma_z = \begin{pmatrix}
		1 & 0 \\
		0 & -1 \\
	\end{pmatrix}
\]
The base vectors $\keit{u}$ and $\keit{d}$ are the canonical
basis vectors for $\mathbb{R}^2$:
\[
	\keit{u} = \begin{pmatrix} 1 \\ 0 \\ \end{pmatrix};\qquad
	\keit{d} = \begin{pmatrix} 0 \\ 1 \\ \end{pmatrix}
\]
We're trying to understand how for instance an operator $\sigma_x$
define on Alice's state spaces can be extended to work on a state
vector, taken from a combined state space involving Alice's. \\

The core idea is that the operator will only act on the "component"
of the vector that is related to Alice's state space, while leaving
the components involving other state spaces untouched. \\

Eqs. $6.6$ (first column below) simply encode how the spin operators
act on the basis vectors, in Alice's state space; Eqs. $6.7$
(second column below) are identical, but for Bob's state space:
\begin{equation*}\begin{aligned}
	\sigma_z\keit{u} &=&\keit{u}; &&& \tau_z\ket{u} &=&& \ket{u} \\
	\sigma_z\keit{d} &=&-\keit{d}; &&& \tau_z\ket{d} &=&& -\ket{d} \\
	\sigma_x\keit{u} &=&\keit{d}; &&& \tau_x\ket{u} &=&& \ket{d} \\
	\sigma_x\keit{d} &=&\keit{u}; &&& \tau_x\ket{d} &=&& \ket{u} \\
	\sigma_y\keit{u} &=&i\keit{d}; &&& \tau_y\ket{u} &=&& i\ket{d} \\
	\sigma_y\keit{d} &=&-i\keit{u}; &&& \tau_y\ket{d} &=&& -i\ket{u} \\
\end{aligned}\end{equation*}

Now verifying that the matrix products indeed evaluates as such is
child's play (matrix $\times$ vector products), there's no use
of being more explicit here. \\

For similar reasons, I'll just write a completed $6.8$ here, but won't
develop the computations: one just have to follow the aforementioned
rule: act with the operator on the correct component, extract the eventual
scalar factor, and generally update the corresponding vector component.
This yields, in agreement with the appendix:
\begin{equation*}\begin{aligned}
	\sigma_z\ket{uu} &=&\ket{uu};  &&& \tau_z\ket{uu} &=&& \ket{uu} \\
	\sigma_z\ket{ud} &=&\ket{ud};  &&& \tau_z\ket{ud} &=&& -\ket{ud} \\
	\sigma_z\ket{du} &=&-\ket{du}; &&& \tau_z\ket{du} &=&& \ket{du} \\
	\sigma_z\ket{dd} &=&-\ket{dd}; &&& \tau_z\ket{dd} &=&& -\ket{dd} \\
	 \cline{3-6}
	\sigma_x\ket{uu} &=&\ket{du}; &&& \tau_x\ket{uu} &=&& \ket{ud} \\
	\sigma_x\ket{ud} &=&\ket{dd}; &&& \tau_x\ket{ud} &=&& \ket{uu} \\
	\sigma_x\ket{du} &=&\ket{uu}; &&& \tau_x\ket{du} &=&& \ket{dd} \\
	\sigma_x\ket{dd} &=&\ket{ud}; &&& \tau_x\ket{dd} &=&& \ket{du} \\
	 \cline{3-6}
	\sigma_y\ket{uu} &=&i\ket{du};  &&& \tau_y\ket{uu} &=&& i\ket{ud} \\
	\sigma_y\ket{ud} &=&i\ket{dd};  &&& \tau_y\ket{ud} &=&& -i\ket{uu} \\
	\sigma_y\ket{du} &=&-i\ket{uu}; &&& \tau_y\ket{du} &=&& i\ket{dd} \\
	\sigma_y\ket{dd} &=&-i\ket{ud}; &&& \tau_y\ket{dd} &=&& -i\ket{du} \\
\end{aligned}\end{equation*}
\end{document}
