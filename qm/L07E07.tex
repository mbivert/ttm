\documentclass[solutions.tex]{subfiles}

\xtitle

\begin{document}
\maketitle
\begin{exercise} Use Eq. $7.24$ to calculate $\rho^2$. How does this
result confirm that $\rho$ represents an entangled state? We'll soon discover
that there are other ways to check for entanglement.
\end{exercise}
Here's Eq. $7.24$:
\[
	\rho = \begin{pmatrix}
		1/2 & 0   \\
		0   & 1/2 \\
	\end{pmatrix}
\]
From there it's trivial to see that:
\[
	\rho^2 = \begin{pmatrix}
		1/2 & 0   \\
		0   & 1/2 \\
	\end{pmatrix}^2 = \begin{pmatrix}
		1/4 & 0   \\
		0   & 1/4 \\
	\end{pmatrix}
\]
The authors demonstrated earlier a criteria to determine whether a density
matrix corresponds to an entangled state or not, at the end of section
$7.5$: for a pure state, and a density matrix $\rho$, we \textit{must} have:
\[
	\rho^2 = \rho \text{ and } \Tr(\rho)^2 = 1
\]
While for a mixed or entangled state, we \textit{must} have:
\[
	\rho^2 \neq \rho \text{ and } \Tr(\rho)^2 < 1
\]

Hence, $\boxed{\text{$\rho$ represents an entangled state}}$.

\end{document}
