\documentclass[solutions.tex]{subfiles}

\xtitle

\begin{document}
\maketitle
\begin{exercise} Go back to the definition of Poisson brackets
in \textit{Volume I} and check that the identification in Eq. $4.21$
is dimensionally consistent. Show that without the factor $\hbar$, it
would not be.
\end{exercise}
Let's recall first Eq. $4.21$, where $[.,.]$ is the commutator
and $\{.,.\}$ the Poisson brackets:
\[
	[F, G] \Longleftrightarrow i\hbar\{F,G\}
\]
The Poisson brackets are defined in \textit{Volume I},
Eq. $(9)$ at the end of Lecture $9$ (The Phase Space
Fluid and the Gibbs-Liouville Theorem), as:
\[
	\{F,G\} := \sum_i\left(
		\frac{\partial F}{\partial q_i}
		\frac{\partial G}{\partial p_i}
		-
		\frac{\partial F}{\partial p_i}
		\frac{\partial G}{\partial q_i}
	\right)
\]
Where the $p_i$ are the generalized momentum, and $q_i$ are
the generalized coordinates. Recall that a momentum is typically
defined as a mass in motion, while the coordinates are simply distances
to an origin:
\[
	[p_i] = \text{kg}.\text{m}.\text{s}^{-1};\qquad
	[q_i] = \text{m}
\]
For clarity, let's rewrite one of those partial derivative
in terms of a limit:
\[
	\frac{\partial F}{\partial q_i} = \lim_{\epsilon\rightarrow 0}
	\frac{F(q_i+\epsilon)-F(q_i)}{\epsilon}
\]

First $\epsilon$ must be of the same dimension than $q_i$ is this
case, for otherwise $q_i+\epsilon$ is ill-defined; more generally
it'll have the same dimension that the dimension of the differentiation
variable. \\

Second, observe that, again because otherwise we'd be adding
carrots and potatoes:
\[
	\left[
		\sum_i\left(
				\frac{\partial F}{\partial q_i}
				\frac{\partial G}{\partial p_i}
				-
				\frac{\partial F}{\partial p_i}
				\frac{\partial G}{\partial q_i}
			\right)
	\right] = \left[
		\frac{\partial F}{\partial q_i}
		\frac{\partial G}{\partial p_i}
			-
		\frac{\partial F}{\partial p_i}
		\frac{\partial G}{\partial q_i}
	\right],\quad\text{for any arbitrary $i$ that is}
\]
But then,
\[
	[i\hbar\{F,G\}] = \left[\hbar\left(
		\frac{\partial F}{\partial q_i}
		\frac{\partial G}{\partial p_i}
			-
		\frac{\partial F}{\partial p_i}
		\frac{\partial G}{\partial q_i}
	\right)\right] = [\hbar]\left[
		\frac{\partial F}{\partial q_i}
		\frac{\partial G}{\partial p_i}
	\right] - [\hbar]\left[
		\frac{\partial F}{\partial p_i}
		\frac{\partial G}{\partial q_i}
	\right]
\]
We know $[\hbar] = \text{kg}.\text{m}^2.\text{s}^{-1} = [q_i p_i]$,
and if we make the limits explicit as we did before, it remains
from the previous expression:
\[
	[i\hbar\{F,G\}] = [FG]
\]

On the other side:
\[
	[[F, G]] = [FG - GF]
\]
For $FG-GF$ to be well defined, it must be that $[FG] = [GF]$. And
so we're done:
\[
	\boxed{[[F, G]] = [FG] = [i\hbar\{F,G\}]} \qed
\]

\end{document}
