\documentclass[solutions.tex]{subfiles}
\title{TTM - QM - L02E03}

\begin{document}
\maketitle
\begin{exercise}\ \\
For the moment, forget that Eqs. $2.10$ give us working
definitions for $\ket{i}$ and $\ket{o}$ in terms of $\ket{u}$
and $\ket{d}$, and assume that the compoments $\alpha, \beta, \gamma$
and $\delta$ are unknown:

\begin{align*}
\ket{o} &= \alpha\ket{u} + \beta\ket{d} &
\ket{i} &= \gamma\ket{u} + \delta\ket{d} \\
\end{align*}

a) Use Eqs. $2.8$ to show that
\[
	\alpha^*\alpha = \beta^*\beta = \gamma^*\gamma = \delta^*\delta = \frac1{2}
\]

b) Use the above results and Eqs. $2.9$ to show that
\[
	\alpha^*\beta + \alpha\beta^* = \gamma^*\delta + \gamma\delta^* = 0
\]

c) Show that $\alpha^*\beta$ and $\gamma^*\delta$ must each be pure imaginary. \\

If $\alpha^*\beta$ is pure imaginary, then $\alpha$ and $\beta$ cannot both be
real. The same reasoning applies to $\gamma^*\delta$.

\end{exercise}
\hrr
\end{document}
