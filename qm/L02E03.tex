\documentclass[solutions.tex]{subfiles}

\xtitle

\begin{document}
\maketitle
\begin{exercise}
For the moment, forget that Eqs. $2.10$ give us working
definitions for $\ket{i}$ and $\ket{o}$ in terms of $\ket{u}$
and $\ket{d}$, and assume that the components $\alpha, \beta, \gamma$
and $\delta$ are unknown:

\begin{align*}
\ket{o} &= \alpha\ket{u} + \beta\ket{d} &
\ket{i} &= \gamma\ket{u} + \delta\ket{d} \\
\end{align*}

a) Use Eqs. $2.8$ to show that
\[
	\alpha^*\alpha = \beta^*\beta = \gamma^*\gamma = \delta^*\delta = \frac1{2}
\]

b) Use the above results and Eqs. $2.9$ to show that
\[
	\alpha^*\beta + \alpha\beta^* = \gamma^*\delta + \gamma\delta^* = 0
\]

c) Show that $\alpha^*\beta$ and $\gamma^*\delta$ must each be pure imaginary. \\

If $\alpha^*\beta$ is pure imaginary, then $\alpha$ and $\beta$ cannot both be
real. The same reasoning applies to $\gamma^*\delta$.

\end{exercise}
\hrr

Let's start by recalling Eqs. $2.8$, $2.9$ and $2.10$, which are
respectively:

\begin{equation}\begin{aligned}
\braket{o}{u}\braket{u}{o} &= \frac1{2} &&&
\braket{o}{d}\braket{d}{o} &= \frac1{2} \\
\braket{i}{u}\braket{u}{i} &= \frac1{2} &&&
\braket{i}{d}\braket{d}{i} &= \frac1{2} \label{L02E03:eqn:2.8}
\end{aligned}\end{equation}
\begin{equation}\begin{aligned}
\braket{o}{r}\braket{r}{o} &= \frac1{2} &&&
\braket{o}{l}\braket{l}{o} &= \frac1{2} \\
\braket{i}{r}\braket{r}{i} &= \frac1{2} &&&
\braket{i}{l}\braket{l}{i} &= \frac1{2} \label{L02E03:eqn:2.9}
\end{aligned}\end{equation}
\begin{equation}\begin{aligned}
\ket{i} &= \frac1{\sqrt2}\ket{u} + \frac{i}{\sqrt2}\ket{d} &&&
\ket{o} &= \frac1{\sqrt2}\ket{u} - \frac{i}{\sqrt2}\ket{d} \label{L02E03:eqn:2.10}
\end{aligned}\end{equation}

\hrr

a) Let's start by recalling that the inner-product in a Hilbert
space is defined between a bra and a ket, and that it should satisfy
the following two axioms:
\[
	\bra{C}\{\ket{A}+\ket{B}\} = \braket{C}{A} + \braket{C}{B}
	(\text{linearity})
\]
\[
	\braket{B}{A} = \braket{A}{B}^* (\text{complex conjugation})
\]

Furthermore, the scalar-multiplication of a ket is linear:
\[
	z\in\mathbb{C},\qquad \ket{zA} = z\ket{A}
\]

Then we can multiply $\ket{o} = \alpha\ket{u} + \beta\ket{d}$ to
the left by $\bra{u}$ to compute $\braket{u}{o}$, using the linearity
of the inner-product/scalar multiplication, and the fact that $\ket{u}$
and $\ket{d}$ are, by definition, unitary orthogonal vectors
(meaning, $\braket{u}{d} = 0$ and
$\braket{u}{u}=\braket{d}{d} = 1$)

\begin{equation*}\begin{aligned}
	\braket{u}{o} = \alpha\braket{u}{u} +\beta\braket{u}{d} = \alpha
\end{aligned}\end{equation*}

Because of the complex conjugation rule, we have
\[
	\braket{o}{u} = \braket{u}{o}^* = \alpha^*
\]

And so by Eqs. $2.8$ and the previous computation we have
\[
	\frac12 = \underbrace{\braket{o}{u}}_\alpha
		\underbrace{\braket{u}{o}}_{\alpha^*} = \alpha\alpha^* \qed
\]

The process is very similar to prove $\beta^*\beta = \gamma^*\gamma
= \delta^*\delta = \frac12$:

\begin{equation*}\begin{aligned}
	\frac12 &&=&& \braket{o}{d}\braket{d}{o} \\
	~ &&=&& (\braket{d}{o})^*\braket{d}{o} \\
	~ &&=&& \Bigl(\bra{d}\{\alpha\ket{u} + \beta\ket{d}\}\Bigr)^*
		\Bigl(\bra{d}\{\alpha\ket{u} + \beta\ket{d}\}\Bigr) \\
	~ &&=&& \Bigl(\alpha\underbrace{\braket{d}{u}}_{=0} +
			\beta\underbrace{\braket{d}{d}}_{=1}\Bigr)^*
		\Bigl(\alpha\underbrace{\braket{d}{u}}_{=0} +
			\beta\underbrace{\braket{d}{d}}_{=1}\Bigr) \\
	~ &&=&& \beta^*\beta\qed
\end{aligned}\end{equation*}

\begin{equation*}\begin{aligned}
	\frac12 &&=&& \braket{i}{u}\braket{u}{i} \\
	~ &&=&& (\braket{u}{i})^*\braket{u}{i} \\
	~ &&=&& \Bigl(\bra{u}\{\gamma\ket{u} + \delta\ket{d}\}\Bigr)^*
		\Bigl(\bra{u}\{\gamma\ket{u} + \delta\ket{d}\}\Bigr) \\
	~ &&=&& \Bigl(\gamma\underbrace{\braket{u}{u}}_{=1} +
			\delta\underbrace{\braket{u}{d}}_{=0}\Bigr)^*
		\Bigl(\gamma\underbrace{\braket{u}{u}}_{=1} +
			\delta\underbrace{\braket{u}{d}}_{=0}\Bigr) \\
	~ &&=&& \gamma^*\gamma\qed
\end{aligned}\end{equation*}

\begin{equation*}\begin{aligned}
	\frac12 &&=&& \braket{i}{d}\braket{d}{i} \\
	~ &&=&& (\braket{d}{i})^*\braket{d}{i} \\
	~ &&=&& \Bigl(\bra{d}\{\gamma\ket{u} + \delta\ket{d}\}\Bigr)^*
		\Bigl(\bra{d}\{\gamma\ket{u} + \delta\ket{d}\}\Bigr) \\
	~ &&=&& \Bigl(\gamma\underbrace{\braket{d}{u}}_{=0} +
			\delta\underbrace{\braket{d}{d}}_{=1}\Bigr)^*
		\Bigl(\gamma\underbrace{\braket{d}{u}}_{=0} +
			\delta\underbrace{\braket{d}{d}}_{=1}\Bigr) \\
	~ &&=&& \delta^*\delta\qed
\end{aligned}\end{equation*}

\hrr

b) I don't think we can conclude here without recalling the
definition of $\ket{r}$:
\[
	\ket{r} = \frac1{\sqrt2}\ket{u} + \frac1{\sqrt2}\ket{d}
\]

Let's start with a piece from Eqs. $2.9$, arbitrarily (we could
use $\braket{i}{l}\braket{l}{i} = \frac1{2}$, but I think we'd
still need the previous definition of $\ket{r}$):

\begin{equation*}\begin{aligned}
\braket{i}{r}\braket{r}{i} = \frac1{2}
\end{aligned}\end{equation*}

But:
\[
	\braket{r}{i} = \bra{r}\{\alpha + \ket{u}+\beta\ket{d}\}
		= \alpha\braket{r}{u} + \beta\braket{r}{d}
\]

And:
\[
	\braket{i}{r} = (\braket{r}{i})^* =
		(\alpha\braket{r}{u} + \beta\braket{r}{d})^*
		= \alpha^*\braket{u}{r} + \beta^*\braket{d}{r}
\]

So
\[ \braket{i}{r}\braket{r}{i} = \frac1{2} \]
\[
	\Leftrightarrow
		\Bigl(\alpha^*\braket{u}{r} + \beta^*\braket{d}{r}\Bigr)
		\Bigl(\alpha\braket{r}{u} + \beta\braket{r}{d}\Bigr) = \frac12
\]
\[
	\Leftrightarrow
	\underbrace{\alpha^*\alpha}_{=1/2}\braket{u}{r}\braket{r}{u}
	+\alpha^*\beta\braket{u}{r}\braket{r}{d}
	+\beta^*\alpha\braket{d}{r}\braket{r}{u}
	+\underbrace{\beta^*\beta}_{=1/2}\braket{d}{r}\braket{r}{d} = \frac12
\]
\[
	\Leftrightarrow
	\frac12\Bigl(
		\braket{u}{r}\braket{r}{u}
		+\braket{d}{r}\braket{r}{d}
	\Bigr)+\alpha^*\beta\braket{u}{r}\braket{r}{d}
	+\beta^*\alpha\braket{d}{r}\braket{r}{u} = \frac12
\]

Now if $\ket{r} = \rho_u\ket{u} + \rho_d\ket{d}$, then
\[
	\braket{u}{r}\braket{r}{u}+\braket{d}{r}\braket{r}{d}
	= \rho_u\rho_u^* + \rho_d\rho_d^* = 1
\]

As $\rho_u\rho_u^*$ would be the probability of $\ket{r}$ to
be up, and $\rho_d\rho_d^*$ would the probability of $\ket{r}$ to
be down, which are two orthogonal states in a two-states setting,
and so the sum of their probability must be $1$. \\

Hence the previous expression becomes:
\[
	\alpha^*\beta\braket{u}{r}\braket{r}{d}
	+\beta^*\alpha\braket{d}{r}\braket{r}{u} = 0
\]

Note that so far, we haven't needed the expression of $\ket{r}$, but
I think we don't have a choice but to use it to conclude:
\[
	\ket{r} = \frac1{\sqrt2}\ket{u} + \frac1{\sqrt2}\ket{d}
\]

So, as the coefficient are real numbers:
\[
	\braket{u}{r} = \frac1{\sqrt2} = \braket{r}{u};\qquad
	\braket{d}{r} = \frac1{\sqrt2} = \braket{r}{d}
\]

Replacing in the previous expression we have:
\[
	\alpha^*\beta\underbrace{\braket{u}{r}}_{=1/\sqrt2}
	\underbrace{\braket{r}{d}}_{=1/\sqrt2}
	+\beta^*\alpha\underbrace{\braket{d}{r}}_{=1/\sqrt2}
	\underbrace{\braket{r}{u}}_{=1/\sqrt2} = 0
\]
\[
	\Leftrightarrow
	\frac12\alpha^*\beta + \frac12\beta^*\alpha = 0
\]
\[
	\Leftrightarrow
	\boxed{\alpha^*\beta + \beta^*\alpha = 0}\qed
\]

The process is very similar to prove $\gamma^*\delta + \gamma\delta^* = 0$;
one has to start again from a Eqs. $2.9$, but this time, from another piece
involving $o$, arbitrarily:
\[
	\braket{o}{r}\braket{r}{o} = \frac1{2}
\]
\[
	\Leftrightarrow
	\Bigl(\braket{r}{o}\Bigr)^*\braket{r}{o} = \frac12
\]
\[
	\Leftrightarrow
	\Bigl(\bra{r}\{\gamma\ket{u}+\delta\ket{d}\}\Bigr)^*
	\Bigl(\bra{r}\{\gamma\ket{u}+\delta\ket{d}\}\Bigr) = \frac12
\]
\[
	\Leftrightarrow
	\Bigl(\gamma^*\braket{u}{r}+\delta^*\braket{d}{r}\Bigr)
	\Bigl(\gamma\braket{r}{u}+\delta\braket{r}{d}\Bigr) = \frac12
\]
\[
	\Leftrightarrow
	\underbrace{\gamma^*\gamma}_{=1/2}\braket{u}{r}\braket{r}{u}+
	\gamma^*\delta\braket{u}{r}\braket{r}{d}+
	\delta^*\gamma\braket{d}{r}\braket{r}{u}+
	\underbrace{\delta^*\delta}_{=1/2}\braket{d}{r}\braket{r}{d} = \frac12
\]
\[
	\Leftrightarrow
	\frac12\Bigl(\underbrace{
		\braket{u}{r}\braket{r}{u}
		+\braket{d}{r}\braket{r}{d}
	}_{=1}\Bigr)+
	\gamma^*\delta\braket{u}{r}\braket{r}{d}+
	\delta^*\gamma\braket{d}{r}\braket{r}{u} = \frac12
\]
\[
	\Leftrightarrow
	\gamma^*\delta\underbrace{\braket{u}{r}\braket{r}{d}}_{=1/2}+
	\delta^*\gamma\underbrace{\braket{d}{r}\braket{r}{u}}_{=1/2} = 0
\]
\[
	\Leftrightarrow
	\boxed{\gamma^*\delta+\delta^*\gamma = 0}\qed
\]
\hrr
c) Let's assume $\alpha\beta^*$ is a complex number of the form:
\[
	\alpha\beta^* = a + ib,\qquad (a, b) \in\mathbb{R}^2
\]
But then:
\[
	\Bigl(\alpha\beta^*\Bigr)^* = a - ib = \alpha^*\beta
\]
That's because, for two complex numbers $z=a + ib$ and $w = x + iy$, we have:
\[
	\Bigl(zw\Bigr)^* =z^*w^*
\]
Indeed:
\[
	zw = (a + ib)(x + iy) = (ax - by) + i(bx + ya)
\]
Hence:
\[
	\Bigl(zw\Bigr)^* =  (ax - by) - i(bx + ya)
\]
But:
\[
	z^*w^* = (a - ib)(x - iy) = (ax - by) - i(bx + ya)
\]
Hence the result. Back to our $\alpha$ and $\beta$, we established
in b) that:
\[
	\alpha^*\beta + \alpha\beta^* = 0
\]
Which is equivalent from our previous little proof to:
\[
	\alpha^*\beta + \Bigl(\alpha^*\beta\Bigr)^* = 0
\]
\[
	\Leftrightarrow
	(a+ib) + (a-ib) = 0 \Leftrightarrow 2a = 0
	\Leftrightarrow \boxed{a = 0}
\]
Which is the same as saying that the real part of $\alpha^*\beta$ is zero,
or that it's a pure imaginary number. The exact same argument applies
for $\gamma^*\delta$.

\end{document}
