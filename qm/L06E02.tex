\documentclass[solutions.tex]{subfiles}

\xtitle

\begin{document}
\maketitle
\begin{exercise} Show that if the two normalization conditions
of Eqs. $6.4$ are satisfied, then the state-vector of Eq. $6.5$
is automatically normalized as well. In other words, show that
for this product state, normalizing the overall state-vector does
not put any additional constraints on the $\alpha$'s and the $\beta$'s.
\end{exercise}
Recall that we're in the context of two distinct state-spaces, each
of them referring to a full-blown spin. Spin states for the first space
(Alice's) are denoted:
\[
	\alpha_u\keit{u} + \alpha_d\keit{d},\quad (\alpha_u, \alpha_d)\in\mathbb{C}^2
\]
While spin states for the second space (Bob's) are denoted:
\[
	\beta_u\ket{u} + \beta_d\ket{d},\quad (\beta_u, \beta_d)\in\mathbb{C}^2
\]
Such states are, as usual, normalized: this is the condition referred
to by Eqs. $6.4$:
\[
	\alpha_u^*\alpha_u + \alpha_d^*\alpha_d = 1;\quad
	\beta_u^*\beta_u + \beta_d^*\beta_d = 1
\]
The two underlying state spaces (complex space, but really, Hilbert
spaces) are glued by a tensor product: this allows the creation of
new state space, called the \textit{product state space}, which
states can refer to both Alice's and Bob's state in a single expression.

\begin{remark} I encourage you to have a look at how Mathematicians
formalize the notion of a tensor product of vector spaces: there is
for instance a great introductory YouTube
video\footnote{\url{https://www.youtube.com/watch?v=K7f2pCQ3p3U}}
by Michael Penn on the topic. \\

The core idea is to start with what is called a \textit{formal product
of vector spaces}, which is a new space built from the span of purely
"syntactical" combinations of elements of two (or more) vector spaces.
Equivalence classes are then used to constrain this span to be a vector
space. \\

For instance, the three following elements would be distinct elements in
the formal product of $\mathbb{R}^2$ and $\mathbb{R}^3$:
\[
	2\begin{pmatrix} 1 \\ 2 \\ \end{pmatrix}*
		\begin{pmatrix} 3 \\ 4 \\ 5\\ \end{pmatrix};\qquad
	\begin{pmatrix} 2 \\ 4 \\ \end{pmatrix}*
		\begin{pmatrix} 3 \\ 4 \\ 5\\ \end{pmatrix};\qquad
	\begin{pmatrix} 1 \\ 2 \\ \end{pmatrix}*
		\begin{pmatrix} 6 \\ 8 \\ 10\\ \end{pmatrix}
\]
But they would be identified by equivalence classes so as to
be the same element in the tensor product of $\mathbb{R}^2$ and
$\mathbb{R}^3$. We can keep identifying elements likewise until
the operations (sum, scalar product) on the formal product space
respect the properties the corresponding operations in a vector space.
\end{remark}

Here's Eq. $6.5$, the general form for such a product state, living
in the tensor product space created from Alice's and Bob's state spaces
(I've just named it $\Psi$ so as to refer to it later on):
\[
	|\Psi> = \alpha_u\beta_u\ket{uu} + \alpha_u\beta_d\ket{ud}
		+ \alpha_d\beta_u\ket{du} + \alpha_d\beta_d\ket{dd}
\]

The claim we have to prove is that this vector is naturally normalized,
from the normalization constraints imposed on the individual state spaces.

Let's start by computing the norm of product state (assuming
an ordered basis $\{\ket{uu}, \ket{ud}, \ket{du}, \ket{dd}\}$:
\[
	|\Psi|^2 = \braket{\Psi}{\Psi} =
	\begin{pmatrix}
		(\alpha_u\beta_u)^* & (\alpha_u\beta_d)^* &
		(\alpha_d\beta_u)^* & (\alpha_d\beta_d)^* \\
	\end{pmatrix}\begin{pmatrix}
		\alpha_u\beta_u \\
		\alpha_u\beta_d \\
		\alpha_d\beta_u \\
		\alpha_d\beta_d \\
	\end{pmatrix}
\]

We can develop it further, using the fact that for
$(a, b)\in\mathbb{C}$, $(ab)^* = a^*b^*$:
\begin{equation*}\begin{aligned}
	|\Psi|^2 &=&& \alpha_u^*\beta_u^*\alpha_u\beta_u
		+ \alpha_u^*\beta_d^*\alpha_u\beta_d
		+ \alpha_d^*\beta_u^*\alpha_d\beta_u
		+ \alpha_d^*\beta_d^*\alpha_d\beta_d \\
	~ &=&& \alpha_u^*\alpha_u(\underbrace{\beta_u^*\beta_u + \beta_d^*\beta_d}_{=1})
		+ \alpha_d^*\alpha_d(\underbrace{\beta_u^*\beta_u + \beta_d^*\beta_d}_{=1}) \\
	~ &=&& \underbrace{\alpha_u^*\alpha_u + \alpha_d^*\alpha_d}_{=1} \\
	~ &=&& 1 \\
\end{aligned}\end{equation*}

But the norm is axiomatically positively defined (i.e.
$(\forall\Psi\in\mathcal{H}),\,|\Psi| \geq 0$ with equality
iff $\Psi=0_{\mathcal{H}}$) so:
\[
	\boxed{|\Psi| = 1}\qed
\]

\end{document}
