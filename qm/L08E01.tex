\documentclass[solutions.tex]{subfiles}

\xtitle

\begin{document}
\maketitle
\begin{exercise} Prove that $\bm{X}$ and $\bm{D}$ are linear
operators.
\end{exercise}
The two operators are defined on a Hilbert space $\mathcal{H}$ by:
\[
	\bm{X} : \begin{pmatrix}
		\mathcal{H} &\rightarrow& \mathcal{H} \\
		\psi & \mapsto &(x \mapsto x\psi(x)) \\
	\end{pmatrix};\qquad
	\bm{D} : \begin{pmatrix}
		\mathcal{H} &\rightarrow& \mathcal{H} \\
		\psi & \mapsto & (x \mapsto \frac{d}{dx}\psi(x)) \\
	\end{pmatrix}
\]
Generally speaking, an operator $\bm{L} : \mathcal{H} \rightarrow \mathcal{H}$
is said to be linear if those two axioms are verified:
\[
	(\forall \alpha\in\mathbb{C}),\, \bm{L}(\alpha\psi) = \alpha\bm{L}(\psi)
\]
\[
	(\forall (\psi,\phi)\in\mathcal{H}^2),\, \bm{L}(\psi+\phi) = \bm{L}(\psi)+\bm{L}(\phi)
\]

\begin{remark} It's customary in quantum-mechanics to drop the parentheses when
apply an operator, i.e. $\bm{L}\psi := \bm{L}(\psi)$; I'll keep them here
for clarity. \\

Furthermore, while the authors often use "$\psi(x)$" to denote a function, I'll
be using $\psi$ instead, and reserve $\psi(x)$ to the result of the application of
$\psi$ to the variable $x$, as is usual in mathematics. \\

Finally, recall\footnote{The authors did it a bit quickly a few pages earlier,
but I've done it more carefully in
\href{https://github.com/mbivert/ttm/blob/master/qm/L03E01.pdf}{L03E01}} that
addition and scalar-multiplication are defined pointwise%
\footnote{\url{https://en.wikipedia.org/wiki/Pointwise}} on functions:
\[
	(\forall (\psi,\phi)\in\mathcal{H}),\quad
		\psi+\phi := \Bigl(x \mapsto (\psi+\phi)(x) := \psi(x)+\phi(x)\Bigr)
\]
\[
	(\forall (\psi,\alpha)\in\mathcal{H}\times\mathbb{C}),\quad
		\alpha\psi := \Bigl(x \mapsto (\alpha\psi)(x) := \alpha\psi(x)\Bigr)
\]
To ease notation, I'll use the same symbols for e.g. the addition of complex
numbers and the (pointwise) addition of functions. Don't hesitate to label
them in your mind in case of doubt.
\end{remark}

Starting with $\bm{X}$ and the first axiom; let $x\in\mathbb{R}$,
$\alpha\in\mathbb{C}$ and $\psi\in\mathcal{H}$:
\begin{equation*}\begin{aligned}
	(\bm{X}(\alpha\psi))(x) &=&& x\alpha\psi(x) \\
	~ &=&& \alpha(x\psi(x)) \\
	~ &=&& \alpha(\bm{X}\psi)(x)
\end{aligned}\end{equation*}
As this is true for any $x$, we can conclude:
\[
	\boxed{\bm{X}(\alpha\psi) = \alpha\bm{X}(\psi)}
\]
Moving on to the second axiom; let $x\in\mathbb{R}$ and $(\psi,\phi)\in\mathcal{H}^2$:
\begin{equation*}\begin{aligned}
	(\bm{X}(\psi+\phi))(x) &=&& x(\psi(x)+\phi(x)) \\
	~ &=&& x\psi(x) + x\phi(x) \\
	~ &=&& (\bm{X}(\psi))(x) + (\bm{X}(\phi))(x) \\
	~ &=&& (\bm{X}(\psi) + \bm{X}(\phi))(x) \\
\end{aligned}\end{equation*}
Again, this is true for any $x$ and thus:
\[
	\boxed{\bm{X}(\psi+\phi) = \bm{X}(\psi) + \bm{X}(\phi)}
\]

\hr

The second operator is a little more interesting; let $x\in\mathbb{R}$,
$\alpha\in\mathbb{C}$ and $\psi\in\mathcal{H}$:
\begin{equation*}\begin{aligned}
	(\bm{D}(\alpha\psi))(x) &=&& (\frac{d}{dx}\alpha\psi)(x) \\
	~ &=&& \alpha(\frac{d}{dx}\psi)(x) \\
	~ &=&& \alpha(\bm{D}\psi)(x)
\end{aligned}\end{equation*}

You may be wondering why we're allowed to shift the $\alpha$ outside of the
differential operator. Let me clarify this a little. The (real) differentiation
operator is defined as a limit:
\[
	\frac{d}{dx}\psi(x) := \lim_{\epsilon\rightarrow 0}
	\frac{\psi(x+\epsilon)-\psi(x)}{\epsilon} =: \psi'(x)
\]

So we can develop our previous equation as\footnote{Remember the
pointwise definition of the scalar multiplication of a function.}:
\[
	\frac{d}{dx}(\alpha\psi(x)) = \lim_{\epsilon\rightarrow 0}
	\frac{\alpha\psi(x+\epsilon)-\alpha\psi(x)}{\epsilon} =
	\lim_{\epsilon\rightarrow 0}
	\alpha\frac{\psi(x+\epsilon)-\psi(x)}{\epsilon}
\]
And thus all the difficulty is in knowing wether the $\alpha$ can
"jump" outside of the limit. And the answer is yes\footnote{For a proof, have a look
at \url{https://tutorial.math.lamar.edu/classes/calci/limitproofs.aspx}.}
\textit{assuming the remaining limit exists.} Meaning we can as long as the
following limit exists (it must converges to some fixed point in $\mathbb{C}$,
or equivalently, is must not diverge to $\pm\infty$):
\[
	\lim_{\epsilon\rightarrow 0}
	\frac{\psi(x+\epsilon)-\psi(x)}{\epsilon}
\]
This is equivalent to saying that we can do it as long as $\psi$ is differentiable.
In a physics context, functions are often always assumed to be differentiable
everywhere. Hence the first axiom indeed holds
for $\bm{D}$:
\[
	\boxed{\bm{D}(\alpha\psi) = \alpha\bm{D}(\psi)}
\]

There's an analogue reasoning for the second axiom: let $x\in\mathbb{R}$ and
$(\psi,\phi)\in\mathcal{H}^2$:
\begin{equation*}\begin{aligned}
	(\bm{D}(\psi+\phi))(x) &=&& \frac{d}{dx}(\psi+\phi)(x) \\
	~ &=&& (\frac{d}{dx}\psi+ \frac{d}{dx}\phi)(x) \\
	~ &=&& (\bm{D}(\psi) + \bm{D}(\phi))(x) \\
\end{aligned}\end{equation*}

Again we can rewrite the "questionable" line by expanding
the differentiation as a limit while unwrapping the
pointwise addition of functions:
\begin{equation*}\begin{aligned}
	(\frac{d}{dx}(\psi+\phi))(x) &=&& \lim_{\epsilon\rightarrow 0}
	\frac{(\psi+\phi)(x+\epsilon)-(\psi+\psi)(x)}{\epsilon} \\
	~ &=&& \lim_{\epsilon\rightarrow 0}
	\frac{\psi(x+\epsilon)+\phi(x+\epsilon)-(\psi(x)+\phi(x))}{\epsilon} \\
	~ &=&& \lim_{\epsilon\rightarrow 0}
	\frac{\Bigl(\psi(x+\epsilon)-\psi(x)\Bigr)+\Bigr(\phi(x+\epsilon)-\phi(x)\Bigl)}{\epsilon} \\
	~ &=&& \lim_{\epsilon\rightarrow 0}\Bigl(
	\frac{\psi(x+\epsilon)-\psi(x)}{\epsilon}
	+\frac{\phi(x+\epsilon)-\phi(x)}{\epsilon}\Bigr) \\
\end{aligned}\end{equation*}
Again, we can split the limit of a sum to a sum of limits\footnote{There's a proof on the
same website as before}, \textit{as long as both limits converge}. Hence
the second axioms holds as long as $\psi$ and $\phi$ are differentiable:
\[
	\boxed{\bm{D}(\psi+\phi) = \bm{D}(\psi) + \bm{D}(\phi)}
\]

\end{document}