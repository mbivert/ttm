\documentclass[solutions.tex]{subfiles}

\xtitle

\begin{document}
\maketitle
\begin{exercise} Suppose that a spin is prepared so that
$\sigma_m = +1$. The apparatus is then rotated to the $\hat{n}$
direction and $\sigma_n$ is measured. What is the probability
that the result is $+1$? Note that $\sigma_m = \sigma\cdot\hat{m}$,
using the same convention we used for $\sigma_n$.
\end{exercise}

There are essentially two ways of solving the issue. \\

The first one, and the simplest, is to observe that if we consider
$\hat{n}$ in a frame of reference where $\hat{m}$ acts as our $z-axis$,
then we're essentially in the case of our previous exercise: we've
prepared a spin in the "up" state (now corresponding to a state where
$\sigma_m = +1$), we've moved our apparatus away from $\hat{m}$ by
a a certain angle $\theta$\footnote{$\theta$ really is the angle
between $\hat{m}$ and $\hat{n}$, not some angle between $\hat{n}$ and
the "real" $z$-axis}, and we know from the previous exercise
that the probability of measuring a $+1$ after aligning our apparatus
with the $\hat{n}$ axis is now
\[
	\boxed{P(+1) = \cos^2\frac\theta2}
\]

Which is exactly what we wanted to show (the answer is given in the book
by the authors, after the exercise). \\

\hrr

I'll only draft the second approach, as I expect it to be more time
consuming\footnote{And hopefully, valid$\ldots$}. The idea is not
to rely on the previous observation, and to consider that we've
prepared to spin so that $\sigma_m = +1$, which means the state
of the system is the eigenvector corresponding to this eigenvalue,
which we know from the
\href{https://github.com/mbivert/ttm/blob/master/qm/L03E04.pdf}{previous exercise},
with $\theta_m$ the angle between the $z$-axis and $\hat{m}$, and $\phi_m$ the
angle between the $x$-axis and the projection of $\hat{m}$ on the $xy$-plane:
\[
	\ket{+1_m} = \begin{pmatrix}
		\cos(\theta_m/2) \\
		\exp(i\phi_m)\sin(\theta_m/2) \\
	\end{pmatrix} \\
\]

If we then align the apparatus in the $\hat{n}$ direction, with corresponding
$\theta_n$ / $\phi_n$ angles, \textit{which are relative to the $z$-axis, not
$\hat{m}$} , we now, by the same result, that the eigenvector corresponding
to the probability of measuring a $+1$ in the $\hat{n}$ direction is:
\[
	\ket{+1_n} = \begin{pmatrix}
		\cos(\theta_n/2) \\
		\exp(i\phi_n)\sin(\theta_n/2) \\
	\end{pmatrix} \\
\]

Then, the probability to measure a $+1$ is given, again by using the
fourth principle:
\[
	P(+1) = |\braket{+1_m}{+1_n}|^2
\]

We would then need to develop the inner-product between the two state
vectors, and find a way to identify it with the half-angle between
$\hat{n}$ and $\hat{m}$. \\

All the difficulty is then in expressing this half-angle in terms of
our four angles ($\theta_m$, $\phi_m$, $\theta_n$, $\phi_n$). I
\textit{suppose} we get some insightful elements by cleverly:

\begin{itemize}
	\item Expressing $\hat{m}$ and $\hat{n}$ both in rectangular
	coordinates;
	\item Observing that by the regular $3$-vector dot product,
	$\hat{n}\cdot\hat{m} = \norm{\hat{n}}\norm{\hat{m}}\cos\theta_{mn} =
	\cos\theta_{mn}$
	(where $\theta_{mn}$ is the angle between $\hat{m}$ and $\hat{n}$
	\item Observing that $\cos\frac{\theta_{mn}}{2} = \frac1{\sqrt2}\hat{n}\cdot(\hat{n}+\hat{m})$
	(again from the regular $3$-vector dot product, as $\hat{n}+\hat{m}$ will
	be a (non-unitary) vector bisecting $\theta_{mn}$\footnote{
	\url{https://math.stackexchange.com/a/2285989}: the parallelogram involved
	in the sum of two vectors in a rhombus.})
\end{itemize}
\end{document}