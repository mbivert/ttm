\documentclass[solutions.tex]{subfiles}

\xtitle

\begin{document}
\maketitle
\begin{exercise}
Prove that $\ket{i}$ and $\ket{o}$ satisfy all of the
conditions in Eqs. $2.7$, $2.8$ and $2.9$. Are they unique
in that respect?
\end{exercise}
\hrr

Let us recall, in order, Eqs. $2.7$, $2.8$, $2.9$,
$2.10$, which defines $\ket{i}$ and $\ket{o}$, and both
$2.5$ and $2.6$ which defines $\ket{r}$ and $\ket{l}$:

\begin{align*}
\braket{i}{o} &= 0 \\
\end{align*}
\begin{align*}
\braket{o}{u}\braket{u}{o} &= \frac1{2} &
\braket{o}{d}\braket{d}{o} &= \frac1{2} \\
\braket{i}{u}\braket{u}{i} &= \frac1{2} &
\braket{i}{d}\braket{d}{i} &= \frac1{2} \\
~ & ~ \\
\braket{o}{r}\braket{r}{o} &= \frac1{2} &
\braket{o}{l}\braket{l}{o} &= \frac1{2} \\
\braket{i}{r}\braket{r}{i} &= \frac1{2} &
\braket{i}{l}\braket{l}{i} &= \frac1{2} \\
\end{align*}
\begin{align*}
\ket{i} &= \frac1{\sqrt2}\ket{u} + \frac{i}{\sqrt2}\ket{d} &
\ket{o} &= \frac1{\sqrt2}\ket{u} - \frac{i}{\sqrt2}\ket{d} \\
\end{align*}
\begin{align*}
\ket{r} &= \frac{1}{\sqrt2}\ket{u} + \frac{1}{\sqrt2}\ket{d} &
\ket{l} &= \frac{1}{\sqrt2}\ket{u} - \frac{1}{\sqrt2}\ket{d} \\
\end{align*}

For clarity, let us recall that $\braket{u}{A}$ is the component
of $\ket{A}$ along the orthonormal vector $\ket{u}$. This is because
in an \textit{orthonormal} basis $(\ket{i})_{i\in F}$ we have:

\begin{align*}
\ket{A} &= \sum_{i\in F} \alpha_i\ket{i} \\
\Rightarrow \braket{j}{A} &= \bra{j}\sum_{i\in F} \alpha_i\ket{i}
= \sum_{i\in F} \alpha_i\underbrace{\braket{j}{i}}_{=\delta_{ij}}
= \alpha_j
\end{align*}

And to make better sense of those equations, let us recall
that $\alpha_u^*\alpha_u = \braket{A}{u}\braket{u}{A}$ is the
probability of a state vector $\ket{A} = \alpha_u\ket{u}+
\alpha_d\ket{d}$ to be measured in the state $\ket{u}$.

For Eq. $2.7$, we have

\begin{align*}
\braket{i}{o} &=
\begin{pmatrix}
\iota_u^* & \iota_d^* \\
\end{pmatrix}
\begin{pmatrix}
\omicron_u \\
\omicron_d \\
\end{pmatrix} \\
~ &= \iota_u^*\omicron_u + \iota_d^*\omicron_d \\
~ &= \frac1{\sqrt2}\frac1{\sqrt2} + \frac{-i}{\sqrt2}\frac{-i}{\sqrt2}
 = \frac1{2}-\frac{1}{2} = 0 \qed \\
\end{align*}

For Eqs. $2.8$, we can rely on the projection on an orthonormal vector:

\begin{align*}
\braket{o}{u}\braket{u}{o} &= \frac1{\sqrt2}\frac1{\sqrt2}
= \frac1{2} \qed &
\braket{o}{d}\braket{d}{o} &= \frac{i}{\sqrt2}\frac{-i}{\sqrt2}
= \frac1{2} \qed \\
\braket{i}{u}\braket{u}{i} &= \frac1{\sqrt2}\frac1{\sqrt2}
= \frac1{2} \qed &
\braket{i}{d}\braket{d}{i} &= \frac{-i}{\sqrt2}\frac{i}{\sqrt2}
= \frac1{2} \qed \\
~ & ~ \\
\end{align*}

For Eqs. $2.9$, we need to rely on the column form of the inner-product:

\begin{align*}
\braket{o}{r}\braket{r}{o} &=
\begin{pmatrix}
	\omicron_u^* & \omicron_d^* \\
\end{pmatrix}
\begin{pmatrix}
	\rho_u \\
	\rho_d \\
\end{pmatrix}
\begin{pmatrix}
	\rho_u^* & \rho_d^* \\
\end{pmatrix}
\begin{pmatrix}
	\omicron_u \\
	\omicron_d \\
\end{pmatrix} &
\braket{o}{l}\braket{l}{o} &=
\begin{pmatrix}
	\omicron_u^* & \omicron_d^* \\
\end{pmatrix}
\begin{pmatrix}
	\lambda_u \\
	\lambda_d \\
\end{pmatrix}
\begin{pmatrix}
	\lambda_u^* & \lambda_d^* \\
\end{pmatrix}
\begin{pmatrix}
	\omicron_u \\
	\omicron_d \\
\end{pmatrix} \\
~ &= (\frac1{\sqrt2}\frac1{\sqrt2}+\frac{i}{\sqrt2}\frac{1}{\sqrt2})
(\frac1{\sqrt2}\frac1{\sqrt2}+\frac{1}{\sqrt2}\frac{-i}{\sqrt2})
& ~ &= (\frac1{\sqrt2}\frac1{\sqrt2}+\frac{i}{\sqrt2}\frac{-1}{\sqrt2})
(\frac1{\sqrt2}\frac1{\sqrt2}+\frac{-1}{\sqrt2}\frac{-i}{\sqrt2}) \\
~ &= (\frac1{2}+\frac{i}{2})(\frac1{2}-\frac{i}{2})
& ~ &= (\frac1{2}-\frac{i}{2})(\frac1{2}+\frac{i}{2}) \\
~ &= \frac1{4}(1+i)(1-i)
& ~ &= \frac1{4}(1-i)(1+i) \\
~ &= \frac1{4}(1+i-i+1) = \frac1{2}\qed
& ~ &= \frac1{4}(1-i+i+1) = \frac1{2}\qed \\
\braket{i}{r}\braket{r}{i} &=
\begin{pmatrix}
	\iota_u^* & \iota_d^* \\
\end{pmatrix}
\begin{pmatrix}
	\rho_u \\
	\rho_d \\
\end{pmatrix}
\begin{pmatrix}
	\rho_u^* & \rho_d^* \\
\end{pmatrix}
\begin{pmatrix}
	\iota_u \\
	\iota_d \\
\end{pmatrix} &
\braket{i}{l}\braket{l}{i} &=
\begin{pmatrix}
	\iota_u^* & \iota_d^* \\
\end{pmatrix}
\begin{pmatrix}
	\lambda_u \\
	\lambda_d \\
\end{pmatrix}
\begin{pmatrix}
	\lambda_u^* & \lambda_d^* \\
\end{pmatrix}
\begin{pmatrix}
	\iota_u \\
	\iota_d \\
\end{pmatrix} \\
~ &= (\frac1{\sqrt2}\frac1{\sqrt2}+\frac{-i}{\sqrt2}\frac{1}{\sqrt2})
(\frac1{\sqrt2}\frac1{\sqrt2}+\frac{1}{\sqrt2}\frac{i}{\sqrt2})
& ~ &= (\frac1{\sqrt2}\frac1{\sqrt2}+\frac{-i}{\sqrt2}\frac{-1}{\sqrt2})
(\frac1{\sqrt2}\frac1{\sqrt2}+\frac{-1}{\sqrt2}\frac{i}{\sqrt2}) \\
~ &= (\frac1{2}-\frac{i}{2})(\frac1{2}+\frac{i}{2})
& ~ &= (\frac1{2}+\frac{i}{2})(\frac1{2}-\frac{i}{2}) \\
~ &= \frac1{4}(1-i)(1+i)
& ~ &= \frac1{4}(1+i)(1-i) \\
~ &= \frac1{4}(1+i+i+1) = \frac1{2}\qed
& ~ &= \frac1{4}(1+i-i+1) = \frac1{2}\qed \\
\end{align*}

\hrr

Regarding the unicity of $\ket{i}, \ket{o}$, as for $\ket{r}, \ket{l}$,
there definitely is a phase ambiguity, meaning, we can multiply either
$\ket{i}$ or $\ket{o}$ by a \textit{phase factor}, say $e^{i\theta}$,
without disturbing any of the constraints: orthogonality, probabilities,
and the resulting vectors are still unitary. \\

But as stated by the authors for $\ket{r}, \ket{l}$, measurable
quantities are independant of any phase factors. Thus, so far,
there seems to be unicity, up to such a phase factor. \\

\begin{remark} I think some sort of dimensional argument might
be required to rigorously prove that indeed there's no way to extract
more than three pairs of mutually orthogonal vectors which have a
inner-product to $1/2$, in a $\mathbb{C}$-vector space setting.
\end{remark}

\end{document}
