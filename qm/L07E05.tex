\documentclass[solutions.tex]{subfiles}

\xtitle

\begin{document}
\maketitle
\begin{exercise} a) Show that
\[
	\begin{pmatrix}
		a & 0 \\
		0 & b \\
	\end{pmatrix}^2 = \begin{pmatrix}
		a^2 & 0 \\
		0 & b^2 \\
	\end{pmatrix}
\]
b) Now, suppose
\[
	\rho = \begin{pmatrix}
		1/3 & 0        \\
		0        & 2/3 \\
	\end{pmatrix}
\]
Calculate
\[ \rho^2 \]
\[ \Tr(\rho) \]
\[ \Tr(\rho^2) \]
c) If $\rho$ is a density matrix, does it represent a pure state or
a mixed state?
\end{exercise}
The exercise is fairly trivial. \\

a) \[
	\begin{pmatrix}
		a & 0 \\
		0 & b \\
	\end{pmatrix}^2 =
	\begin{pmatrix}
		a & 0 \\
		0 & b \\
	\end{pmatrix}
	\begin{pmatrix}
		a & 0 \\
		0 & b \\
	\end{pmatrix} = \boxed{\begin{pmatrix}
		a^2 & 0 \\
		0 & b^2 \\
	\end{pmatrix}} \qed
\]

b) By application of the previous result,
\[
	\rho^2 = \begin{pmatrix}
		1/3 & 0        \\
		0        & 2/3 \\
	\end{pmatrix}^2 = \begin{pmatrix}
		(1/3)^2 & 0 \\
		0 & (2/3)^2 \\
	\end{pmatrix}= \boxed{\begin{pmatrix}
		1/9 & 0 \\
		0 & 4/9 \\
	\end{pmatrix}}
\]

Recall that there's a result alluded to by the authors in a footnote
page $195$ (section $7.2$) that the trace of an operator is the sum of
the diagonal elements of any matrix representation of this operator.
Hence:
\[
	\Tr(\rho) = \frac13 + \frac23 = \boxed{1};\qquad
	\Tr(\rho^2) = \frac19 + \frac49 = \boxed{\frac59}
\]

c) We just saw in the book some properties of density matrices.
In particular, for a pure state, and a density matrix $\rho$, we
\textit{must} have:
\[
	\rho^2 = \rho \text{ and } \Tr(\rho)^2 = 1
\]
While for a mixed state, we \textit{must} have:
\[
	\rho^2 \neq \rho \text{ and } \Tr(\rho)^2 < 1
\]

Clearly, in our case, $\boxed{\rho\text{ represents a mixed state.}}$.

\end{document}
