\documentclass[solutions.tex]{subfiles}

\xtitle

\begin{document}
\maketitle
\begin{exercise} Derive Eq. $9.7$ by plugging Eq. $9.6$ into
Eq. $9.5$.
\end{exercise}
Let's recall in order,  Eq. $9.7$, Eq. $9.6$ and Eq. $9.5$:
\[
	E = p^2/2m;\qquad
	\psi(x) = \exp{(ipx/\hbar)};\qquad
	-\frac{\hbar^2}{2m}\frac{\partial^2\psi(x)}{\partial x^2} = E\psi(x)
\]
In that last equation, the RHS could be rewritten as $\bm{H}\ket{\Psi}$,
where $\bm{H}$ is the "quantized" classical Hamiltonian corresponding
to a free particle, that is, a particle not affected by a potential energy:
the Hamiltonian is then built solely from the "quantized" kinetic energy. \\

Eq. $9.6$ (the middle one) is a solution proposal to the ODE yielded by
Eq. $9.5$ (the last one). Let's see how it goes:
\begin{equation*}\begin{aligned}
	~ &&
		-\frac{\hbar^2}{2m}\frac{\partial^2\psi(x)}{\partial x^2}
		&=&& E\psi(x) \\
	\Leftrightarrow &&
		-\frac{\hbar^2}{2m}\frac{\partial^2}{\partial x^2}\exp{(ipx/\hbar)}
		&=&& E\psi(x) \\
	\Leftrightarrow &&
		-\frac{\hbar^2}{2m}\left(\frac{ip}{\hbar}\right)^2\underbrace{
			\exp{(ipx/\hbar)}
		}_{=:\psi(x)}
		&=&& E\psi(x) \\
	\Leftrightarrow &&
		\frac{p^2}{2m}\psi(x) &=&& E\psi(x) \\
\end{aligned}\end{equation*}
And so indeed, at least as long as $\psi(x)\neq0$:
\[
	\boxed{E = p^2/2m} \qed
\]
\end{document}