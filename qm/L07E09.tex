\documentclass[solutions.tex]{subfiles}

\xtitle

\begin{document}
\maketitle
\begin{exercise} Given any Alice observable $\bm{A}$ and Bob observable
$\bm{B}$, show that for a product state, the correlation $C(A, B)$ is zero.
\end{exercise}

Recall that we're in the context of a composite system $S_{AB}$ made from two
state spaces, one corresponding to Alice, $S_A$, and one corresponding to Bob,
$S_B$, mathematically tied by a tensor product. \\

The correlation $C(\bm{A}, \bm{B})$ between two observables $\bm{A}$ and $\bm{B}$
is defined as\footnote{The authors are a bit irregular in their use of boldface
for operators; I'll try to do better, but things should be clear from the context}:
\[
	C(\bm{A}, \bm{B}) := \avg{\bm{AB}} - \avg{\bm{A}}\avg{\bm{B}}
\]

Remember that the authors proved\footnote{p106, section \textit{
$4.7$ - Expectation values}} that the expected value
$\avg{\bm{L}}$ of an observable $\bm{L}$ being
in a state $\ket{\Psi}$ is:
\[
	\avg{\bm{L}} = \bra{\Psi}\bm{L}\ket{\Psi}
\]

\hr
Here's a first derivation, where we use the following formula\footnote{p206,
section \textit{$7.5$ - Entanglement for two spins}} defined for an
observable $\bm{L}$, and a system described by a density matrix $\rho$:
\[
	\avg{\bm{L}} = \Tr(\rho\bm{L})
\]
Recall\footnote{p209, section \textit{$7.5$ - Entanglement for two spins}}
 that for any operator $\bm{A}$ and $\bm{B}$, in particular,
where $\bm{AB} \neq \bm{BA}$, we still have:
\[
	\Tr(\bm{AB}) = \Tr(\bm{BA})
\]

We also know\footnote{p202, section \textit{$7.5$ - Entanglement for two spins}}
that, because we're dealing with \textit{a} product state, this can't be
a mixed state (it cannot be expressed as a weighted sum of multiple
states), i.e if we name $\ket{\Psi}$ that (pure) product state:
\[
	\rho = \ket{\Psi}\bra{\Psi}
\]

Finally\footnote{p207, section \textit{$7.5$ - Entanglement for two spins}},
again because that product state is pure, we have $\rho^2 = \rho$. \\

It follows that:
\begin{equation*}\begin{aligned}
	C(A, B) &:=&& \avg{\bm{AB}} - \avg{\bm{A}}\avg{\bm{B}} \\
	~ &=&& Tr(\rho \bm{AB}) - \avg{\bm{A}}\avg{\bm{B}} \\
	~ &=&& Tr(\rho^2\bm{A}\bm{B}) - \avg{\bm{A}}\avg{\bm{B}} \\
	~ &=&& Tr(\rho(\bm{A}\rho\bm{B})) - \avg{\bm{A}}\avg{\bm{B}} \\
	~ &=&& \avg{\bm{A}\rho\bm{B}} - \avg{\bm{A}}\avg{\bm{B}} \\
	~ &=&& \bra{\Psi}\bm{A}\rho\bm{B}\ket{\Psi}
		- \avg{\bm{A}}\avg{\bm{B}} \\
	~ &=&& \bra{\Psi}\bm{A}\rho\bm{B}\ket{\Psi}
		- \bra{\Psi}\bm{A}\underbrace{\ket{\Psi}\bra{\Psi}}_{\rho}
			\bm{B}\ket{\Psi} \\
	~ &=&& \boxed{0} \qed \\
\end{aligned}\end{equation*}

\hr

Here's a second solution, rephrased from
\href{https://leminimumtheorique.jimdofree.com/le%C3%A7on-7/exercice-7-9/}%
{Michel Rennes's approach}. \\

We start by expressing the expectation value in terms of an inner-product
again, assuming we start in the state $\ket{\Psi}$:
\[
	\avg{\bm{AB}} = \bra{\Psi}\bm{AB}\ket{\Psi}
\]

Then, recall that $\bm{A}$ and $\bm{B}$ are two observables respectively
from Alice and Bob' state spaces, which have been extended, as previously
studied, so as to be able to act on a state vector $\ket{\Psi}$, taken from
the composite system $S_{AB}$. \\

We definitely need this to be able to express the correlation $C(\bm{A},\bm{B})$
in terms of those inner-products, for otherwise, the second terms in the
equation below applying $\bm{A}$ or $\bm{B}$ to $\ket{\Psi}$ wouldn't
make any sense:
\[
	C(\bm{A}, \bm{B}) = \bra{\Psi}\bm{AB}\ket{\Psi}
		- \bra{\Psi}\bm{A}\ket{\Psi}\bra{\Psi}\bm{B}\ket{\Psi}
\]

Hence there's an abuse of notation: with $\bm{I}_X$ being the
identity operator on the space $S_X$:
\[
	\bm{A}\ "="\ \bm{A}\otimes\bm{I}_B;\quad
	\bm{B}\ "="\ \bm{I}_A\otimes\bm{B}
\]

For clarity, I'll note $\bm{A}_A$ the observable $\bm{A}$
expressed in the system $S_A$, and similarly for $\bm{B}_B$:
\[
	\bm{A}=\bm{A}_A\otimes\bm{I}_B;\quad
	\bm{B}=\bm{I}_A\otimes\bm{B}_B
\]

Regarding $\ket{\Psi}$, this is a product state, and we know\footnote{%
p164, section \textit{$6.5$ - Product states}} that it can be expressed
as a tensor product of a state in $S_A$ and of a state in $S_B$:
\[
	\ket{\Psi} = \ket{\psi}\otimes\ket{\phi}
\]

We can then rewrite:
\begin{equation*}\begin{aligned}
	\avg{\bm{AB}} &=&& \bra{\Psi}\bm{AB}\ket{\Psi} \\
	~ &=&& \left(\bra{\psi}\otimes\bra{\phi}\right)
		\bm{AB}\left(\ket{\psi}\otimes\ket{\phi}\right) \\
	~ &=&& \left(\bra{\psi}\otimes\bra{\phi}\right)
		\bm{A}\left(
			(\bm{I}_A\otimes\bm{B}_B)\left(\ket{\psi}\otimes\ket{\phi}\right)
		\right) \\
	~ &=&& \left(\bra{\psi}\otimes\bra{\phi}\right)
		\bm{A}\left(\underbrace{\bm{I}_A\ket{\psi}}_{\ket{\psi}}
			\otimes\bm{B}_B\ket{\phi}\right) \\
	~ &=&& \left(\bra{\psi}\otimes\bra{\phi}\right)
		\left(\bm{A}_A\ket{\psi}\otimes\bm{B}_B\ket{\phi}\right) \\
\end{aligned}\end{equation*}

Where I've skipped the development for the application of
$\bm{A}$ (same procedure as for applying $\bm{B}$). Then,
observe\footnote{It would be interesting to formalized that more thoroughly.
If I'm not mistaken the idea is that the bras of $S_A\otimes S_B$ can be
expressed as a combination of one bra from $S_A$ and one bra from $S_B$.
More precisely, the bras being elements of the dual spaces, it's because
of the following (canonical) isomorphism:
$S_{AB}^* = (S_A\otimes S_B)^* \cong S_A^*\otimes S_B^*$, see for instance
\url{https://planetmath.org/tensorproductofdualspacesisadualspaceoftensorproduct}}
that $\bra{\psi}$ is an operator defined on $S_A$, and similarly for
$\bra{\phi}$ being an operator defined on $S_B$. Their tensor product is
then an operator defined on $S_{AB}$ and the usual rules for applying this
combined operator hold:
\begin{equation*}\begin{aligned}
	\avg{\bm{AB}} &=&&
		\left(\bra{\psi}\bm{A}_A\ket{\psi}\right)
		\otimes
		\left(\bra{\phi}\bm{B}_B\ket{\phi}\right) \\
	~ &=&& \avg{\bm{A}}\avg{\bm{B}} \\
\end{aligned}\end{equation*}

Hence clearly, $\boxed{C(\bm{A},\bm{B}) :=
\avg{\bm{AB}} - \avg{\bm{A}}\avg{\bm{B}} = 0}$.

% ----------------------------------------------------------------------

% I'm commenting this one out as it's still unfinished, but I
% think it'd be an interesting exercise to finish it. The idea
% is to redo the demonstration "fully", without relying much on
% existing results, which are "inlined", so to speak.
\iffalse
\hr

As stated previously, recall that $\bm{A}$ and $\bm{B}$ refer
to "upgraded" operators from the subspace via the usual procedure:
we can apply them on state vectors from $S_{AB}$ without issues. \\

Let's take an ordered basis  $\{\ket{ab}\}$ of the product space,
express $\ket{\Psi}$ in this basis, and compute $\bra{\Psi}$:
\[
	\ket{\Psi} = \sum_{ab}\psi(ab)\ket{ab};\qquad
		\bra{\Psi} = \sum_{ab}\psi^*(ab)\bra{ab}
\]
Remember that we can then expand the expected values:
\begin{equation*}\begin{aligned}
	\avg{\bm{L}} &:=&& \bra{\Psi}\bm{L}\ket{\Psi} \\
	~ &=&&
		\left(\sum_{a'b'}\psi^*(a'b')\bra{a'b'}\right)
		\bm{L}
		\left(\sum_{ab}\psi(ab)\ket{ab}\right) \\
	~ &=&& \sum_{ab,a'b'} \psi^*(a'b')\psi(ab)
		\underbrace{\bra{a'b'}\bm{L}\ket{ab}}_{L_{a'b',ab}} \\
	~ &=&& \sum_{ab,a'b'} \psi^*(a'b')\psi(ab)L_{a'b',ab} \\
\end{aligned}\end{equation*}

Furthermore, in the specific case where $\bm{L}$ is an operator
from Alice's space extended to the product state space, say, $\bm{L} = \bm{A}$,
we have:
\[
	\avg{\bm{A}} = \sum_{a,a',b}\psi^*(a'b)\psi(ab)A_{a'a}
\]
Similarly,
\[
	\avg{\bm{B}} = \sum_{b,b',a}\psi^*(ab')\psi(ab)B_{b'b}
\]

So we can rewrite the correlation as

\begin{equation*}\begin{aligned}
	C(A, B) &=&& \left(\sum_{ab,a'b'}\psi^*(a'b')(AB)_{a'b',ab}\psi(ab)\right)
		-\left(
			\sum_{a,a',b}\psi^*(a'b)\psi(ab)A_{a'a}
		\right)
		\left(
			\sum_{b,b',a}\psi^*(ab')\psi(ab)B_{b'b}
		\right)
		\\
	~ &=&&
\end{aligned}\end{equation*}

% TODO: from there I guess it's about following the reasoning on p205
% to understand how to relate rho_AB and rho_A/rho_B (see p188 / previous
% exercises for how to build composed operator via matrix), plus having
% some terms going to zero when joining the sums?
\fi

\hr

For completeness, here's one last solution, rephrased from
\href{https://onedrive.live.com/redir?resid=21D08FA0C16B93A5!28241&authkey=!AAM3H-TDeYaAbaI&ithint=file\%2cpdf}{Filip Van Lijsebetten’s approach (p52)},
which relies on the probabilistic definition of the average value. \\

Remember that the average value of an observable $\bm{L}$ is
(mathematically) defined\footnote{p105, section \textit{$4.7$ -
Expectation values}} as:
\[
	\avg{\bm{L}} := \sum_i\lambda_iP(\lambda_i)
\]

Hence:
\[
	\avg{\bm{AB}} = \sum_{ab}\lambda_{ab}P(\lambda_{ab});\quad
	\avg{\bm{A}} = \sum_a\lambda_aP(\lambda_a);\quad
	\avg{\bm{B}} = \sum_b\lambda_bP(\lambda_b)
\]

Recall that the $ab$ corresponds to all labels created by concatenating
all potential values for $a$ and $b$. This means that we'll have
$\sum_{ab} = \sum_{a,b} := \sum_a\sum_b$. Let's rewrite the correlation
$C(\bm{A},\bm{B})$:
\begin{equation*}\begin{aligned}
	C(\bm{A},\bm{B}) &:=&& \avg{\bm{AB}} - \avg{\bm{A}}\avg{\bm{B}} \\
	~ &=&& \left(\sum_{ab}\lambda_{ab}P(\lambda_{ab})\right) -
		\left(\sum_a\lambda_aP(\lambda_a)\right)
		\left(\sum_b\lambda_bP(\lambda_b)\right) \\
	~ &=&& \left(\sum_{ab}\lambda_{ab}P(\lambda_{ab})\right) -
		\left(\sum_a\lambda_aP(\lambda_a)\left(\sum_b\lambda_bP(\lambda_b)\right)\right) \\
	~ &=&&\left(\sum_{ab}\lambda_{ab}P(\lambda_{ab})\right) -
		\left(\sum_a\sum_b\lambda_aP(\lambda_a)\lambda_bP(\lambda_b)\right) \\
	~ &=&&\left(\sum_{ab}\lambda_{ab}P(\lambda_{ab})\right) -
		\left(\sum_{a,b}\lambda_a\lambda_bP(\lambda_a)P(\lambda_b)\right) \\
	~ &=&& \sum_{a,b}\Bigl(
		\lambda_{ab}P(\lambda_{ab})-
		\lambda_a\lambda_bP(\lambda_a)P(\lambda_b)
	\Bigr) \\
\end{aligned}\end{equation*}

Now the notation is a bit confusing\footnote{I could have made things
a bit clearer: for instance, we really have three different probability
distributions, one for each state involved, but they are all denoted
very similarly.}, but recall than $\lambda_{ab}$ corresponds
to the value we get for our combined state (which occurs with a probability
of $P(\lambda_{ab})$). And this precisely corresponds the fact that we
have $\lambda_a$ in the subspace $S_A$ and $\lambda_b$ in the subspace $S_B$: so
we can read it like $\lambda_{ab} \simeq \lambda_a\lambda_b$. Hence this
factors as:
\[
	C(\bm{A},\bm{B}) = \sum_{a,b}
		\lambda_{ab}\bigl(P(\lambda_{ab})-P(\lambda_a)P(\lambda_b)
	\Bigr)
\]

\begin{remark} So far, we've essentially just restated with a different
notation what we did in
\href{https://github.com/mbivert/ttm/blob/master/qm/L06E01.pdf}{L06E01}
\end{remark}

Now by definition for a product state, there is independence between
the two "events": the measurement of either $A$ or $B$ doesn't affect
the other one. That is, $P(\lambda_{ab}) = P(\lambda_a)P(\lambda_b)$%
\footnote{This is the definition of independence of events in ordinary
probability theory: \url{https://en.wikipedia.org/wiki/Independence\_(probability\_theory)\#For_events}},
hence the correlation really is zero. $\qed$

\end{document}
