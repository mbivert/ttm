\documentclass[solutions.tex]{subfiles}

\xtitle

\begin{document}
\maketitle
\begin{exercise} Prove that the four vectors $\ket{\text{sing}}$,
$\ket{T_1}$, $\ket{T_2}$, and $\ket{T_3}$ are eigenvectors of
$\bm{\sigma}\cdot\bm{\tau}$. What are their eigenvalues?
\end{exercise}
Recall the definition of those four vectors:
\begin{equation*}\begin{aligned}
	\ket{\text{sing}} &=&& \frac1{\sqrt2}\left(\ket{ud}-\ket{du}\right); &&&
	\ket{T_1} &=&& \frac1{\sqrt2}\left(\ket{ud}+\ket{du}\right) \\
	\ket{T_2} &=&& \frac1{\sqrt2}\left(\ket{uu}+\ket{dd}\right); &&&
	\ket{T_3} &=&& \frac1{\sqrt2}\left(\ket{uu}-\ket{dd}\right)
\end{aligned}\end{equation*}

And the definition of $\bm{\sigma}\cdot\bm{\tau}$:
\[
	\bm{\sigma}\cdot\bm{\tau} = \sigma_x\tau_x + \sigma_y\tau_y + \sigma_z\tau_z
\]
Again for this exercise, we won't need to explicitly use the
Pauli matrices $\sigma_i / \tau_j $. But actually, we won't even need
the multiplication table either, as we've already done most of the work
in earlier exercises. Indeed, if we want to prove that $\ket{\Psi}$ is
an eigenvector for $\bm{\sigma}\cdot\bm{\tau}$, we expect to be able to
carry some computation following this pattern:
\begin{equation*}\begin{aligned}
	(\bm{\sigma}\cdot\bm{\tau})\ket{\Psi} &=&&
		(\sigma_x\tau_x + \sigma_y\tau_y + \sigma_z\tau_z)\ket{\Psi} \\
	~ &=&& (\sigma_x\tau_x)\ket{\Psi} + (\sigma_y\tau_y)\ket{\Psi} +
		(\sigma_z\tau_z)\ket{\Psi} \\
	~ &=&& \ldots \\
	~ &=&& \lambda_\Psi\ket{\Psi} \\
\end{aligned}\end{equation*}

But we know from the book that:
\[
	\sigma_x\tau_x\ket{\text{sing}} =
	\sigma_y\tau_y\ket{\text{sing}} =
	\sigma_z\tau_z\ket{\text{sing}} = -\ket{\text{sing}}
\]

From
\href{https://github.com/mbivert/ttm/blob/master/qm/L06E07.pdf}{L06E07} that
\begin{equation*}\begin{aligned}
	\sigma_x\tau_x\ket{T_1} &=&& \frac1{\sqrt2}(\ket{du}+\ket{ud}) &=:&& T_1;\\
	\sigma_y\tau_y\ket{T_1} &=&& \frac1{\sqrt2}(\ket{du}+\ket{ud}) &=:&& T_1;\\
	\sigma_z\tau_z\ket{T_1} &=&& -\frac1{\sqrt2}(\ket{du}+\ket{ud}) &=:&& -T_1;\\
\end{aligned}\end{equation*}

And from
\href{https://github.com/mbivert/ttm/blob/master/qm/L06E08.pdf}{L06E08} that:
\begin{equation*}\begin{aligned}
	\sigma_x\tau_x\ket{T_2} &=&& \frac1{\sqrt2}(\ket{uu}+\ket{dd}) &=:&& T_2; &&&
	\sigma_x\tau_x\ket{T_3} &=&& \frac1{\sqrt2}(\ket{dd}-\ket{uu}) &=:&& -T_3; \\
	%
	\sigma_y\tau_y\ket{T_2} &=&& -\frac1{\sqrt2}(\ket{uu}+\ket{dd}) &=:&& -T_2; &&&
	\sigma_y\tau_y\ket{T_3} &=&& \frac1{\sqrt2}(\ket{uu}-\ket{dd}) &=:&& T_3; \\
	%
	\sigma_z\tau_z\ket{T_2} &=&& \frac1{\sqrt2}(\ket{uu}+\ket{dd}) &=:&& T_2; &&&
	\sigma_z\tau_z\ket{T_3} &=&& \frac1{\sqrt2}(\ket{uu}-\ket{dd}) &=:&& T_3. \\
\end{aligned}\end{equation*}

It follows that:
\begin{equation*}\begin{aligned}
	(\bm{\sigma}\cdot\bm{\tau})\ket{\text{sing}} &=&&
		\sigma_x\tau_x)\ket{\text{sing}}
		+ (\sigma_y\tau_y)\ket{\text{sing}}
		+ (\sigma_z\tau_z)\ket{\text{sing}} &=&& \boxed{-3}\ket{\text{sing}} \\
	%
	(\bm{\sigma}\cdot\bm{\tau})\ket{T_1} &=&&
		\sigma_x\tau_x)\ket{T_1}
		+ (\sigma_y\tau_y)\ket{T_1}
		+ (\sigma_z\tau_z)\ket{T_1} &=&& \boxed{+1}\ket{T_1} \\
	%
	(\bm{\sigma}\cdot\bm{\tau})\ket{T_2} &=&&
		\sigma_x\tau_x)\ket{T_2}
		+ (\sigma_y\tau_y)\ket{T_2}
		+ (\sigma_z\tau_z)\ket{T_2} &=&& \boxed{+1}\ket{T_2} \\
	%
	(\bm{\sigma}\cdot\bm{\tau})\ket{T_3} &=&&
		\sigma_x\tau_x)\ket{T_3}
		+ (\sigma_y\tau_y)\ket{T_3}
		+ (\sigma_z\tau_z)\ket{T_3} &=&& \boxed{+1}\ket{T_3}
\end{aligned}\end{equation*}
Hence, as foretold by the authors after this exercise, the triplets
share a degenerate eigenvalue ($+1$), while the singlet is associated to
a unique eigenvalue ($-3$), which justifies \textit{a posteriori} their
names.
\end{document}

