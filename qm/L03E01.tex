\documentclass[solutions.tex]{subfiles}

\xtitle

% XXX/TODO: \bm doesn't work with bare mathjax, so we
% tweak things around here. \bm from the bm package seems
% to be the recommended solution for bold math fonts.
% (https://tex.stackexchange.com/a/596)
\renewcommand{\bm}[1]{\pmb{#1}}

% XXX/TODO: similar issue for \qed
%
% FurthermoreIn case we're using \begin{proof}, \qed will be quietly
% called, without math mode, but \Box expects it.
%
% BUT, \relax\iifmode & cie aren't parsed by mathjax. Luckily
% for this file, we don't use \begin{proof} so it's "okay"
\renewcommand{\qed}{\,\boxed{\,}}

\begin{document}
\maketitle
\begin{exercise}
Prove the following: If a vector space in $N$-dimensional, an
orthonormal basis of $N$ vectors can be constructed from the eigenvectors
of a Hermitian operator.
\end{exercise}

We're here asked to prove a portion of an important theorem. I'm
going to be somehow thorough in doing so, but to save space,
I'll assume familiarity with linear algebra, up to diagonalization.
Let's start with some background. \\

This exercise is about proving one part of what the authors call the
\textit{Fundamental theorem}, also often called in the literature the
\textit{(real) Spectral theorem}. So far, we've been working more or
less explicitly in finite-dimensional spaces, but this result in
particular has a notorious analogue in infinite-dimensional Hilbert
spaces, called the \textit{Spectral theorem}\footnote{See \url{
https://ncatlab.org/nlab/show/spectral+theorem} and \url{
https://en.wikipedia.org/wiki/Spectral_theorem}}. \\

Now, I'm \textit{not} going to prove the infinite dimension
version here. There's a good reason why quantum mechanics courses
often start with spins: they don't require the generalized
results, which demands heavy mathematical machinery.
You may want to refer to F. Schuller YouTube lectures on
quantum mechanics\footnote{\url{
https://www.youtube.com/watch?v=GbqA9Xn_iM0&list=PLPH7f_7ZlzxQVx5jRjbfRGEzWY_upS5K6
}; see also the lectures notes (.pdf) made by a student (Simon Rea):
\url{https://drive.google.com/file/d/1nchF1fRGSY3R3rP1QmjUg7fe28tAS428/view}}
for a deeper mathematical development. \\

Finally, I'm going to use a mathematically inclined approach
here (definitions/theorems/proofs), and as we won't need
it, I won't be using the bra-ket notation.

\hrr

To fix things, here's the theorem we're going to prove (I'll slightly
restate it with minor adjustments later on):

\begin{theorem} Let $H : V \rightarrow V$ be a Hermitian operator on a
\textit{finite}-dimensional vector space $V$, equipped with an
inner-product\footnote{Remember, we need it to be able to talk
about orthogonality.}.

% note: using this over fbox because it fails to render "naturally"
% with mathjax.
\[
	\boxed{\text{Then, the eigenvectors of $H$ form an orthonormal basis}}
\]

Saying it otherwise, it means that a matrix representation $M_H$ of $H$
is diagonalizable, and that two eigenvectors associated with distinct
eigenvalues are orthogonal.
\end{theorem}

For clarity, let's recall a few definitions.

\begin{definition} Let $L : V \rightarrow V$ be a linear operator
on a vector space $V$ over a field $\mathbb{F}$.
We say that a \underline{non-zero} $\bm{p}\in U$ is an eigenvector for $L$,
with associated eigenvalue $\lambda\in\mathbb{F}$ whenever:
\[
	L(\bm{p}) = \lambda\bm{p}
\]
\end{definition}

\begin{remark} As this can be a source of confusion later on, note
that the definition of eigenvector/eigenvalue does \underline{not}
depend on the diagonalizability of $L$.
\end{remark}

\begin{remark} Note also that while eigenvectors must be
non-zero, no such restrictions are imposed on the eigenvalues.
\end{remark}

\begin{definition} Two vectors $\bm{p}$ and $\bm{q}$ from
a vector space $V$ over a field $\mathbb{F}$ equipped with
an inner product $\innerprod{.}{.}$ are said to be orthogonal
(with respect to the inner-product) whenever:
\[
	\boxed{\innerprod{\bm{p}}{\bm{q}} = 0_{\mathbb{F}}}
\]
\end{definition}

The following lemma will be of great use later on. Don't
let yourself be discouraged by the length of the proof: it can
literally be be shorten to just a few lines, but I'm going to
be very precise, hence very explicit, as to make the otherwise
simple underlying mathematical constructions as clear as I can.

\begin{lemma} A linear operator $L : V \rightarrow V$
on a $n\in\mathbb{N}$ dimensional vector space $V$
over the complex numbers has at least one eigenvalue.
\end{lemma}
\begin{proof}
Let's take a $\bm{v}\in V$. We assume $V$ is not trivial, that
is, $V$ isn't reduced to its zero vector $\bm{0}_V$, and so
we can always choose $\bm{v} \neq \bm{0}_V$\footnote{Note that
if $V$ is trivial, because an eigenvalue is always associated
to a non-zero vector, there are no eigenvalues/eigenvectors, and
the result is trivial.}. \\

Consider the following set of $n+1$ vectors:
\[
	\{ \bm{v}, L(\bm{v}), L^2(\bm{v}), \ldots, L^n(\bm{v}) \}
\]
where:
\[
	L^0 := \text{id}_V;\qquad L^i := \underbrace{
		L\circ L\circ\ldots\circ L
	}_{i\in\mathbb{N}\text{ times}}
\]
It's a set of $n+1$ vectors, but the space is $n$ dimensional, so
its vectors are \textit{not} all linearly independent. This
means there's a set of $(\alpha_0, \alpha_1, \ldots, \alpha_n)\in\mathbb{C}^n$
which are not all zero, such that:
\begin{equation}
	\sum_{i=0}^n \alpha_iL^i(\bm{v}) = \bm{0}_V \label{qm:L03E01:not-linindep}
\end{equation}

Here's the "subtle" part. You remember what a polynomial is,
something like: \[ x^2 - 2x + 1 \]

You know it's customary to then consider this a function of
a single variable $x$, which for instance, can be a real number:
\[
	L : \begin{pmatrix}
		\mathbb{R} & \rightarrow & \mathbb{R} \\
		x & \mapsto & x^2 - 2x + 1 \\
	\end{pmatrix}
\]
This allows you to graph the polynomial and so forth:

\begin{figure}[H]
	\centering
	\begin{tikzpicture}
		\tikzmath{
			\xmin = -3;
			\xmax = 3;
			\ymin = -3;
			\ymax = 3;
		}
		\draw[->] (\xmin-1, 0) -- (\xmax+1, 0) node[right] {$x$};
		\draw[->] (0, \ymin)   -- (0, \ymax+2) node[above] {$L(x)$};
		\draw[color=gray!30, dashed] (\xmin-1,\ymin) grid (\xmax+1,\ymax+2);
		\draw[scale=1, domain=\xmin:\xmax, smooth, samples=100, variable=\x, blue]
			plot ({\x}, {\x^2-2*\x+1});
	\end{tikzpicture}
	\caption{$L(x) = x^2 - 2x + 1$}
\end{figure}

But that's "kindergarten" polynomials so to speak. "Advanced"
polynomials are \textit{not} functions of a real variable. Rather,
we say that $L(x)$ or $L$ is a polynomial of a single
variable/indeterminate\footnote{
\url{https://en.wikipedia.org/wiki/Indeterminate\_(variable)}} $x$, where
$x$ stands for an abstract symbol. \\

The reason is that, when you say that $x$ is a real number (or
a complex number, or whatever), you tacitly assume that you can
for instance add, subtract or multiply various occurrences of
$x$, but when mathematicians study polynomials, they want to do
so without requiring additional (mathematical) structure on $x$. \\

Hence, $x$ is just a placeholder, an abstract symbol. \\

The set of polynomials of a single variable $X$ with coefficient
in a field $\mathbb{F}$ is denoted $\mathbb{F}[X]$. For instance,
$\mathbb{C}[f]$ is the set of all polynomials with complex
coefficient of a single variable $f$, say,
$P(f) = (3+2i)f^3 + 5f \in \mathbb{C}[f]$. \\

Now you'd tell me, wait a minute: if I have a $P(X) = X^2 - 2X +1$,
am I not then adding a polynomial $X^2-2X$ with an element from
the field, $1$? \\

Well, you'd be somehow right: the notation \textit{is} ambiguous, in
part inherited from the habits of kindergarten polynomials, in part
because the context often makes things clear, and perhaps
most importantly, because a truly unambiguous notation is unpractically
verbose. Actually, $X^2 - 2X +1$ is a shortcut
notation for $X^2 - 2X^1 +1 X^0$. So no: all the $+$ here are between
polynomials. \\

What does this mean that the $+$ are between polynomials? Well, most
often when you encounter $\mathbb{F}[X]$, it's actually
a shortcut for $(\mathbb{F}[X], +_{\mathbb{F}[X]}, ._{\mathbb{F}[X]})$,
which is a \textit{ring\footnote{\url{
https://en.wikipedia.org/wiki/Ring\_(mathematics)}. Note that there
is no notion of subtraction in a ring: the minus signs actually are
part of the coefficients.} of polynomials of a single indeterminate
over a field\footnote{\url{
ttps://en.wikipedia.org/wiki/Field\_(mathematics)}} $\mathbb{F}$}.
This means that  $X^2 - 2X +1$ is actually a shortcut for:
\[
	1._{\mathbb{F}[X]}X^2 +_{\mathbb{F}[X]} (-2)._{\mathbb{F}[X]}X^1
		+_{\mathbb{F}[X]} (1)._{\mathbb{F}[X]}X^0
\]

Awful, right? Hence why we often use ambiguous notations and
reasonable syntactical shortcuts. \\

The main takeaway though is that mathematicians have defined a set
of precise rules (addition, scalar multiplication, exponentiation of
an indeterminate), and that by cleverly combining such rules and
only such rules, they have obtain a bunch of interesting results,
and we want to use one of them in particular.\\

Let's get back to our equation $(\ref{qm:L03E01:not-linindep})$; let me
add some parenthesis for clarity:
\[
	\sum_{i=0}^n \left(\alpha_iL^i(\bm{v})\right) = \bm{0}_V
\]

Our goal is to transform this expression so that it involves
a polynomial in $\mathbb{C}[L]$\footnote{Remember, this means a
polynomial of a single variable $L$, with coefficient in $\mathbb{C}$.}. \\

Let's start by pulling out the $\bm{v}$ on the left-hand side as such:
\[
	\left(\underbrace{
		\sum_{i=0}^n \alpha_iL^i
	}_{=:P(L)}\right)(\bm{v}) = \bm{0}_V
\]

What's $P$ ? It's a function which takes a linear operator on $V$
and returns $\ldots$ A polynomial? But then, we don't know how to
evaluate a polynomial on a vector $\bm{v}\in V$ so there's an problem
somewhere. \\

$P$ actually returns a new \textit{linear operator on $V$}:
\[
	P : \begin{pmatrix}
		(V \rightarrow V) & \rightarrow & (V \rightarrow V) \\
		L & \mapsto & \sum_{i=0}^n \alpha_iL^i \\
	\end{pmatrix}
\]

But this means that while in (\ref{qm:L03E01:not-linindep}) the
$\sum$ was a sum of complex numbers, it's now a sum of functions,
and that $\alpha_i L_i$ went from a multiplication between complex
numbers to a scalar multiplication on a function. \\

The natural way, that is, the simplest consistent way, to do so,
is to define them pointwise\footnote{\url{
https://en.wikipedia.org/wiki/Pointwise}} for two functions
$f, g : X \rightarrow Y$, we define $(f+g) : X \rightarrow Y$
by:
\[
	(\forall x \in X),\ (f+g)(x) := f(x) + g(x)
\]
The process is similar for scalar multiplication:
\[
	(\forall x \in X), (\forall y \in Y),\ (yf)(e) := yf(e)
\]

We \textit{equip} the space of (linear) functions (on $V$) with
additional laws. All in all, $P$ is well defined\footnote{Meaning,
the laws we introduce on functions are consistent with the results
we would otherwise get without using them; you can check this out
if you want}, and that we can indeed pull the $\bm{v}$ out. \\

How then can we go from such a weird "meta" function $P$ to a
polynomial? Well, as we stated earlier, polynomials are defined
by a set of specific rules: addition, scalar multiplication,
and exponentiation of the indeterminate. \\

But if you look closely:
\begin{itemize}
	\item Our point-wise addition has the same property as
	the additions on polynomial (symmetric, existence of inverse
	elements, neutral element, etc.)
	\item Similarly for our scalar multiplication;
	\item And our rules of exponentiation on function by repeated
	application also follows the rules of exponentiation for
	an indeterminate variable.
\end{itemize}

This mean that if we squint a little, if we only look at the
expression $P(L)$ as having nothing but those properties, then
it behaves exactly as a polynomial. Hence, for all intents and
purposes, it "is" a polynomial, and we can manipulate it as such.

So we can apply the fundamental theorem of algebra\footnote{
\url{https://en.wikipedia.org/wiki/Fundamental\_theorem\_of\_algebra}},
we know that we can always factorize polynomials with
complex coefficients as such:
\[
	(\exists (c, \lambda_1, \ldots, \lambda_n)\in\mathbb{C}^{n+1}, c\neq 0),\
		P(L) = c\prod_{i=0}^n (L-\lambda_i)
\]

But don't we have a problem here? $L$ is an abstract symbol, and we're
"subtracting" it a scalar? Well, there are a few implicit elements:
\[
	P(L) = c\prod_{i=0}^n (L^1 + (-\lambda_i)L^0)
\]

Let's replace this new expression for $P(L)$ in our previous equation,
which we can do essentially re-using our previous argument: the
rules (addition, scalar multiplication, etc.) to manipulate polynomials
are "locally" consistent with the rules to manipulate our (linear) functions:
\[
	\left(c\prod_{i=0}^n (L^1-\lambda_i L^0)\right)(\bm{v}) = \bm{0}_V
\]

Note that $L^0$ becomes the identity function, and by using
the previous point-wise operations, we can reduce it to:
\[
	c\prod_{i=0}^n (L(\bm{v})-\lambda_i\text{id}_V(\bm{v})) =
	c\prod_{i=0}^n (L(\bm{v})-\lambda_i\bm{v})
		= \bm{0}_V
\]

Now, $c\neq 0$ by the fundamental theorem of algebra. So
we must have:
\[
	\prod_{i=0}^n (L(\bm{v})-\lambda_i\bm{v}) = \bm{0}_V
\]

Which implies that there's at least a $\lambda_j$ for which
\[
	L(\bm{v})-\lambda_j\bm{v} = \bm{0}_V
	\Leftrightarrow L(\bm{v}) = \lambda_j\bm{v}
\]

But we've selected $\bm{v}$ to be non-zero: $\lambda_j$
is then an eigenvalue $\lambda_j$ associated to the
eigenvector $\bm{v}$.

\end{proof}

OK; let me adjust the fundamental theorem a little bit, and let's
prove it.

\begin{theorem} Let $H : V \rightarrow V$ be a Hermitian operator on a
\textit{finite}, $n$-dimensional vector space $V$, equipped with an
inner-product $\innerprod{.}{.}$.
% again, this over fbox for mathjax
\[
	\boxed{\text{
		Then, the eigenvectors of $H$ form an orthogonal basis of $V$,
 and the associated eigenvalues are real.
	}}
\]

Saying it otherwise, it means that a matrix representation $M_H$ of $H$
is diagonalizable, and that two eigenvectors associated with distinct
eigenvalues are orthogonal.
\end{theorem}
\begin{proof} I'm assuming that this is clear for you that the eigenvectors
associated to the eigenvalues of a diagonalizable matrix makes
a basis for the vector space. Again, refer to a linear algebra course for
more. \\

Furthermore, you can refer to the book for a proof of orthogonality
of the eigenvectors associated to distinct eigenvalues\footnote{I'm
not doing it here, as I've avoided the bra-ket notation, and this
would force me to talk about dual spaces, and so on.}. \\

Note that I've included a mention to characterize the eigenvalues
as real numbers: there's already a proof in the book, but it
comes with almost no effort with the present proof, so I've
included it anyway. \\

Remains then to prove that the matrix representation $M_H$ of $H$ is
diagonalizable (and that the eigenvalues are real). Let's prove this
by induction on the dimension of the vector space. If you're not familiar
with proofs by induction, the idea is as follow:

\begin{itemize}
	\item Prove that the result is true, say, for $n=1$;
	\item Then, prove that if the result is true for $n=k$, then
	the result must be true for $n=k+1$.
	\item If the two previous points hold, then you can combine them:
	if the first point hold then by applying the second point, the
	result must be true $n=1+1=2$. But then by applying the second
	point again, it must be true that the result holds for $n=2+1=3$.
	\item And so on: the result is true $\forall n \in \mathbb{N}\backslash\{0\}$.
\end{itemize}

$\boxed{n=1}$ Then, $H$ is reduced to a $1\times 1$ matrix, containing
a single element $h$. This is trivially diagonal already, and because
$H$ is assumed to be Hermitian, the only eigenvalue $h = h^*$ is real. \\

$\boxed{\text{Induction}}$ Assume the result holds for any Hermitian
operator $H : W \rightarrow W$ on a $k$-dimensional vector space $W$
over $\mathbb{C}$. \\

Let $V$ be a $k+1$-dimensional vector space over $\mathbb{C}$.
By our previous lemma, $H : V\rightarrow V$ must have at
least one eigenvalue $\lambda\in\mathbb{C}$ associated to an eigenvector
$\bm{v}\in V$. \\

Pick $\{\bm{v}, \bm{v}_2, \ldots, \bm{v}_{k+1} \} \subset V$
so that $\{\bm{v}, \bm{v}_2, \bm{v}_3, \ldots, \bm{v}_{k+1} \}$ is an
(ordered) basis of $V$\footnote{Start with $W=\{ \bm{v} \}$, and progressively
augment it with elements of $V$ so that all elements in $W$ are linearly
independent. If we can't select such elements no more, this mean we've
got a basis. Ordering naturally follows from the iteration steps.}. \\

Apply the Gram-Schmidt procedure\footnote{
\url{https://en.wikipedia.org/wiki/Gram\%E2\%80\%93Schmidt\_process}} to
extract from it an (ordered) orthonormal basis
$\{ \bm{b}_1, \bm{b}_2, \ldots, \bm{b}_{k+1} \}$ of $V$; note that
by construction:
\[
	\bm{b}_1 = \frac{\bm{v}}{\norm{\bm{v}}}
\]

That's to say, $\bm{b}_1$ is still an eigenvector for $\lambda$\footnote{
$H(\bm{b}_1) = H(\bm{v} / \norm{\bm{v}})$, by linearity of $H$,
this is equal to $\frac1{\norm{\bm{v}}}H(\bm{v})$. But $\bm{v}$ is
an eigenvector for an eigenvalue $\lambda$, so this is equal to
$\frac{\lambda}{\norm{v}}\bm{v} = \lambda\frac{\bm{v}}{\norm{\bm{v}}}
= \lambda\bm{b_1}$}. \\

Now we're trying to understand what's the matrix representation $D_H$ of $H$,
in this orthonormal basis. If you've taken the blue pill, you know how to
"read" a matrix:
\[
	D_H = \begin{pmatrix}
		\begin{pmatrix}
			\bigg\vert \\
			H(\bm{b}_1) \\
			\bigg\vert \\
		\end{pmatrix} &
		\begin{pmatrix}
			\bigg\vert \\
			H(\bm{b}_2) \\
			\bigg\vert \\
		\end{pmatrix} &
		\ldots &
		\begin{pmatrix}
			\bigg\vert \\
			H(\bm{b}_{k+1}) \\
			\bigg\vert \\
		\end{pmatrix}
	\end{pmatrix}
\]

OK; let's start by what we know: $\bm{b}_1$ is an eigenvector for $H$ associated
to $\lambda$, meaning:
\[
	H(\bm{b_1}) = \lambda\bm{b}_1
		= \lambda\bm{b}_1 + \sum_{i=2}^{k+1} 0 \times \bm{b}_i
		= \begin{pmatrix}
			\lambda \\
			0       \\
			\vdots \\
			0      \\
		\end{pmatrix}
\]

Rewrite $D_H$ accordingly, and break it into blocks:
\[
	D_H = \begin{pmatrix}
		\begin{matrix}
			\lambda \\
			0 \\
			\vdots \\
			0
		\end{matrix} &
		\begin{pmatrix}
			\bigg\vert \\
			H(\bm{b}_2) \\
			\bigg\vert \\
		\end{pmatrix} &
		\ldots &
		\begin{pmatrix}
			\bigg\vert \\
			H(\bm{b}_{k+1}) \\
			\bigg\vert \\
		\end{pmatrix}
	\end{pmatrix} = \begin{pmatrix}\begin{array}{c|c}
		\lambda & \qquad A \qquad\\
		\hline
		0       & \qquad ~ \qquad \\
		\vdots  & \qquad C \qquad \\
		0       & \qquad ~ \qquad \\
	\end{array}\end{pmatrix}
\]

Where A is a $1\times k$ matrix (a row vector), and C
 a $k\times k$ matrix. But then $H$ is Hermitian, which
means its matrix representation obeys:
\[
	D_H = (D_H^T)^* = D_H^\dagger
\]

This implies first that $\lambda = \lambda^*$, i.e $\lambda$ is
real, and we'll see shortly, can be considered an eigenvalue,
as we can transform $D_H$ in a diagonal matrix with $\lambda$
on the diagonal. \\

Second, $A^\dagger = (0\ 0\ \ldots\ 0) = A$, i.e:
\[
	\begin{pmatrix}\begin{array}{c|c}
		\lambda & 0\quad\ldots\quad 0     \\
		\hline
		0       & \qquad ~ \qquad \\
		\vdots  & \qquad C \qquad \\
		0       & \qquad ~ \qquad \\
	\end{array}\end{pmatrix}
\]

Third, $C = C^\dagger$. But then, C is a $k\times k$ Hermitian
matrix, corresponding to a Hermitian operator in a $k$-dimensional
vector space. Using the induction assumption, it is diagonalizable,
with real valued eigenvalues. Hence $D_H$ is diagonalizable, and
all its eigenvalues are real. \\

\end{proof}
\end{document}
