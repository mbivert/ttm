\documentclass[solutions.tex]{subfiles}

\xtitle

\begin{document}
\maketitle
\begin{exercise} Write the tensor product $I\otimes\tau_x$ as
a matrix, and apply that matrix to each of the $\ket{uu}$, $\ket{ud}$,
$\ket{du}$, and $\ket{dd}$ column vectors. Show that Alice's half
of the state-vector is unchanged in each case. Recall that $I$ is the
$2\times2$ unit matrix.
\end{exercise}
Recall that $\tau_x$ is a Pauli matrix, while $I$ really is the
identity matrix:
\[
	\tau_x = \begin{pmatrix}
		0 & 1 \\
		1 & 0 \\
	\end{pmatrix};\qquad I=\begin{pmatrix}
		1 & 0 \\
		0 & 1 \\
	\end{pmatrix}
\]

We saw two different ways of building $I\otimes\tau_x$. Let's
start with the first one: consider the usual ordered basis of the
underlying composite space: $\{ \ket{uu}, \ket{ud}, \ket{du}, \ket{dd} \}$.
Then, the elements of the matrix representation of $I\otimes\tau_x$
in this basis are given by:
\[
	(I\otimes\tau_x)_{ab,cd} = \bra{ab}(I\otimes\tau_x)\ket{cd}
\]

We can then use the multiplication table from either the appendix or from
\href{https://github.com/mbivert/ttm/blob/master/qm/L06E04.pdf}{L06E04},
where, remember, $\tau_x$ in this multiplication table was a shortcut
notation for $I\otimes\tau_x$.
\begin{equation*}\begin{aligned}
	\tau_x\ket{uu} &=&& \ket{ud}; &&&
	\tau_x\ket{ud} &=&& \ket{uu} \\
	\tau_x\ket{du} &=&& \ket{dd}; &&&
	\tau_x\ket{dd} &=&& \ket{du} \\
\end{aligned}\end{equation*}

And we're now ready to evaluate the operator's matrix form:
\begin{equation*}\begin{aligned}
	I\otimes\tau_x &"="&& \begin{pmatrix}
		\bra{uu}(I\otimes\tau_x)\ket{uu} & \bra{uu}(I\otimes\tau_x)\ket{ud} &
			\bra{uu}(I\otimes\tau_x)\ket{du} & \bra{uu}(I\otimes\tau_x)\ket{dd} \\
		\bra{ud}(I\otimes\tau_x)\ket{uu} & \bra{ud}(I\otimes\tau_x)\ket{ud} &
			\bra{ud}(I\otimes\tau_x)\ket{du} & \bra{ud}(I\otimes\tau_x)\ket{dd} \\
		\bra{du}(I\otimes\tau_x)\ket{uu} & \bra{du}(I\otimes\tau_x)\ket{ud} &
			\bra{du}(I\otimes\tau_x)\ket{du} & \bra{du}(I\otimes\tau_x)\ket{dd} \\
		\bra{dd}(I\otimes\tau_x)\ket{uu} & \bra{dd}(I\otimes\tau_x)\ket{ud} &
			\bra{dd}(I\otimes\tau_x)\ket{du} & \bra{dd}(I\otimes\tau_x)\ket{dd} \\
	\end{pmatrix} \\
	~ &"="&& \begin{pmatrix}
		\braket{uu}{ud} & \braket{uu}{uu} & \braket{uu}{dd} & \braket{uu}{du} \\
		\braket{ud}{ud} & \braket{ud}{uu} & \braket{ud}{dd} & \braket{ud}{du} \\
		\braket{du}{ud} & \braket{du}{uu} & \braket{du}{dd} & \braket{du}{du} \\
		\braket{dd}{ud} & \braket{dd}{uu} & \braket{dd}{dd} & \braket{dd}{du} \\
	\end{pmatrix} \\
	~ &"="&& \boxed{\begin{pmatrix}
		0 & 1 & 0 & 0 \\
		1 & 0 & 0 & 0 \\
		0 & 0 & 0 & 1 \\
		0 & 0 & 1 & 0 \\
	\end{pmatrix}} \\
\end{aligned}\end{equation*}

\hrr

Let's move on to the second way, which consists in using Eq. $7.6$ of the book:
\[
	A\otimes B = \begin{pmatrix}
		A_{11}B & A_{12}B \\
		A_{21}B & A_{22}B \\
	\end{pmatrix}
\]

Which then yields:
\begin{equation*}\begin{aligned}
	I\otimes\tau_x &"="&& \begin{pmatrix}
		1\times\tau_x & 0\times\tau_x \\
		0\times\tau_x & 1\times\tau_x \\
	\end{pmatrix} \\
	~ &"="&& \begin{pmatrix}
		\begin{pmatrix}
			0 & 1 \\
			1 & 0 \\
		\end{pmatrix} & \begin{pmatrix}
			0 & 0 \\
			0 & 0 \\
		\end{pmatrix} \\
		\begin{pmatrix}
			0 & 0 \\
			0 & 0 \\
		\end{pmatrix} & \begin{pmatrix}
			0 & 1 \\
			1 & 0 \\
		\end{pmatrix} \\
	\end{pmatrix} \\
	~ &"="&& \boxed{\begin{pmatrix}
		0 & 1 & 0 & 0 \\
		1 & 0 & 0 & 0 \\
		0 & 0 & 0 & 1 \\
		0 & 0 & 1 & 0 \\
	\end{pmatrix}} \\
\end{aligned}\end{equation*}

Which is exactly what we've found earlier, albeit less tediously.

\hrr

In our usual ordered basis $\{ \ket{uu}, \ket{ud}, \ket{du}, \ket{dd} \}$,
the column representations of the basis vectors are as follow:
\[
	\ket{uu} = \begin{pmatrix}
		1 \\
		0 \\
		0 \\
		0 \\
	\end{pmatrix};\quad
	\ket{ud} = \begin{pmatrix}
		0 \\
		1 \\
		0 \\
		0 \\
	\end{pmatrix};\quad
	\ket{du} = \begin{pmatrix}
		0 \\
		0 \\
		1 \\
		0 \\
	\end{pmatrix};\quad
	\ket{dd} = \begin{pmatrix}
		0 \\
		0 \\
		0 \\
		1 \\
	\end{pmatrix}
\]
\begin{remark} Remember than the column notation is merely a syntactical
shortcut over linear combinations of the basis vectors:
\[
	\begin{pmatrix}
		a \\
		b \\
		c \\
		d \\
	\end{pmatrix} := a\ket{uu} + b\ket{ud} + c\ket{du} + d\ket{dd}
\]
\end{remark}
\begin{remark} Note that we could also have used, as the authors did
in the book, Eq. $7.6$ to derive them.
\end{remark}

Then it's just a matter of computing some elementary matrix$\times$vector
products. As a shortcut, one can also recall from one's linear algebra class
than such products, when they involve basis vectors, are simply a matter
of extracting the columns of the matrix (which is fairly trivial to see):
\[
	(I\otimes\tau_x)\ket{uu} = \begin{pmatrix}
		0 \\
		1 \\
		0 \\
		0 \\
	\end{pmatrix} = \ket{ud};\quad
	(I\otimes\tau_x)\ket{ud} = \begin{pmatrix}
		1 \\
		0 \\
		0 \\
		0 \\
	\end{pmatrix} = \ket{uu};
\]
\[
	(I\otimes\tau_x)\ket{du} = \begin{pmatrix}
		0 \\
		0 \\
		0 \\
		1 \\
	\end{pmatrix} = \ket{dd};\quad
	(I\otimes\tau_x)\ket{dd} = \begin{pmatrix}
		0 \\
		0 \\
		1 \\
		0 \\
	\end{pmatrix} = \ket{du}
\]
\begin{remark} Naturally, this is consistent with the multiplication
table we've recalled earlier; and Alice's part of the state is indeed
kept unchanged, as expected.
\end{remark}

\end{document}
