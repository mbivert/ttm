\documentclass[solutions.tex]{subfiles}

\xtitle

\begin{document}
\maketitle
\begin{exercise} Consider the following states
\begin{equation*}\begin{aligned}
	\ket{\psi_1} &=&& \frac12\left(\ket{uu}+\ket{ud}+\ket{du}+\ket{dd}\right) \\
	\ket{\psi_2} &=&& \frac1{\sqrt2}\left(\ket{uu}+\ket{dd}\right) \\
	\ket{\psi_3} &=&& \frac15\left(3\ket{uu}+4\ket{ud}\right) \\
\end{aligned}\end{equation*}
For each one, calculate Alice's density matrix, and Bob's density matrix.
Check their properties.
\end{exercise}
Let's recall first the definition of the matrix elements for Alice's density
matrix, and second, by symmetry, Bob's:
\[
	\rho_{a'a} = \sum_b\psi^*(a,b)\psi(a',b);\qquad
	\rho_{b'b} = \sum_a\psi^*(a,b)\psi(a,b')
\]
\hr
Let's start with $\ket{\psi_1}$. We know Alice's matrix must be of the form:
\[
	\rho_A = \begin{pmatrix}
		\rho_{uu} & \rho_{ud} \\
		\rho_{du} & \rho_{dd} \\
	\end{pmatrix}
\]
And so must be Bob's actually. Filling in with our previous formulas, we
obtain:
\begin{equation*}\begin{aligned}
	\rho_{1A} &=&& \begin{pmatrix}
		\psi_1^*(u,u)\psi_1(u,u) + \psi_1^*(u,d)\psi_1(u,d) &
			\psi_1^*(d,u)\psi_1(u,u) + \psi_1^*(d,d)\psi_1(u,d) \\
		\psi_1^*(u,u)\psi_1(d,u) + \psi_1^*(u,d)\psi_1(d,d)&
			\psi_1^*(d,u)\psi_1(d,u) + \psi_1^*(d,d)\psi_1(d,d) \\
	\end{pmatrix} \\
	~ &=&& \begin{pmatrix}
		(1/2)(1/2) + (1/2)(1/2) & (1/2)(1/2) + (1/2)(1/2) \\
		(1/2)(1/2) + (1/2)(1/2) & (1/2)(1/2) + (1/2)(1/2) \\
	\end{pmatrix} \\
	~ &=&& \boxed{\begin{pmatrix}
		1/2 & 1/2 \\
		1/2 & 1/2 \\
	\end{pmatrix}}
\end{aligned}\end{equation*}
Where, remember, the wave function's values correspond to the basis
vector coefficients, which are all $1/2$ here. By symmetry, we would
obtain exactly the same matrix for Bob:
\[
	\rho_{1B} = \boxed{\begin{pmatrix}
		1/2 & 1/2 \\
		1/2 & 1/2 \\
	\end{pmatrix}}
\]
Let's check the density matrices properties:
\begin{itemize}
	\item Clearly, $\rho_{1A}=\rho_{1B}$ is Hermitian;
	\item Its trace is $1/2+1/2=1$, as expected;
	\item Let's compute its square:
	\[
		\rho_{1A}^2=\rho_{1B}^2 = \begin{pmatrix}
			1/2 & 1/2 \\
			1/2 & 1/2 \\
		\end{pmatrix}\begin{pmatrix}
			1/2 & 1/2 \\
			1/2 & 1/2 \\
		\end{pmatrix} = \begin{pmatrix}
			1/2 & 1/2 \\
			1/2 & 1/2 \\
		\end{pmatrix} = \rho_{1A}=\rho_{1B}
	\]
	And $\Tr(\rho_{1A}^2) = \Tr(\rho_{1B}^2) = 1$, from which we
	can conclude that $\boxed{\text{ $\psi_1$ is a pure state.}}$
	\item Without having to compute them explicitly, this implies that its eigenvalues must be $0$ and $1$.
\end{itemize}
Let's compute the eigenvalues by partially diagonalizing the matrix
anyway for practice: an eigenvector $\ket{\lambda}$ is tied to an
eigenvalue $\lambda$ by:
\[
	\rho_{1A}\ket{\lambda} = \lambda\ket{\lambda}
	\Leftrightarrow \rho_{1A}\ket{\lambda} - \lambda\ket{\lambda} = 0
	\Leftrightarrow (\rho_{1A}-\lambda I)\ket{\lambda} = 0
\]
Because an eigenvector is by definition non-zero, this implies that
$\rho_{1A}-\lambda I$ must be non-invertible\footnote{For otherwise,
multiply both sides of the equation by its inverse: LHS is equal to
$\ket{\lambda}$ while the RHS is still equal to 0}. This implies
that:
\[
	\det(\rho_{1A}-\lambda I) = 0
	\Leftrightarrow 0 = \begin{vmatrix}
		1/2-\lambda & 1/2 \\
		1/2 & 1/2-\lambda \\
	\end{vmatrix} = \left(\frac12-\lambda)^2-\frac12^2\right)
	= (\frac12-\lambda-\frac12)(\frac12-\lambda+\frac12)
	= \lambda(\lambda-1)
\]
\[
	\Leftrightarrow \boxed{\begin{cases}
		\lambda = 0 \\
		\lambda = 1 \\
	\end{cases}}
\]
As expected.

\hr
Let's move on to $\psi_2$: by a similar reasoning as before we have:
\begin{equation*}\begin{aligned}
	\rho_{2A} &=&& \begin{pmatrix}
		\psi_2^*(u,u)\psi_2(u,u) + \psi_2^*(u,d)\psi_2(u,d) &
			\psi_2^*(d,u)\psi_2(u,u) + \psi_2^*(d,d)\psi_2(u,d) \\
		\psi_2^*(u,u)\psi_2(d,u) + \psi_2^*(u,d)\psi_2(d,d)&
			\psi_2^*(d,u)\psi_2(d,u) + \psi_2^*(d,d)\psi_2(d,d) \\
	\end{pmatrix} \\
	~ &=&& \begin{pmatrix}
		(1/\sqrt2)(1/\sqrt2) + (0)(0) & (0)(1/\sqrt2) + (1/\sqrt2)(0) \\
		(1/\sqrt2)(0) + (0)(1/\sqrt2) & (0)(0) + (1/\sqrt2)(1/\sqrt2) \\
	\end{pmatrix} \\
	~ &=&& \boxed{\begin{pmatrix}
		1/2 & 0 \\
		0 & 1/2 \\
	\end{pmatrix}}
\end{aligned}\end{equation*}
Again, by a symmetry argument, we can already conclude that
$\rho_{2B} = \rho_{2A}$ (the idea is that you can swap the labels
corresponding to Bob and Alice in the description of the state $\psi_2$
and by reordering the terms, you see that the state is unchanged). \\

Finally, let's check the density matrices properties:
\begin{enumerate}
	\item Clearly Hermitian;
	\item $\Tr(\rho_{2A}) = 1/2+1/2 = 1$;
	\item Let's compute the square to determine the state quality:
	\[ \rho_{2A}^2 = \begin{pmatrix}
		1/4 & 0 \\
		0 & 1/4 \\
	\end{pmatrix} \neq \rho_{2A}
	\] and $\Tr(\rho_{2A}^2) = 1/2 < 1$: $\boxed{\text{$\psi_2$ is
	a mixed state}}$;
	\item The matrix is diagonal: clearly, all its eigenvalue (there's
	a single degenerate eigenvalue $1/2$) are positive and $\leq 1$.
\end{enumerate}

\hr

Moving on to the last one. Observe that this time, there is not symmetry
between Alice and Bob matrices, so we'll have to compute them both.

\begin{equation*}\begin{aligned}
	\rho_{3A} &=&& \begin{pmatrix}
		\psi_3^*(u,u)\psi_3(u,u) + \psi_3^*(u,d)\psi_3(u,d) &
			\psi_3^*(d,u)\psi_3(u,u) + \psi_3^*(d,d)\psi_3(u,d) \\
		\psi_3^*(u,u)\psi_3(d,u) + \psi_3^*(u,d)\psi_3(d,d)&
			\psi_3^*(d,u)\psi_3(d,u) + \psi_3^*(d,d)\psi_3(d,d) \\
	\end{pmatrix} \\
	~ &=&& \begin{pmatrix}
		(3/5)(3/5) + (4/5)(4/5) & (0)(3/5) + (0)(4/5) \\
		(3/5)(0) + (4/5)(0) & (0)(0) + (0)(0) \\
	\end{pmatrix} \\
	~ &=&& \begin{pmatrix}
		9/25+16/25 & 0 \\
		0 & 0 \\
	\end{pmatrix} \\
	&=&& \boxed{\begin{pmatrix}
		1 & 0 \\
		0 & 0 \\
	\end{pmatrix}}
\end{aligned}\end{equation*}

Regarding density matrices properties:
\begin{enumerate}
	\item Hermitian;
	\item $\Tr(\rho_{3A}) = 1+0 = 1$;
	\item $\rho_{3A}^2 = \rho_{3A}$ : $\boxed{\text{ $\psi_3$ is a pure
	state}}$;
	\item This is confirmed by the eigenvalues $1$ and $0$ (matrix
	trivially diagonal).
\end{enumerate}

Remains Bob's matrix!

\begin{equation*}\begin{aligned}
	\rho_{3B} &=&& \begin{pmatrix}
		\psi_3^*(u,u)\psi_3(u,u) + \psi_3^*(d,u)\psi_3(d,u) &
			\psi_3^*(u,u)\psi_3(u,d) + \psi_3^*(d,u)\psi_3(d,d) \\
		\psi_3^*(u,d)\psi_3(u,u) + \psi_3^*(d,d)\psi_3(d,u)&
			\psi_3^*(u,d)\psi_3(u,d) + \psi_3^*(d,d)\psi_3(d,d) \\
	\end{pmatrix} \\
	~ &=&& \begin{pmatrix}
		(3/5)(3/5) + (0)(0) & (3/5)(4/5) + (0)(0) \\
		(4/5)(3/5) + (0)(0) & (4/5)(4/5) + (0)(0) \\
	\end{pmatrix} \\
	&=&& \boxed{\frac{1}{25}\begin{pmatrix}
		9 & 12 \\
		12 & 16 \\
	\end{pmatrix}}
\end{aligned}\end{equation*}

One last time, let's check its density matrices properties:
\begin{enumerate}
	\item Clearly Hermitian;
	\item $\Tr(\rho_{3B}) = 9/25 + 16/25 = 1$;
	\item Let's square it to determine the state quality:
	\begin{equation*}\begin{aligned}
		\rho_{3B}^2 &=&& \frac{1}{25^2}\begin{pmatrix}
			9\times9 + 12\times12 & 9\times12 + 12\times16 \\
			12\times9 + 16\times12 & 12\times12 + 16\times16 \\
		\end{pmatrix} \\
		~ &=&& \frac{1}{25^2}\begin{pmatrix}
			81 + 100 + 40 + 4 & 90+18 + 100 + 80 + 12 \\
			90+18 + 100 + 80 + 12 & 100 + 40 + 4 + 100 + 120 + 36 \\
		\end{pmatrix} \\
		~ &=&& \frac{1}{25^2}\begin{pmatrix}
			225 & 300 \\
			300 & 400 \\
		\end{pmatrix} \\
		~ &=&& \frac{1}{25^2}\begin{pmatrix}
			(4\times2+1)\times25 & 3\times4\times25 \\
			3\times4\times25 & 4\times4\times25 \\
		\end{pmatrix} \\
		~ &=&& \frac1{25}\begin{pmatrix}
			9 & 12 \\
			12 & 16 \\
		\end{pmatrix} = \rho_{3B}
	\end{aligned}\end{equation*}
	Thus $\Tr(\rho_{3B}^2) = \Tr(\rho_{3B}) = 1$ and
	$\boxed{\text{ $\psi_3$ is a pure state}}$;
	\item This implies again that its eigenvalues must be $0$ and $1$
\end{enumerate}
Let's compute the eigenvalues for practice, going a bit faster this
time:
\[
	\begin{vmatrix}
		9/25-\lambda & 12/25 \\
		12/25 & 16/25-\lambda \\
	\end{vmatrix} = 0 \Leftrightarrow
	\left(
		\left(\frac{9}{25}-\lambda\right)
			\left(\frac{16}{25}-\lambda\right)
		- \left(\frac{12}{25}\right)^2\right) = 0
\]
\[
	\Leftrightarrow \lambda^2 - \lambda +
		\frac{9\times16}{25^2} - \left(\frac{12}{25}\right)^2 = 0
\]
\[
	\Leftrightarrow \lambda^2 - \lambda +
		\frac{3\times3\times4\times4}{25^2}
		-\frac{3\times4\times3\times4}{25^2} = 0
\]
\[
	\Leftrightarrow \lambda(\lambda-1) = 0
\]
\[
	\Leftrightarrow \boxed{\begin{cases}
		\lambda = 0 \\
		\lambda = 1 \\
	\end{cases}}
\]
\end{document}
