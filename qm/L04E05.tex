\documentclass[solutions.tex]{subfiles}

\xtitle

\begin{document}
\maketitle
\begin{exercise} Take any unit $3$-vector $\bm{n}$ and form
the operator
\[
	H = \frac{\hbar\omega}2\sigma\cdot\bm{n}
\]
Find the energy eigenvalues and eigenvectors by solving the
time-independent Schr\"odinger equation. Recall that Eq. $3.23$
gives $\sigma\cdot\bm{n}$ in component form.
\end{exercise}
Let's recall Eq. $3.23$, which is general form of the spin $3$-vector operator:
\[
	\sigma_n = \sigma\cdot\bm{n} = \begin{pmatrix}
		n_z          & (n_x - in_y) \\
		(n_x + in_y) & -n_z         \\
	\end{pmatrix}
\]
And the \textit{time-independent Schr\"odinger equation}\footnote{That's
quite a fancy name for describing the eigenvectors of an
operator, by comparison with the "iconic" Schr\"odinger equation$\ldots$}:
\[
	H\ket{E_j} = E_j\ket{E_j}
\]
In an \href{https://github.com/mbivert/ttm/blob/master/qm/L03E04.pdf}{earlier
exercise (L03E04)}, we actually diagonalized $\sigma_n$: this gave us two
eigenvalues $+1$ and $-1$, and two eigenvectors:
\[
	\ket{+1} = \begin{pmatrix}
		\cos(\theta/2) \\
		\exp(i\phi)\sin(\theta/2) \\
	\end{pmatrix};\qquad
	\ket{-1} = \begin{pmatrix}
		-\sin(\theta/2) \\
		\exp(i\phi)\cos(\theta/2) \\
	\end{pmatrix}
\]
Where $\bm{n}$ was a regular unitary $3$-vector expressed in spherical
coordinates:
\[
	\bm{n} = \begin{pmatrix}
		\sin\theta\cos\phi \\
		\sin\theta\sin\phi \\
		\cos\theta
	\end{pmatrix}
\]
Let's see how we can leverage this previous work to our advantage:
such an $\bm{n}$ vector still fit our purpose here. Furthermore,
we know that the eigenvalues of $\sigma_n$ are the only solutions to:

\[ \sigma_n\ket{F_j} = F_j\ket{F_j} \]

But if we multiply both sides of this equation by $\dfrac{\hbar\omega}2$,
we get exactly the equation we want to solve:
\[
	\underbrace{\frac{\hbar\omega}2\sigma_n}_{H}\ket{F_j} =
		\left(\frac{\hbar\omega}2F_j\right)\ket{F_j}
\]

Multiplying the equation by a constant doesn't change the eigenvectors:
they still are the only solutions, but the associated eigenvalues are
now different:
\[
	\boxed{\lambda_1 = \frac{\hbar\omega}2;\qquad
		\ket{\lambda_1} = \begin{pmatrix}
		\cos(\theta/2) \\
		\exp(i\phi)\sin(\theta/2) \\
	\end{pmatrix}}
\]\[
	\boxed{\lambda_2 = -\frac{\hbar\omega}2;\qquad
		\ket{\lambda_2} = \begin{pmatrix}
		-\sin(\theta/2) \\
		\exp(i\phi)\cos(\theta/2) \\
	\end{pmatrix}}
\]

\end{document}
